
The Gauss Circle Problem concerns finding asymptotics for the number of lattice point
lying inside a circle in terms of the radius of the circle.
The heuristic that the number of points is very nearly the area of the circle is
surprisingly accurate.
This seemingly simple problem has prompted new ideas in many areas of number theory and
mathematics, and it is now recognized as one instance of a general phenomenon.
In this work, we describe two variants of the Gauss Circle problem that exhibit similar
characteristics.




The first variant concerns sums of Fourier coefficients of $\GL(2)$ cusp forms.
These sums behave very similarly to the error term in the Gauss Circle problem.
Normalized correctly, it is conjectured that the two satisfy essentially the same
asymptotics.


We introduce new Dirichlet series with coefficients that are squares of partial sums of
Fourier coefficients of cusp forms.
We study the meromorphic properties of these Dirichlet series and use these series to give
new perspectives on the mean square of the size of sums of these Fourier coefficients.
These results are compatible with current conjectures.




The second variant concerns the number of lattice points of bounded size on one-sheeted
hyperboloids.
This problem is very similar to counting the number of lattice points
within a spheres of various dimensions, except with the additional constraint of lying on
a hyperboloid.
It turns out that this problem is equivalent to estimating sums of the shape
$\sum r_d(n^2 + h)$, where $r_d(m)$ is the number of representations of $m$ as a sum of
$d$ squares.
We prove improved smoothed and sharp estimates of the second moment of these sums,
yielding improved estimates of the number of lattice points.




In both variants, the problems are related to modular forms and, in particular, to shifted
convolution sums associated to these modular forms.
We introduce new techniques and new Dirichlet series which are quite general.
At the end of this work, we describe further extensions and questions for further
investigation.




% vim: tw=90
