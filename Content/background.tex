
In this chapter, we review and describe some topics that will be used heavily in later
chapters of this thesis.
This chapter should be used as a reference.
The topics covered are classical and well-understood, but included for the sake of
presenting a complete idea.



\section{Notation Reference}

\index{Landau!$\ll$}
\index{Landau!$\Omega$}
We use the basic notation of Landau, so that
\begin{equation}
  F(x) \ll G(x)
\end{equation}
means that there are constants $C$ and $X$ such that for all $x > X$, we have that
\begin{equation}
  \lvert F(x) \rvert \leq C G(x).
\end{equation}
We use $F(x) \ll G(x)$ and $F(x) = O(G(x))$ interchangeably.
We also use
\begin{equation}
  F(x) = o(G(x))
\end{equation}
to mean that for any $\epsilon > 0$, there exists an $X$ such that for all $x > X$, we
have that $F(x) \leq \epsilon G(x)$.
On the other hand,
\begin{equation}
  F(x) = \Omega(G(x))
\end{equation}
means that $F$ and $G$ do not satisfy $F(x) = o(G(x))$.
Stated differently, $F(x)$ is as least as large as $G(x)$ (up to a constant) infinitely
often.



We will use $r_d(n)$ to denote the number of representations of $n$ as a sum of $d$
squares, i.e.\
\index{r@$r_d(n)$}
\begin{equation}
  r_d(n) := \# \{ \bm x \in \mathbb{Z}^d : x_1^2 + \cdots + x_d^2 = n \}.
\end{equation}
We will denote partial sums of $r_d(n)$ by $S_d$, i.e.\
\index{S@$S_d(X)$}
\begin{equation}
  S_d(X) := \sum_{n \leq X} r_d(n).
\end{equation}


\index{B@$B_d(R)$}
We use $B(R)$ to denote the disk of radius $R$ centered at the origin.
It is traditional to talk about the \emph{Gauss Circle problem} instead of the \emph{Gauss
Disk problem}, and so we will frequently refer to $B(R)$ as a circle.
We use $B_d(R)$ to denote the dimension $d$ ball of radius $R$ centered at the origin, and
for similar conventional reasons we will frequently refer to $B_d(R)$ as a sphere.
Throughout this work, we will never distinguish between points on the surface of a sphere
and those points contained within the sphere, so we use these terms interchangeably.


With respect to modular forms, we typically use the conventions and notations
of~\cite{Bump98, Goldfeld2006automorphic}.


Typically, $f$ will be a weight $k$ holomorphic modular form on a congruence subgroup of
$\SL(2, \mathbb{Z})$.
The Fourier expansion of $f$ will be written as
\begin{equation}
  f(z) = \sum_{n \geq 1} a(n) e(nz),
\end{equation}
where $e(nz) = e^{2\pi i n z}$.\index{e@$e(nz)$}
Note that these coefficients are not normalized.
When it is necessary to use a second holomorphic modular form, we will denote it by $g$
and its Fourier expansion by
\begin{equation}
  g(z) = \sum_{n \geq 1} b(n) e(nz).
\end{equation}
Partial sums of these coefficients are denoted by
\index{S@$S_f(X)$}
\begin{equation}
  S_f(X) := \sum_{n \leq X} a(n), \qquad S_g(X) := \sum_{n \leq X} b(n).
\end{equation}


Note that we will use $k$ to denote a full integer or a half integer in reference to
``modular forms of weight $k$.''
This is a different convention than some authors, who use $k/2$ in discussion of
half-integral weight forms.


Given two modular forms $f$ and $g$ defined on a congruence subgroup $\Gamma \subseteq
\SL(2, \mathbb{Z})$, we will let $\langle f, g \rangle$ denote the Petersson inner
product
\index{Petersson inner product}
\begin{equation}
  \langle f, g \rangle := \iint_{\Gamma \backslash \mathcal{H}} f(z) \overline{g(z)}
  d\mu(z).
\end{equation}
In this expression, $\mathcal{H}$ denotes the complex upper half plane and $d\mu(z)$
denotes the Haar measure, which in this case is given by
\begin{equation}
  d\mu(z) = \frac{dx\, dy}{y^2}.
\end{equation}


We will use $\kron{a}{n}$ to denote the Jacobi symbol.
Note that this sometimes may look very similar to a fraction.
Generally, the Jacobi symbol involves only arithmetic data, while fractions will have
complex valued arguments.


We also roughly follow some conventions concerning variable names.


The primary indices of summation will be $m$ and $n$.
If there is a shifted summation, we will usually use $h$ to denote the shift.
We will sometimes use $\ell$ to denote a distinguished index of summation (or more
generally a distinguished variable in local discussion).
The one major exception to this convention is the index $j$, which we reserve for spectral
summations (and in particular the discrete spectrum) or as an index for residues and
residual terms appearing from spectral analysis.


Complex variables will be denoted by $z,s,w$, and $u$.
Almost always $z = x + iy$ will be in the upper half plane $\mathcal{H}$, and denotes the complex
variable of the underlying modular form.
The other variables, $s,w$, and $u$ denote generic complex variables and usually appear as
the variables of various Dirichlet series.
We often follow the odd but classical convention of mixing Roman and Greek characters, and
write $s = \sigma + it$.
Note that we sometimes use $t$ to denote a generic real variable in integrals.


\index{L@$L^{(q)}(s)$}
If $L(s)$ is an $L$-function, then $L^{(q)}(s)$ denotes that $L$ -function, but with the
Euler factors corresponding to primes $p$ dividing $q$ removed.


\index{a@$\mathfrak{a}$}
We use $\mathfrak{a}$ to denote a cusp of a modular curve.
In this thesis, this almost always means that $\mathfrak{a}$ is one of the three cusps of
$\Gamma_0(4) \backslash \mathcal{H}$.




\section{On Full and Half-Integral Weight Eisenstein Series}\label{sec:Eisen_summary}
\index{Eisenstein series!weight $k$}
\index{Eisenstein series!half-integral weight}
\index{E@$E_\mathfrak{a}^k(z,w)$| \\see {Eisenstein series!weight $k$} }


The primary goal of this section is to describe some characteristics of the weight $k$
Eisenstein series for $\Gamma_0(4)$, both for full-integral weight and half-integral
weight $k$.
These details are often considered standard exercises in the literature, and are usually
tedious to compute.
We will emphasize the important properties of the Eisenstein series necessary for the
argument in Chapter~\ref{c:hyperboloid} --- some with complete proofs, and some with
descriptions of the method of proof.


\index{Eisenstein series!weight $0$}
Selberg defined the weight $0$ classical real analytic Eisenstein series $E(z,w)$ on
$\SL(2, \mathbb{Z})$ by
\begin{equation}
  E(z,w) = \sum_{\gamma \in \Gamma_\infty \backslash \SL(2, \mathbb{Z})} \Im (\gamma z)^s.
\end{equation}
The weight $0$ Eisenstein series $E(z,w)$ is very classical and very well-understood, and
both~\cite[Chapter 3]{Goldfeld2006automorphic} and~\cite[Chapter 13]{Iwaniec97} provide an
excellent description of its properties.


Weight $k$ Eisenstein series are also very classical, but they appear less often and
with much less exposition in the literature.
Half-integral weight Eisenstein series occupy an even smaller role in the literature,
although in recent years the study of metaplectic forms and metaplectic Eisenstein series
has grown in popularity.


We will use $E_\mathfrak{a}^k(z,w)$ to denote the weight $k$ Eisenstein series associated
to the cusp $\mathfrak{a}$.
In this expression, $k$ can be either a half-integer or a full-integer.
In this thesis, we use the half-integral weight Eisenstein series on the congruence
subgroup $\Gamma_0(4)$.
The quotient $\Gamma_0(4) \backslash \mathcal{H}$ has three cusps, at $\infty, 0$, and
$\frac{1}{2}$.


We shall describe the three Eisenstein series $E^k_\infty(z,w), E^k_0(z,w)$, and
$E^k_{\frac{1}{2}}(z,w)$, both for full-integral weight and half-integral weight $k$.
These Eisenstein series are defined as
\begin{align}
  E_\infty^k(z,w) &= \sum_{\gamma \in \Gamma_\infty \backslash \Gamma_0(4)} \Im(\gamma
  z)^w J(\gamma, z)^{-2k}
  =
  \sum_{\substack{\pm (c,d) \in \mathbb{Z}^2 \\ (c,d) \neq (0,0) \\ \gcd(4c,d) = 1}}
  \frac{y^w \varepsilon_d^{2k} \kron{4c}{d}^{2k}}{\lvert 4cz + d \rvert^{2w-k}(4cz+d)^k}
  \\
  E_0^k(z,w) &= \Big( \frac{z}{\lvert z \rvert} \Big)^{-k} E_\infty^k(\tfrac{-1}{4z}, w)
  =
  \sum_{\substack{\pm(c,d) \in \mathbb{Z}^2 \\ (c,d) \neq (0,0) \\ \gcd(4c, d) = 1}}
  \frac{(y/4)^w \varepsilon_d^{2k} \kron{4c}{d}^{2k}}{\lvert -c + dz \rvert^{2w - k} (-c +
  dz)^k}
  \\
  E_{\frac{1}{2}}(z,w) &= \Big( \frac{2z+1}{\lvert 2z+1 \rvert} \Big)^{-k}
  E_\infty^k(\tfrac{z}{2z+1}, w) =
  \sum_{\substack{\pm(c,d) \in \mathbb{Z}^2 \\ (c,d) \neq (0,0) \\ \gcd(2c, d) = 1}}
  \frac{y^w \varepsilon_d^{2k} \kron{2c}{d}^{2k}}{\lvert 2cz+d \rvert^{2w -
  k}(2cz+d)^k}.
\end{align}
In these expressions, $\varepsilon_d$ denotes the sign of the Gauss sum associated to the
primitive real quadratic Dirichlet character $\chi_d$, and is given by
\index{epsilon@$\varepsilon_d$}
\begin{equation}
  \varepsilon_d = \begin{cases}
    1 & d \equiv 1 \bmod 4, \\
    i & d \equiv 3 \bmod 4.
  \end{cases}
\end{equation}
We use $J(\gamma, z)$ to denote the normalized half-integral weight cocycle
\index{J@$J(\gamma, z)$}
$J(\gamma, z) = j(\gamma, z)/\lvert j(\gamma, z) \rvert$,
where $j(\gamma, z)$ is the standard $\theta$ cocycle\index{j@$j(\gamma, z)$}
\begin{equation}
  j(\gamma, z) :=
  \theta(\gamma z)/\theta(z) = \varepsilon_d^{-1} \kron{c}{d} (cz + d)^{\frac{1}{2}}.
\end{equation}
Here, $\theta(z)$ denotes the Jacobi theta function,\index{theta@$\theta(z)$}
which is a modular form of weight $\frac{1}{2}$ on $\Gamma_0(4)$.
Thus this cocycle law holds for $\gamma = \big( \begin{smallmatrix} a&b\\c&d
\end{smallmatrix} \big) \in \Gamma_0(4)$.



Notice that when $k$ is a full integer, the weighting factor simplifies to
\begin{equation}
  \varepsilon_d^{2k} \kron{4c}{d}^{2k} = \kron{-1}{d}^k,
\end{equation}
and when $k$ is a half-integer, the weighting factor simplifies slightly to
\begin{equation}
  \varepsilon_d^{2k} \kron{4c}{d}^{2k} = \varepsilon_d^{2k} \kron{4c}{d},
\end{equation}
(where in both cases the $4$ is replaced by $2$ for $E_\frac{1}{2}^k$).
This is a restatement of the fact that when $k$ is a full integer, the transformation law
is full-integral with character $\kron{-1}{\cdot}^k$, while for half-integers the
transformation law is a power of a standard normalized theta cocycle.


When $k$ is a half-integer, these three Eisenstein series are essentially the same three
Eisenstein series that appear in~\cite{goldfeld1985eisenstein}, only with some notational
differences.
Firstly, in~\cite{goldfeld1985eisenstein}, half-integer weights are denoted by
$\frac{k}{2}$, while in this work we denote all weights (both integral and half-integral)
by $k$.
Secondly, we shift the spectral argument, replacing $w$ with $w - \frac{k}{2}$ as compared
to~\cite{goldfeld1985eisenstein}.
This has the effect of using the normalized cocycle $J(\gamma,z)$ instead of
$j(\gamma,z)$, and also normalizes the arguments of the $L$-functions that appear in the
coefficients of the Eisenstein series.


\index{rho@$\rho_\mathfrak{a}^k$}
Each Eisenstein series has a Fourier-Whittaker expansion of the form
\begin{equation}
  E_\mathfrak{a}^k(z,w) = y^w + \rho_\mathfrak{a}^k(0, w) y^{1-w} + \sum_{h \neq 0}
  \rho_\mathfrak{a}^k(h, w) W_{\frac{|h|}{h}\frac{k}{2}, w - \frac{1}{2}}(4\pi \lvert h
  \rvert y) e^{2\pi i h x}
\end{equation}
for some coefficients $\rho_\mathfrak{a}^k(h,w)$, and where $W_{\alpha, \nu}(y)$ is the
$\GL(2)$ Whittaker function.
This is given by~\cite[3.6.3]{GoldfeldHundleyI}
\index{Whittaker function}
\index{W@$W_{\alpha, \nu}(y)$ | see {Whittaker function } }
\begin{equation}
  W_{\alpha, \nu}(y) = \frac{y^{\nu + \frac{1}{2}}
  e^{-\frac{y}{2}}}{\Gamma(\nu - \alpha + \frac{1}{2})}
  \int_0^\infty e^{-yt} t^{\nu - \alpha - \frac{1}{2}}
  (1+t)^{\nu + \alpha - \frac{1}{2}} dt,
\end{equation}
valid for $\Re(\nu - \alpha) > \frac{-1}{2}$
and $\lvert \text{arg}(y) \rvert < \pi$.


When $\alpha = 0$, the Whittaker function is just the $K$-Bessel function
\index{K@$K$-Bessel function, $K_\nu(y)$}
\begin{equation}
  W_{0, \nu}(y) = \big( \frac{y}{\pi}\big)^{\frac{1}{2}} K_\nu\big( \frac{y}{2} \big),
  \quad
  K_\nu(y) = \frac{1}{2} \int_0^\infty e^{-\frac{1}{2} y(u + u^{-1})}
  u^\nu \frac{du}{u}.
\end{equation}
Thus in the weight $0$ case, the Whittaker functions simplify to $K$-Bessel functions (as
noted and employed in~\cite{Goldfeld2006automorphic} for instance).
This additional difficulty in weight $k$ is perhaps one reason why most expository
accounts stop at the weight $0$ case.


According to the general theory of Selberg (and described in~\cite[Theorem
13.2]{Iwaniec97}), the potential poles of $E_\mathfrak{a}^k(z,w)$ for $w > \frac{1}{2}$
can be recognized from the poles of the constant coefficient $\rho_{\mathfrak{a}}^k(0,w)$.




Although the individual expansions vary, they are usually of a very similar form.
For full integral weight, the coefficients have the shape
\index{rho@$\rho_\mathfrak{a}^k$!\\full integral weight}
\begin{align}
  \rho_\mathfrak{a}^k(0, w) &= (*) \frac{L^k_\mathfrak{a}(2w-1)}{L^k_\mathfrak{a}(2w)}
  \frac{\Gamma(2w-1)}{\Gamma(w + \frac{k}{2}) \Gamma(w - \frac{k}{2})} \\
  \rho_\mathfrak{a}^k(h, w) &= (*) \frac{\lvert h \rvert^{w-1}}{L^k_\mathfrak{a}(2w)}
  \frac{1}{\Gamma(w + \frac{\lvert h \rvert}{h} \frac{k}{2})} D_\mathfrak{a}^k(h,w)
\end{align}
where $(*)$ is a (possibly zero) constant times a collection of powers of $2$ and $\pi$,
$L^k_\mathfrak{a}(s)$ is a $\GL(1)$ $L$-function (maybe missing some Euler factors), and
$D_\mathfrak{a}^k$ is a finite Dirichlet sum.
For $\Re w > \frac{1}{2}$, the only potential pole of $E_\mathfrak{a}^k(z,w)$ is at $w = 1$.
This is shown in general in~\cite{Iwaniec97}, and the calculations are very similar to
those in~\cite[Chapter 3]{Goldfeld2006automorphic}.
For completeness and as a unifying reference, we recompute and state the exact
coefficients of $E_\infty^k$ in \S\ref{ssec:full_integral_eisenstein}.





For half-integral weight, the coefficients have similar shape,
\index{rho@$\rho_\mathfrak{a}^k$!\\half-integral weight}
\begin{equation}
  \begin{split}\label{eq:Eisenstein_halfweight_generalshape_coeffs}
    \rho_\mathfrak{a}^k(0, w) &= (*) \frac{L^k_\mathfrak{a}(4w-2)}{L^k_\mathfrak{a}(4w-1)}
    \frac{\Gamma(2w-1)}{\Gamma(w + \frac{k}{2}) \Gamma(w - \frac{k}{2})} \\
    \rho_\mathfrak{a}^k(h, w) &= (*) \lvert h \rvert^{w-1} \frac{1}{\Gamma(w +
    \frac{\lvert h \rvert}{h} \frac{k}{2})} D_\mathfrak{a}^k(h,w),
  \end{split}
\end{equation}
except that now $D_\mathfrak{a}^k(h,w)$ is a complete Dirichlet series formed from Gauss
sums, and which factors as
\begin{equation}\label{eq:Eisenstein_halfweight_Dirichletseries_factors}
  D_\mathfrak{a}^k(h,w) = \frac{L(2w - \frac{1}{2}, \chi_{k,h})}{\zeta(4w-1)}
  \widetilde{D}_\mathfrak{a}(h,w),
\end{equation}
where $\widetilde{D}_\mathfrak{a}(h,w)$ is a finite Dirichlet polynomial and $\chi_{k,h}$
\index{chi@$\chi_{k,h}$}
is the real quadratic character associated to the extension $\mathbb{Q}(\sqrt{\mu_k h})$,
and where $\mu_k = (-1)^{k - \frac{1}{2}}$.
In other words, these are almost the same as the full integral weight Eisenstein series,
except with an additional $L$-function in the numerator, and with slightly different
arguments of the involved $L$-functions.
For $\Re w > \frac{1}{2}$, the only potential pole of $E_\mathfrak{a}^k(z,w)$ is at $w
= \frac{3}{4}$.
This is all described in~\cite{goldfeld1985eisenstein}, including the factorization
in~\eqref{eq:Eisenstein_halfweight_Dirichletseries_factors}, which is given
in~\cite[Corollary 1.3]{goldfeld1985eisenstein}.
For completeness and as a unifying reference, we recompute and state the shape of the
coefficients of $E_\infty^k$ in \S\ref{ssec:half_integral_eisenstein}, including a proof
of the factorization~\eqref{eq:Eisenstein_halfweight_Dirichletseries_factors}.


The rest of this section consists of a complete description of these coefficients.
However, the general shapes of the coefficients are sufficient for the rest of this
thesis.





\subsection{Full Integral Weight}\label{ssec:full_integral_eisenstein}


\begin{claim}
  For $k \geq 1$ a full integer, the Fourier-Whittaker coefficients of $E_\infty^k(z,w)$
  are given by \begin{align}
    \rho_\infty^k(0,w) &= \begin{cases}
    0 & \text{if } k \; \text{odd}  \\
    \displaystyle y^{1-w} 2\pi 4^{1-3w} \frac{\zeta(2w-1)}{\zeta^{(2)}(2w)}
    \frac{\Gamma(2w-1)}{\Gamma(w + \frac{k}{2}) \Gamma(w - \frac{k}{2})}
    & \text{if } k \; \text{even}
    \end{cases}
    \\
    \rho_\infty^k(h,w) &= \frac{\pi^w  \lvert h \rvert^{w-1} e^{-\frac{i \pi k}{2}}}
    {L\big( 2w, \kron{-4}{\cdot}^k\big) \Gamma(w + \frac{\lvert h \rvert}{h}
    \frac{k}{2})} D_\infty^k(h,w)
  \end{align}
  where
  \begin{equation}
    D_\infty^k(h,w) = \sum_{c \mid h} \frac{c}{(4c)^{2w}}\Big( e^{\frac{\pi i h}{2c}} +
    (-1)^k e^{\frac{3\pi i h}{2c}}\Big)
\end{equation}
  is a finite Dirichlet sum.
\end{claim}

This is a classical computation.
For completeness and ease of reference, we go through this computation completely.
The coefficients for other full integral weight Eisenstein coefficients are very similar,
and are achieved through (essentially) the same work.

\begin{proof}
  In the expression for $E_\infty^k(z,w)$, when $(c,d) = (0,1)$, there is the single term $y^w$.
  We compute the rest of the $0$th Fourier-Whittaker coefficient through
  \begin{align}
    \rho_\infty^k(0, w) &= \int_0^1 E_\infty^k(z,w) - y^w dx = \sum_{\substack{c > 0, d
    \in \mathbb{Z} \\ \gcd(4c,d) = 1}} \int_0^1 \frac{y^s \kron{-4}{d}^k}{\lvert 4cz + d
    \rvert^{2w - k} (4cz + d)^k}.
  \end{align}
  We write $\kron{-4}{d}$ in place of $\kron{-1}{d}$ to facilitate our next step, and as a
  way of reinforcing that $\gcd(d,4) = 1$.  Multiplying through by $L\big(2w,
  \kron{-4}{\cdot}^k\big)$ and distributing allows us to remove the $\gcd(4c, d) = 1$
  condition from the sum, so that
  \begin{align}
    \rho_\infty^k(0,w) &= \frac{y^w}{L\big(2w, \kron{-4}{\cdot}^k\big)} \sum_{c > 0}
    \sum_d \kron{-4}{d}^k \int_0^1  \frac{1}{\lvert 4cz+d \rvert^{2w-k}(4cz + d)^k} dx \\
    &= \frac{y^w}{L\big(2w, \kron{-4}{\cdot}^k\big)} \sum_{c > 0} \frac{1}{(4c)^{2w}} \sum_d
    \kron{-4}{d}^k \int_{\frac{d}{4c}}^{1 + \frac{d}{4c}}  \frac{1}{\lvert z
    \rvert^{2w-k}z^k} dx.
  \end{align}
  We can write $d = d' + 4cq$ for each $d' \bmod{4c}$ and $q \in \mathbb{Z}$ uniquely, and
  executing the resulting $q$ sum tiles the integral to the whole real line, giving
  \begin{equation}
    \rho_\infty^k(0,w) = \frac{y^w}{L\big(2w, \kron{-4}{\cdot}^k\big)} \sum_{c > 0}
    \frac{1}{(4c)^{2w}} \sum_{d \bmod 4c} \kron{-4}{d}^k \int_{-\infty}^{\infty}
    \frac{1}{\lvert z \rvert^{2w-k}z^k} dx.
  \end{equation}
  Notice that the $d$ sum is $0$ if $k$ is odd, and is $2c$ (the number of odd integers up
  to $4c$) when $k$ is even.  Further, when $k$ is even the $L$-function in the
  denominator simplifies to a zeta function missing its $2$-factor.

  It remains necessary to compute the integral.
  By~\cite[Section 13.7]{Iwaniec97}, we have the classical integral transform
  \begin{equation}\label{eq:IntegralIdentity_0th}
    \int_{-\infty}^\infty \frac{1}{\lvert z \rvert^{2w - k} z^k} = y^{1-2w} \pi 4^{1-w}
    \frac{\Gamma(2w-1)}{\Gamma(w + \frac{k}{2}) \Gamma(w - \frac{k}{2})}.
  \end{equation}
  Therefore, we have for $k$ even that
  \begin{equation}
    \rho_\infty^k(0,w) = y^{1-w} 2\pi 4^{1-3w} \frac{\zeta(2w-1)}{\zeta^{(2)}(2w)}
    \frac{\Gamma(2w-1)}{\Gamma(w + \frac{k}{2}) \Gamma(w - \frac{k}{2})}
  \end{equation}

  For the $h$th Fourier coefficient, the method begins similarly.
  Following the same initial steps, we compute
  \begin{align}
    \rho_\infty^k(h,w) &= \int_0^1 E_\infty^k(z,w) e^{-2\pi i h x} dx \\
    &=
    \frac{y^w}{L(2w, \kron{-4}{\cdot}^k)} \sum_{c > 0} \frac{1}{(4c)^{2w}} \sum_d
    \kron{-4}{d}^k e^{2\pi i \frac{hd}{4c}} \int_{\frac{d}{4c}}^{1 + \frac{d}{4c}}
    \frac{e^{-2\pi i h x}}{\lvert z \rvert^{2w - k} z^k} dx.
  \end{align}
  Writing $d = d' + 4cq$ as above again tiles out the integral, so that
  \begin{equation}
    \rho_\infty^k(h,w) = \frac{y^w}{L(2w, \kron{-4}{\cdot}^k)} \sum_{c > 0}
    \frac{1}{(4c)^{2w}} \sum_{d\bmod 4c} \kron{-4}{d}^k e^{2\pi i \frac{hd}{4c}}
    \int_{-\infty}^{\infty} \frac{e^{-2\pi i h x}}{\lvert z \rvert^{2w - k} z^k} dx.
  \end{equation}

  For $h \neq 0$, the integral can be evaluated as in~\cite[\S13.7]{Iwaniec97} or
  using~\cite[3.385.9]{GradshteynRyzhik07} to be
  \begin{equation}\label{eq:IntegralIdentity_hth}
    \int_{-\infty}^{\infty} \frac{e^{-2\pi i h x}}{\lvert z \rvert^{2w - k} z^k} dx =
    \frac{y^{-w} i^{-k} \pi^w \lvert h \rvert^{w-1}}{\Gamma(w + \frac{\lvert h \rvert}{h}
    \frac{k}{2})} W_{\frac{\lvert h \rvert}{h}\frac{k}{2}, \frac{1}{2} - w}(4\pi \lvert h
  \rvert y).
  \end{equation}
  To understand the arithmetic part,
  \begin{equation}
    \sum_{c > 0} \frac{1}{(4c)^{2w}} \sum_{d \bmod 4c} \kron{-4}{d}^k e^{\pi i \frac{hd}{2c}},
  \end{equation}
  note that each $d$ can be written uniquely in the form $d = d' + 4q$ for $0 \leq d' < 4$
  and $0 \leq q < c$.
  Then the arithmetic part is written
  \begin{equation}
    \sum_{c > 0} \frac{1}{(4c)^{2w}} \sum_{d=0}^3 \kron{-4}{d}^k e^{\pi i
  \frac{hd}{2c}} \sum_{q \bmod c} e^{2\pi i \frac{hq}{c}}.  \end{equation}
  The final sum over $q$ is
  \begin{equation}
    \sum_{q \bmod c} e^{2\pi i \frac{hq}{c}} = \begin{cases}
      0 & \text{if } c \nmid h \\
      c & \text{if } c \mid h.
    \end{cases}
  \end{equation}
  Therefore the arithmetic part simplifies down to
  \begin{equation}
    \sum_{c \mid h} \frac{c}{(4c)^{2w}} \big( e^{\pi i \frac{h}{2c}} + (-1)^k
  e^{\frac{3\pi i h}{2c}} \big).  \end{equation}
  Simplification completes the proof.
\end{proof}

%%%%%%%%%%%%%%%%%%%%%%%%%%%%%%%%%%%%%%%%%%%%%%%%%%%%%%%%%%%%%%%%%%%%%%%%%%%%%%%
% Removed claims for the other two Eisenstein series. These are not necessary,
% and in fact the exact claims here haven't been fully verified. But complete
% coefficients are in Research Notebook I, pg 40-47.
%%%%%%%%%%%%%%%%%%%%%%%%%%%%%%%%%%%%%%%%%%%%%%%%%%%%%%%%%%%%%%%%%%%%%%%%%%%%%%%
%Very similar techniques yield the following two Fourier-Whittaker expressions.
%
%\begin{claim}
%  For $k$ a full integer, the Fourier-Whittaker coefficients of $E_0^k(z,w)$ are given by
%  \begin{align}
%    \rho_0^k(0,w) &= \frac{\pi}{4^w} \frac{L(2w-1, \kron{-4}{\cdot}^k)}{L(2w, \kron{-4}{\cdot}^k)} \frac{\Gamma(2w-1)}{\Gamma(w + \frac{k}{2}) \Gamma(w - \frac{k}{2})}
%    \\
%    \rho_0^k(h,w) &= \frac{\pi^w i^{-k} \lvert h \rvert^{w-1} W_{\frac{\lvert h \rvert}{h} \frac{k}{2}, \frac{1}{2} - w}(4\pi \lvert h \rvert y)}{4^w L(2w, \kron{-4}{\cdot}^k) \Gamma(w + \frac{\lvert h \rvert}{h} \frac{k}{2})} D_0^k(h,w)
%  \end{align}
%  where
%  \begin{equation}
%    D_0^k(h,w) = \sum_{d \mid h} \frac{\kron{-4}{d}^k}{d^{2w-1}}
%  \end{equation}
%  is a finite Dirichlet sum.
%\end{claim}
%
%\begin{claim}
%  For $k$ a full integer, the Fourier-Whittaker coefficients of $E_\frac{1}{2}^k(z,w)$ is given by
%  \begin{align}
%    \rho_\frac{1}{2}^k(0,w) &= \begin{cases}
%    0 & \text{if } k \; \text{odd}  \\
%    \pi 4^{1-2w} \frac{\zeta^{(2)}(2w-1)}{\zeta^{(2)}(2w)} \frac{\Gamma(2w-1)}{\Gamma(w + \frac{k}{2}) \Gamma(w - \frac{k}{2})} & \text{if } k \; \text{even}
%    \end{cases}
%    \\
%    \rho_\frac{1}{2}^k(h,w) &= \frac{2\pi^w i^{-k} \lvert h \rvert^{w-1}}{4^{w + 1} L(2w, \kron{-4}{\cdot}^k)} W_{\frac{\lvert h \rvert}{h} \frac{k}{2}, \frac{1}{2} - w} (4\pi \lvert h \rvert y) D_{\frac{1}{2}}^k(h,w)
%  \end{align}
%  where
%  \begin{equation}
%    D_{\frac{1}{2}}^k(h,w) = \sum_{\substack{c \mid h \\ \gcd(2,c) = 1}} \frac{1}{(c)^{2w}}\Big( e^{\frac{\pi i h}{c}} + (-1)^k e^{\frac{3\pi i h}{c}}\Big)
%  \end{equation}
%  is a finite Dirichlet sum.
%\end{claim}


\subsection{Half Integral Weight}\label{ssec:half_integral_eisenstein}

The pattern here is almost the exact same as with full-integral weight.
We will prove the expansion for $E_\infty^k(z,w)$ completely.
The primary difference is that there is now a Dirichlet series of Gauss sums, which can be
factored as a ratio of a Dirichlet $L$-function and a zeta function, up to a short
correction factor.

\begin{claim}
  For $k \geq \frac{1}{2}$ a half integer, the Fourier-Whittaker coefficients of $E_\infty^k(z,w)$ is given by
  \begin{align}
    \rho_\infty^k(0,w) &= \bigg( \frac{1+i^{2k}}{2} \bigg) \frac{\pi 4^{1-w}}{2^{4w - 1} - 1}
    \frac{\zeta(4w - 2)}{\zeta(4w - 1)}
    \frac{\Gamma(2w - 1)}{\Gamma(w + \frac{k}{2}) \Gamma(w - \frac{k}{2})}
    \\
    \rho_\infty^k(h,w) &= \frac{e^{\frac{-i\pi k}{2}} \pi^w \lvert h \rvert^{w-1}}
    {\Gamma(w + \frac{\lvert h \rvert}{h} \frac{k}{2})}
    \sum_{c > 0} \frac{g_h(4c)}{(4c^{2w})},
  \end{align}
  where\index{g@$g_h(c)$}
  \begin{equation}
    g_h(c) = \sum_{d \bmod c} \varepsilon_d^{2k} \kron{c}{d} e \big( \tfrac{hd}{c} \big)
  \end{equation}
  is a Gauss sum.
  The Dirichlet series associated to these Gauss sums can be written
  as\index{D@$\widetilde{D}_\infty^k(h,w)$}
  \begin{equation}
    \sum_{c \geq 1} \frac{g_h(4c)}{(4c)^{2w}}
    =
    \frac{L^{(2)} (2w - \frac{1}{2}, \chi_{k,h})}{\zeta^{(2h)}(4w-1)}
    \widetilde{D}_\infty^k(h,w),
  \end{equation}
  as proved in Proposition~\ref{prop:back:half_integral_L}.
\end{claim}

\begin{proof}
  As in the full-integral weight case, we compute the constant Fourier coefficient through
  \begin{equation}
    \rho_\infty^k(0,w) = \int_0^1 E_\infty^k(z,w) - y^w dx =
    \sum_{\substack{c > 0, d \in \mathbb{Z} \\ \gcd(4c, d) = 1}}
    \int_0^1 \frac{y^w \varepsilon_d^{2k} \kron{4c}{d}}
    {\lvert 4cz + d \rvert^{2w - k} (4cz + d)^k}.
  \end{equation}
  The only difference in comparison to the full-integral weight case is that the numerator
  has $\varepsilon_d^{2k} \kron{4c}{d}$.
  Notice that the character enforces $\gcd(4c, d) = 1$, so that this condition can be
  dropped from the summation indices.

  Factoring $(4c)^{-2w}$, writing $d = d' + 4cq$ for each $d' \bmod 4c$ and a unique
  $q \in \mathbb{Z}$, performing the change of variables $x \mapsto x - \frac{d}{4c}$, and
  tiling the integral functions exactly as in the full-integral proof, and leads to
  \begin{equation}
    y^w \sum_{c > 0} \frac{1}{(4c)^{2w}} \sum_{d \bmod 4c} \varepsilon_d^{2k} \kron{4c}{d}
    \int_{-\infty}^\infty \frac{1}{\lvert z \rvert^{2w-k} z^k} dx.
  \end{equation}
  Using the evaluation of this integral from~\eqref{eq:IntegralIdentity_0th}, we see that
  the $0$th coefficient can be written as
  \begin{equation}
    y^{1-w} \sum_{c > 0} \frac{1}{(4c)^{2w}} \sum_{d \bmod 4c} \varepsilon_d^{2k}
    \kron{4c}{d} \frac{\pi 4^{1-w} \Gamma(2s - 1)}{\Gamma(w + \frac{k}{2})
    \Gamma(w - \frac{k}{k})}.
  \end{equation}

  It is now necessary to understand the arithmetic part, given by
  \begin{equation}\label{eq:back:E_halfk_0_I}
    \sum_{c > 0} \frac{1}{(4c)^{2w}} \sum_{d \bmod 4c} \varepsilon_d^{2k} \kron{4c}{d}.
  \end{equation}
  Noting that we can rewrite $\varepsilon_d$ as
  \begin{equation}
    \varepsilon_d = \frac{1}{2}\big( \chi_{-1}^2(d) + \chi_{-1}(d) \big) +
    \frac{i}{2}\big( \chi_{-1}^2(d) - \chi_{-1}(d) \big),
  \end{equation}
  where $\chi_{-1}(d) = \kron{-1}{d}$ is the Legendre character associated to when $-1$ is
  a square, we see that the $d$ summation is trivial unless the character $\kron{4c}{d}$
  is trivial.
  (This follows from classical observation that summing a non-trivial character over a
  whole group gives zero).
  Thus the sum is nontrivial if and only if $c$ is a square, and when $c$ is a square we
  have that
  \begin{equation}
    \sum_{d \bmod 4c^2} \varepsilon_d^{2k}\kron{4c^2}{d}
    = \sum_{\substack{d \bmod 4c^2 \\ \gcd(d,4c) = 1}} \varepsilon_d^{2k}
    = \frac{1 + i^{2k}}{2} \varphi(4c^2).
  \end{equation}
  Therefore~\eqref{eq:back:E_halfk_0_I} can be written as
  \begin{equation}
     \frac{1+i^{2k}}{2} \sum_{c > 0} \frac{1}{(4c^2)^{2w}} \varphi(4c^2)
     = \frac{1+i^{2k}}{2} \frac{1}{2^{4w - 1} - 1} \frac{\zeta(4w - 2)}{\zeta(4w - 1)}.
  \end{equation}
  The last equality follows from comparing Euler products, correcting the missing
  $2$-factor, and simplifying.
  Combining the arithmetic part with the analytic part gives the $\rho_\infty^k(0,w)$.


  Computing the $h$th Fourier coefficient is very similar.
  After following the same initial steps to tile out the integral, we get the equality
  \begin{equation}
    \int_0^1 E_\infty^k(z,w) e^{-2\pi i h x} dx = \sum_{c > 0} \frac{1}{(4c)^{2w}}
    \sum_{d \bmod 4c} \varepsilon_d^{2k} \kron{4c}{d} e\big(\tfrac{hd}{4c}\big)
    \int_{-\infty}^\infty \frac{y^w e^{-2\pi i h x}}{\lvert z \rvert^{2w-k} z^k} dx.
  \end{equation}
  This integral was evaluated in~\eqref{eq:IntegralIdentity_hth}, so we see that this
  becomes
  \begin{equation}
     \frac{e^{-i\pi k/2} \pi^w \lvert h \rvert^{w-1}}
    {\Gamma(w + \frac{\lvert h \rvert}{h}\frac{k}{2})}
    W_{\frac{\lvert h \rvert}{h}\frac{k}{2}, \frac{1}{2} - w} (4 \pi \lvert h \rvert y)
    \sum_{c > 0} \frac{1}{(4c)^{2w}}
    \sum_{d \bmod 4c} \varepsilon_d^{2k} \kron{4c}{d} e\big(\tfrac{hd}{4c}\big).
  \end{equation}
  The arithmetic part is now a Dirichlet series of Gauss sums, and is not finite.
  Simplifiying completes the computation.
\end{proof}





\subsection*{On the $L$-series associated to half-integral weight Eisenstein series
coefficients}


We now consider the Dirichlet series
\begin{equation}
  \sum_{c \geq 1} \frac{g_h(4c)}{(4c)^{2w}}
\end{equation}
appearing in the $h$th Fourier coefficients of $E_\infty^k(z,w)$ when $k$ is a half
integer.
The goal of this section is to show the following proposition.


\begin{proposition}\label{prop:back:half_integral_L}
  \begin{equation}
    \sum_{c \geq 1} \frac{g_h(4c)}{(4c)^{2w}}
    =
    \frac{L^{(2)} (2w - \frac{1}{2}, \chi_{k,h})}{\zeta^{(2h)}(4w-1)}
    \widetilde{D}_\infty^k(h,w),
  \end{equation}
  where $\widetilde{D}_\infty^k(h,w)$ is a finite Dirichlet polynomial.
\end{proposition}


We prove this proposition through a series of lemmata.\footnote{Yes, that is a real word.
And it's a fantastic word.}
First, we first consider just an individual Gauss sum
$g_h(4c) = \sum_{d \bmod 4c} \varepsilon_d^{2k} e(\frac{hd}{4c}) \kron{4c}{d}$.
It is necessary to break each Gauss sum into two pieces.

\begin{lemma}\label{lem:back:gauss_two_pieces}
  \begin{equation}
    g_h(4c) = \sum_{d_2\bmod 2^\alpha} \varepsilon_{d_2c'}^{2k} (-1)^{\frac{c'-1}{2}
    \frac{d_2c'-1}{2}}\varepsilon_{c'}^{-1} \kron{2^\alpha}{d_2}
    \varepsilon_{c'}\sum_{d_1\bmod c'} \kron{d_1}{c'}  e\big( \tfrac{h d_1}{c'} \big).
  \end{equation}
\end{lemma}


\begin{proof}

Write $4c = 2^\alpha c'$ where $\gcd(c', 2) = 1$.
Note that we necessarily have $\alpha \geq 2$.
For any $d \bmod 4c$, we write $d = d_1 2^\alpha + d_2c'$ with $d_1$ varying mod $c'$ and
$d_2$ varying mod $2^\alpha$.
Using this, we can write
\begin{align}
    \sum_{d \bmod 4c} \varepsilon_d^{2k} e\big(\tfrac{hd}{4c}\big) \kron{4c}{d}
    &=
    \sum_{(d_1\bmod c')} \sum_{(d_2\bmod 2^\alpha)} \varepsilon_{d_2c'}^{2k}
    e\big( \tfrac{h (d_1 2^\alpha + d_2 c')}{2^\alpha c'} \big)
    \kron{2^\alpha c'}{d_1 2^\alpha + d_2c'}
    \\
    &=
    \sum_{d_2 \bmod 2^\alpha} \varepsilon_{d_2c'}^{2k}
    e\big(\tfrac{h d_2}{2^\alpha}\big)
    \kron{2^\alpha}{d_2c'} \sum_{d_1 \bmod c'}
  \kron{c'}{d_1 2^\alpha+ d_2c'} e\big( \tfrac{h d_1}{c'} \big).
\end{align}

Note that $d_2$ is necessarily odd.
By quadratic reciprocity, we can rewrite the inner sum as
\begin{align}
    \sum_{d_1 \bmod c'} \kron{c'}{d_1 2^\alpha+ d_2c'} e\big( \tfrac{h d_1}{c'} \big)
    &=
    \sum_{d_1 \bmod c'} \kron{d_1 2^\alpha + d_2 c'}{c'}
    (-1)^{\frac{c'-1}{2} \frac{d_2c'-1}{2}} e \big( \tfrac{h d_1}{c'} \big)
    \\
    &=
    \sum_{d_1 \bmod c'} \kron{d_1 2^\alpha}{c'} (-1)^{\frac{c'-1}{2}
    \frac{d_2c'-1}{2}} e\big( \tfrac{h d_1}{c'}\big).
\end{align}
Inserting this back into $g_h(4c)$ gives
\begin{align}
  g_h(4c)
  &=
  \sum_{d_2\bmod 2^\alpha} \varepsilon_{d_2c'}^{2k} (-1)^{\frac{c'-1}{2}
  \frac{d_2c'-1}{2}} e\big( \tfrac{h d_2}{2^\alpha} \big) \kron{2^\alpha}{d_2}
  \kron{2^\alpha}{c'}\sum_{d_1\bmod c'} \kron{2^\alpha}{c'}\kron{d_1}{c'}
  e\big( \tfrac{h d_1}{c'} \big) \\
  &=
  \sum_{d_2\bmod 2^\alpha} \varepsilon_{d_2c'}^{2k} (-1)^{\frac{c'-1}{2}
  \frac{d_2c'-1}{2}} e\big( \tfrac{h d_2}{2^\alpha} \big)
  \varepsilon_{c'}^{-1} \kron{2^\alpha}{d_2}
  \varepsilon_{c'}\sum_{d_1\bmod c'} \kron{d_1}{c'}  e\big( \tfrac{h d_1}{c'} \big).
\end{align}
Note that we have introduced $\varepsilon_c \varepsilon_c^{-1}$.
This ends the proof of the lemma.
\end{proof}


The $d_1$ sum is now completely decoupled from the $d_2$ sum, and they can be understood
separately.
We first understand the $d_1$ summation.


\begin{lemma}\label{lem:back:Hh_multiplicative}
  For odd $c$, define
  \begin{equation}
    H_h(c) := \varepsilon_c \sum_{d \bmod c} \kron{d}{c} e \big( \tfrac{h d}{c} \big),
  \end{equation}
  as occurs in the $d_1$ summation in $g_h(4c)$.
  Then $H_h(c)$ is multiplicative.
\end{lemma}


\begin{proof}


Let $n_1$ and $n_2$ be two odd relative prime integers.
By the Chinese Remainder Theorem, and congruence class $d \bmod n_1 n_2$ can be uniquely
written as $d = b_2 n_1 + b_1 n_2$ where $b_2$ is defined modulo $n_2$ and $b_1$ is
defined modulo $n_1$.
Then
\begin{align}
  H_h(n_1 n_2)
  &=
  \varepsilon_{n_1 n_2} \sum_{d \bmod n_1n_2} \kron{d}{n_1n_2}
  e\big(\tfrac{d h}{n_1 n_2} \big)
  \\
  &= \varepsilon_{n_1 n_2} \sum_{b_1 \bmod n_1} \sum_{b_2 \bmod n_2}
  \kron{b_2 n_1 +b_1 n_2}{n_1 n_2}  e \big( \tfrac{h b_2 n_1 +b_1 n_2}{n_1 n_2} \big)
  \\
  &= \varepsilon_{n_1 n_2}\kron{n_2}{n_1} \kron{n_1}{n_2}
  \sum_{b_1 \bmod n_1} \kron{b_1}{n_1} e\big( \tfrac{h b_1}{n_1} \big)
  \sum_{b_2 \bmod n_2} \kron{b_2}{n_2} e\big( \tfrac{h b_2}{n_2} \big)
  \\
  &= \varepsilon_{n_1 n_2} \varepsilon_{n_1}^{-1} \varepsilon_{n_2}^{-1}
  \kron{n_2}{n_1} \kron{n_1}{n_2} H_h(n_1) H_{h}(n_2).
\end{align}
Casework and quadratic reciprocity shows that
$\varepsilon_{n_1n_2} \varepsilon_{n_1^{-1}} \varepsilon_{n_2}^{-1}
\kron{n_2}{n_1}\kron{n_1}{n_2}= 1$,
so that $H_h(n_1 n_2) = H_h(n_1) H_h(n_2)$.
%
\end{proof}


In order to understand a Dirichlet series formed from $H_h(c')$, it will be sufficient to
understand $H_h(p^k)$ for odd primes $p$.
When $p \nmid h$, these are particularly easy to understand.


\begin{lemma}\label{lem:back:Hh_goodp_eval}
  Suppose $p$ is an odd prime and $p \nmid h$.
  Then
  \begin{equation}
    H_h(p^k) = \begin{cases}
      \kron{-h}{p} \sqrt p & k = 1 \\
      0 & k \geq 2.
    \end{cases}
  \end{equation}
\end{lemma}


\begin{proof}
  When $k = 1$, we have
  \begin{equation}
    H_h(p) = \varepsilon_p \sum_{d \bmod p} \kron{d}{p} e \big( \tfrac{dh}{p} \big) =
    \varepsilon_p^2 \kron{h}{p} \sqrt p = \kron{-h}{p} \sqrt p,
  \end{equation}
  as $H_h(p)$ is very nearly a standard Gauss Sum, as considered in the beginning
  of~\cite{Davenport1980}.

  For $k \geq 2$, there are two cases.
  If $k$ is even, then the sum is exactly a sum over the primitive $p^k$-roots of unity,
  and therefore is zero.
  If $k$ is odd, then writing $\eta = e^{2\pi i / p^k}$ as a primitive $p^k$-root of
  unity, we have
  \begin{align}
    H_h(p^k) &= \sum_{d\bmod p^k} \eta^d \kron d {p^k} = \sum_{d\bmod p^k} \eta^d \kron{d}{p} =
    \sum_{\substack{b \bmod p \\ c \bmod p^{k-1}}} \eta^{b+pc} \kron b p \\
    &= \sum_{b\bmod p} \eta^b
    \kron b p \sum_{c\bmod p^{k-1}} \eta^{pc} = 0,
  \end{align}
  as the inner sum is a sum over the primitive $p^{k-1}$ roots of unity.
\end{proof}


The case is substantially more complicated for primes $p$ dividing $h$.
For this thesis, we do not need to calculate these explicitly.
(Although we could, using these techniques).
It is sufficient to know that only finitely many contribute.


\begin{lemma}\label{lem:back:Hh_badp_eval}
  Suppose $p$ is an odd prime and $p \mid h$.
  Further, suppose $p^\ell \mid h$ but $p^{\ell + 1} \nmid h$.
  Then for $k \geq \ell + 2$, we have $H_h(p^k) = 0$.
\end{lemma}


\begin{proof}

  This is substantially similar to the case when $p \nmid h$, and the same proof carries
  over.
  When $k$ is even, $H_h(p^k)$ is a sum over primitive $p^{k-\ell}$ roots of unity.
  When $k$ is odd, $H_h(p^k)$ has an inner exponential sum over the $p^{k - \ell - 1}$
  roots of unity.
%
\end{proof}


\begin{remark}
  Although we do not compute it here, it is possible to compute the exact contribution
  from each factor $p$ dividing $h$.
  One complete reference can be found at the author's website~\cite{mixedmathGauss}.
  Some subcases are included in~\cite{goldfeld1985eisenstein}.
\end{remark}


We now seek to understand the $d_2$ summation in Lemma~\ref{lem:back:gauss_two_pieces}.


\begin{lemma}\label{lem:back:d2_sum_simplifies}
  We have
  \begin{equation}\label{eq:back:d2_sum_I}
    \varepsilon^{2k}_{d_2 c'} (-1)^{\frac{c'-1}{2} \frac{d_2 c' - 1}{2}}
    \varepsilon_{c'}^{-1}
    =
    \varepsilon_{d_2}^{2k} \kron{(-1)^{k + \frac{1}{2}}}{c'}.
  \end{equation}
  In particular,
  \begin{equation}
    \sum_{d_2 \bmod 2^\alpha}
    \varepsilon^{2k}_{d_2 c'} (-1)^{\frac{c'-1}{2} \frac{d_2 c' - 1}{2}}
    e\big( \tfrac{h d_2}{2^\alpha} \big)
    \varepsilon_{c'}^{-1} \kron{2^\alpha}{d_2}
    =
    \sum_{d_2 \bmod 2^\alpha}
    \varepsilon_{d_2}^{2k} \kron{(-1)^{k + \frac{1}{2}}}{c'} \kron{2^\alpha}{d_2}
    e\big( \tfrac{h d_2}{2^\alpha} \big).
  \end{equation}
\end{lemma}

\begin{proof}

  First note that as $\varepsilon_d^4 = 1$, we can reduce the analysis into two cases: when
  $2k \equiv 1 \bmod 4$ and when $2k \equiv 3 \bmod 4$.
  After this reduction, the equality in~\eqref{eq:back:d2_sum_I} is quickly verified by
  considering the possible values of $d_2$ and $c'$ modulo $4$ (recalling that both $d_2$
  and $c'$ are odd).
  The rest of the lemma follows immediately.
%
\end{proof}


In this simplified $d_2$ summation, there appears $\varepsilon_{d_2}^{2k}$, which acts a bit
like a character in $d_2$ modulo $4$, and $\kron{2^\alpha}{d_2}$, which acts a bit like a
character modulo $8$, and an unrestrained exponential.
Therefore we should expect that when $2^\alpha > 8$, or rather when $\alpha \geq 4$, then
the entire $d_2$ sum vanishes unless $h$ is highly divisible by $2$.


\begin{lemma}\label{lem:back:d2_sum_is_finite}
  Suppose $2^\ell \mid h$ and $2^{\ell + 1} \nmid h$.
  If $\alpha \geq \ell + 4$, then the $d_2$ summation vanishes, i.e.\
  \begin{equation}
    \sum_{d_2 \bmod 2^\alpha}
    \varepsilon_{d_2}^{2k} \kron{2^\alpha}{d_2} e\big( \tfrac{h d_2}{2^\alpha} \big) = 0.
  \end{equation}
\end{lemma}


\begin{proof}
  Without loss of generality, choose least non-negative representations for each $d_2 \bmod
  2^\alpha$.
  Write $d_2 = 8d' + d''$ where $0 \leq d' < 2^{\alpha - 3}$ and $0 \leq d'' < 8$.
  Then $\varepsilon_{8d' + d''} = \varepsilon_{d''}$ and
  $\kron{2^\alpha}{8d' + d''} = \kron{2^\alpha}{d''}$, so the $d'$ summation can be
  considered separately.
  This summation is
  \begin{equation}
    \sum_{d' = 0}^{2^{\alpha-3} - 1} e \bigg( \frac{h d'}{2^{\alpha - 3}} \bigg),
  \end{equation}
  which is $0$ unless $2^{\alpha - 3} \mid h$.
%
\end{proof}


As $c = 2^\alpha c'$, we see that the $d_2$ summation constrains the contribution from the
$2$-factor of $c$.
We are now finally ready to prove Proposition~\ref{prop:back:half_integral_L}.


\begin{proof}[Proof of Prop~\ref{prop:back:half_integral_L}]

  We try to understand
  \begin{align}
    &\sum_{c > 0} \frac{1}{(4c)^{2w}} \sum_{d \bmod 4c} \varepsilon_d^{2k}
    e \big( \tfrac{hd}{4c} \big) \kron{4c}{d}
    \\
    &= \sum_{\alpha \geq 2} \sum_{\substack{c \geq 1 \\ \gcd(2,c) = 1}}
    \frac{1}{(2^\alpha c)^{2w}} \kron{(-1)^{k + \frac{1}{2}}}{c} H_h(c)
    \sum_{d_2 \bmod 2^\alpha} \varepsilon_{d_2}^{2k}
    e \big( \tfrac{d_2 h}{2^\alpha} \big) \kron{2^\alpha}{d_2}
    \\
    &= \Bigg( \sum_{\alpha \geq 2} \sum_{d_2 \bmod 2^\alpha}
      \frac{1}{2^{2\alpha w}} \varepsilon_{d_2}^{2k} e \big( \tfrac{d_2 h}{2^\alpha} \big)
      \kron{2^\alpha}{d_2}
    \Bigg)
    \Bigg( \sum_{\substack{c \geq 1 \\ \gcd(c,2) = 1}}
      \frac{1}{c^{2w}}
      \Big(\frac{(-1)^{k + \frac{1}{2}}}{c}\Big) H_h(c)
    \Bigg). \label{eq:back:Lprop_proof_I}
  \end{align}
  We have used Lemma~\ref{lem:back:gauss_two_pieces} and
  Lemma~\ref{lem:back:d2_sum_simplifies} to split and simplify this expression.
  Let $\chi_k(c) := \kron{(-1)^{k + \frac{1}{2}}}{c}$ for ease of notation.
  As the summands over $c$ are multiplicative (by Lemma~\ref{lem:back:Hh_multiplicative}),
  the $c$ sum can be written (for $\Re w \gg 1$) as
  \begin{equation}
    \sum_{\substack{c \geq 1 \\ \gcd(c,2) = 1}}
    \frac{1}{c^{2w}} \chi_k(c) H_h(c)
    =
    \prod_{\substack{p \\ p \neq 2}} \Big( 1 + \frac{\chi_k(p) H_h(p)}{p^{2w}} +
    \frac{\chi_k(p^2) H_h(p^2)}{p^{4w}} + \cdots  \Big).
  \end{equation}


  For primes not dividing $h$, this expression simplifies significantly as $H_h(p^k) = 0$
  for $k \geq 2$ (as shown in Lemma~\ref{lem:back:Hh_goodp_eval}).
  The product over these primes is then
  \begin{equation}
    \prod_{\substack{p \\ p \neq 2 \\ p \nmid h}} \Big( 1 + \frac{\big(\frac{h (-1)^{k -
    \frac{1}{2}}}{p}\big)} {p^{2w - \frac{1}{2}}} \Big).
  \end{equation}
  On primes avoiding $h$ and $2$, it's quickly checked that this perfectly matches the
  Euler product for $L(2w-\frac{1}{2}, \chi_{k,h}) \zeta(4w-1)^{-1}$, where
  $\chi_{k,h}(\cdot) = \big( \frac{h (-1)^{k - \frac{1}{2}}}{\cdot} \big)$.
  Therefore we can write the product over those primes avoiding $h$ and $2$ as
  \begin{equation}
    \prod_{\substack{p \\ p \neq 2 \\ p \nmid h}} \Big( 1 + \frac{\big(\frac{h (-1)^{k -
    \frac{1}{2}}}{p}\big)} {p^{2w - \frac{1}{2}}} \Big)
    =
    \frac{L^{(2)} (2w - \frac{1}{2}, \chi_{k,h})}{\zeta^{(2h)}(4w-1)}.
  \end{equation}
  (Note that we are using the convention that $L^{(Q)}(s)$ denotes an $L$-function $L(s)$,
  but with the Euler factors for primes $p$ dividing $Q$ removed).


  We now consider the primes $p$ which do divide $h$ in the Euler product
  in~\eqref{eq:back:Lprop_proof_I}.
  By Lemma~\ref{lem:back:Hh_badp_eval}, we see that $H_h(p^k) = 0$ when $p^{k-1} \nmid h$.
  Therefore the product over primes dividing $h$ is a product of finitely many terms of
  finite length, and is thus just a Dirichlet polynomial.


  Similarly, by Lemma~\ref{lem:back:d2_sum_is_finite}, the sum over $d_2$ and $\alpha$
  in~\eqref{eq:back:Lprop_proof_I} is a finite sum whose length depends on the $2$-factor
  of $h$, and is also a Dirichlet polynomial.


  We group the product over primes dividing $h$ with the $d_2$ and $\alpha$ summation
  in~\eqref{eq:back:Lprop_proof_I}, which are both finite Dirichlet polynomials, into a
  single Dirichlet polynomial
  \begin{equation}
    \begin{split}
      \widetilde{D}_\infty^k (h,w) :=
      \sum_{\alpha \geq 2} \sum_{d_2 \bmod 2^\alpha}
        \frac{1}{2^{2\alpha w}} \varepsilon_{d_2}^{2k} e \big( \tfrac{d_2 h}{2^\alpha} \big)
        \kron{2^\alpha}{d_2}
      \times
      \prod_{\substack{p \\ p \mid h \\ p \neq 2}}
        \sum_{j \geq 0} \frac{\chi_k(p^j) H_h(p^j)}{p^{2jw}}.
    \end{split}
  \end{equation}
  Note carefully that although this is written as an infinite polynomial, it is a finite
  Dirichlet polynomial.


  Collecting these pieces together we have now shown that
  \begin{equation}
    \sum_{c > 0} \frac{1}{(4c)^{2w}} \sum_{d \bmod 4c} \varepsilon_d^{2k} e \big( \tfrac{hd}{4c} \big) \kron{4c}{d}
    =
    \frac{L^{(2)} (2w - \frac{1}{2}, \chi_{k,h})}{\zeta^{(2h)}(4w-1)}
    \widetilde{D}_\infty^k(h,w),
  \end{equation}
  as we set out to show.
%
\end{proof}


\begin{remark}
  If $h$ is squarefree, then it is possible to show that $\widetilde{D}_\infty^k(h,w)$ has
  the necessary Euler factors to ``fill in'' the $h$ factors of $\zeta^{(2h)}(4w - q)$
  (although not the $2$ factor), and the expression for $\widetilde{D}_\infty^k(h,w)$
  simplifies significantly.
  Thus the case when $h$ is squarefree is significantly simpler.
\end{remark}



\section{Cutoff Integrals and Their Properties}\label{sec:cutoff_integrals}


We recall the Mellin transform and inverse Mellin transform, and use these to construct
appropriate integral transforms to analyze properties of the coefficients of a Dirichlet
series.
In general, if
\index{Mellin transform}
\begin{equation}
  F(s) = \int_0^\infty f(x) x^s \frac{dx}{x},
\end{equation}
then $F(s)$ is the \emph{Mellin transorm} of $f(x)$.
Mellin transforms are deeply related to Laplace transforms and Fourier transforms, and
when $f$ and $F$ are sufficiently nice, there is an analogous inversion theorem giving
that
\begin{equation}
  f(x) = \frac{1}{2\pi i} \int_{(\sigma)} F(s) x^{-s} ds.
\end{equation}



\subsubsection*{Ces\`{a}ro Cutoff Transform}
\index{Integral transform!Ces\'{a}ro Weights}
\index{Integral transform!Riesz Means}


In this thesis, we reintroduce and use Ces\`{a}ro weights.
Note that these are sometimes referred to as ``Riesz Means.''
Given a positive integer $k$ and a Dirichlet series
\begin{equation}
  D(s) = \sum_{n \geq 1} \frac{a(n)}{n^s},
\end{equation}
we have the fundamental relationship
\begin{equation}\label{eq:cesaro_transform}
  \begin{split}
    \frac{1}{k!} \sum_{n \leq X} a(n) \Big(1 - \frac{n}{X}\Big)^k
    &= \frac{1}{2\pi i} \int_{(\sigma)} D(s) \frac{X^s}{s(s+1)\cdots(s+k)} ds
    \\
    &= \frac{1}{2\pi i} \int_{(\sigma)} D(s) \frac{X^s\Gamma(s)}{\Gamma(s+k+1)} ds,
  \end{split}
\end{equation}
where $\sigma$ is large enough that $D(s)$ and the integral absolutely converge.
The individual weights $(1 - \frac{n}{X})^k$ on each $a(n)$ are the $k$-Ces\`{a}ro
weights, and give access to smoothed asymptotics.

The relationship~\eqref{eq:cesaro_transform} is follows from the following integral
equality.
\begin{lemma}
  For $\sigma > 0$, we have
  \begin{equation}
    \frac{1}{2\pi i} \int_{(\sigma)} \frac{Y^s}{s(s+1)\cdots(s+k)} ds = \begin{cases}
      \frac{1}{k!} \big(1 - \frac{1}{Y}\big)^k & \text{if } Y \geq 1, \\
      0 & \text{if }Y < 1.
    \end{cases}
  \end{equation}
\end{lemma}
\begin{proof}
  In the case when $Y \geq 1$, shifting the contour infinitely far to the left shows that
  the integral can be evaluated as
  \begin{equation}
    \sum_{j = 0}^k \Res_{s = -j} \bigg( \frac{Y^s}{s(s+1)\cdots(s+k)} \bigg) = \sum_{j =
    0}^k \frac{(-1)^j Y^{-j}}{j! (k-j)!} = \frac{1}{k!} \Big(1 - \frac{1}{Y}\Big)^k.
  \end{equation}
  Note that the last equality is an application of the binomial theorem.

  In the case when $Y < 1$, shifting the contour infinitely far to the right shows that
  the integral is $0$.
\end{proof}

To recover~\eqref{eq:cesaro_transform}, one expands $D(s)$ within the integral and applies
the lemma to each individual term with $Y = (X/n)$.



\subsubsection*{Exponentially Smoothed Integral}
\index{Integral transform!exponential smoothing}

The integral definition of the Gamma function
\begin{equation}
  \Gamma(s) = \int_0^\infty t^s e^{-t} \frac{dt}{t}
\end{equation}
is a Mellin integral, and gives the inverse Mellin integral
\begin{equation}
  e^{-x} = \frac{1}{2 \pi i} \int_{(\sigma)} x^{-s} \Gamma(s) ds.
\end{equation}
Applied to the Dirichlet series $D(s)$, we have
\begin{equation}
  \sum_{n \geq 1} a(n) e^{-n/X} = \frac{1}{2\pi i} \int_{(\sigma)} D(s) X^s \Gamma(s) ds.
\end{equation}


\subsubsection*{Concentrating Integral}
\index{Integral transform!concentrating}

We now produce an integral transform that has the effect of concentrating the mass of the
integral around the parameter $X$.
We claim that
\begin{equation}\label{eq:back:conc_I}
  \frac{1}{2\pi i} \int_{(2)} \exp\left(\frac{\pi s^2}{Y^2}\right) \frac{X^s}{Y}ds =
  \frac{1}{2\pi} \exp\left(-\frac{Y^2 \log^2 X}{4\pi}\right).
\end{equation}

\begin{proof}
  Write $X^s = e^{s\log X}$ and complete the square in the exponents.
  Since the integrand is entire and the integral is absolutely convergent, performing the
  change of variables $s \mapsto s-Y^2 \log X/2\pi$ and shifting the line of integration
  back to the imaginary axis yields
  \begin{equation*}
    \frac{1}{2\pi i} \exp\left( - \frac{Y^2 \log^2 X}{4\pi}\right) \int_{(0)} e^{\pi
    s^2/Y^2} \frac{ds}{Y}.
  \end{equation*}
  The change of variables $s \mapsto isY$ transforms the integral into the standard
  Gaussian, completing the proof.
\end{proof}


Applied to a Dirichlet series $D(s)$, we have
\begin{equation}
  \frac{1}{2\pi} \sum_{n \geq 1} a(n) \exp \bigg( - \frac{Y^2 \log^2 (X/n)}{4\pi} \bigg) =
  \frac{1}{2\pi i} \int_{(\sigma)} \exp \big( \frac{\pi s^2}{Y^s} \big) \frac{X^s}{Y} ds.
\end{equation}
Note in particular that when $\lvert n - X \rvert$ is large, there is significant
exponential decay.
Therefore this integral concentrates the mass of the expression very near $X$ (and in
particular in an interval of width $X/Y$ around $X$).



\subsubsection*{Cutoff Integrals from Smooth, Compactly Supported Functions}
\index{Integral transform!compactly supported transform}
\index{phi@$\Phi_Y(s) and \phi_Y(n)$}


It will also be useful to document a more general family of cutoff transforms.
For $X,Y > 0$, let $\phi_Y(X)$ denote a smooth non-negative function with maximum value $1$,
satisfying
\begin{enumerate}
  \item $\phi_Y(X) = 1$ for $X \leq 1$,
  \item $\phi_Y(X) = 0$ for $X \geq 1 + \frac{1}{Y}$.
\end{enumerate}
Let $\Phi_Y(s)$ denote the Mellin transform of $\phi_Y(x)$, given by
\begin{equation}
  \Phi_Y(s) = \int_0^\infty t^s \phi_Y(t) \frac{dt}{t},
\end{equation}
defined initially for $\Re s > 0$.
Repeated applications of integration by parts and differentiation under the integral shows
that $\Phi_Y(s)$ satisfies the following four properties:
\begin{enumerate}
  \item $\Phi_Y(s) = \frac{1}{s} + O_s(\frac{1}{Y})$,
  \item $\Phi'_Y(s) = -\frac{1}{s^2} + O_s(\frac{1}{Y})$
  \item $\Phi_Y(s) = -\frac{1}{s} \int_1^{1 + \frac{1}{Y}} \phi'_Y(t) t^s dt$,
  \item and for all positive integers $m$, for $s$ constrained within a vertical strip and
    $\lvert s-1 \rvert > \epsilon$, we have
    \begin{equation}
      \Phi_Y(s) \ll_\epsilon \frac{1}{Y} \Big( \frac{Y}{1 + \lvert s \rvert} \Big)^m.
    \end{equation}
\end{enumerate}
Further, the last property can be extended to real $m > 1$ through the
Phragm\'{e}n-Lindel\"{o}f principle.
The Mellin transform pair $\Phi_Y(s), \phi_Y(x)$ gives a general set of integral cutoff
relations,
\begin{equation}
  \sum_{n \leq X} a(n) + \sum_{X < n \leq X + X/Y} a(n) \phi_Y\Big(\frac{n}{X}\Big) =
  \frac{1}{2\pi i} \int_{(\sigma)} D(s) \Phi_Y(s) X^s ds.
\end{equation}

% vim: tw=90
