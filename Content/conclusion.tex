

In the preceding, we considered two problems closely related to the classical Gauss
Circle problem.


We first considered partial sums $S_f(n) = \sum_{m \leq n} a(n)$ of Fourier coefficients
of a cusp form $f(z) = \sum a(n) e(nz)$, and showed that on average these partial sums
behave like the error term in the Gauss Circle problem.
To study these partial sums, we constructed the Dirichlet series
\begin{equation}
  D(s, S_f \times S_g) = \sum_{n \geq 1} \frac{S_f(n) S_g(n)}{n^{s + k - 1}},
\end{equation}
as well as some related Dirichlet series, and showed that these Dirichlet series have
meromorphic continuation to the plane.


The primary challenge comes from understanding the properties of the shifted convolution
sum
\begin{equation}
  \sum_{n,m \geq 1} \frac{a(n+m) a(n)}{(n+m)^s n^w}.
\end{equation}
In Chapter~\ref{c:sums}, we approached this sum through a spectral expansion of a Poincare
series appearing in an inner product against an appropriately chosen product of cuspforms.
We showed that this spectral expansion is, to a large extent, explicitly understandable.
By relating the properties of partial sums of coefficients of cusp forms to a particular
spectral expansion, we explicitly relate the arithmetic properties of the coefficients to
the very analytic properties of the spectrum of the hyperbolic Laplacian.


It is interesting to reflect on the successes and limitations of this approach.
In Chapters~\ref{c:sums} and~\ref{c:sums_apps}, we showed that we are now able to prove
many improvements to classical results.
But we were unable to improve the estimate for the size of a single partial sum.
Instead, we are only capable of matching the estimate due to Hafner and Ivi\'c that
\begin{equation}
  S_f(X) \ll X^{\frac{k-1}{2} + \frac{1}{3} + \epsilon}.
\end{equation}
From the point of view of this thesis, the obstacle to further improvement was directly
seen to be lack of understanding of the discrete spectrum (in terms of their distribution
and cancellation) and the Lindel\"{o}f Hypothesis (or rather subconvexity) type bounds for
Rankin-Selberg convolutions.
Fundamentally, the techniques in this thesis are very different from previously attempted
techniques --- thus the similarity in the final bounds heuristically seems to represent an
absolute bound on our current understanding.




In Chapter~\ref{c:sums_apps} we noted several other applications of the Dirichlet series
$D(s, S_f \times S_g)$, including several projects that have already led to published
papers~\cite{hkldw, hkldwShort, hkldwSigns}.
We have shown that the Dirichlet series $D(s, S_f \times S_g)$ present powerful tools to
study a variety of questions related to the size, shape, and behavior of sums of
coefficients of cusp forms.
Through analysis of $D(s, S_f \times S_g)$, it is possible to prove results on long
interval estimates, short interval estimates, and sign changes.
More generally, one can apply many different Mellin integral transforms to $D(s, S_f
\times S_g)$ in order to directly study different aspects.


One avenue of exploration that my collaborators and I have begun to explore it to perform
an analogous construction of $D(s, S_f \times S_f)$ in cases when $f$ is not a cusp form.
This is proving to be a very interesting direction, and we will be able to explore more
and different variants of the Gauss Circle problem.
One particular direction is discussed at the very end of Chapter~\ref{c:sums_apps}.


There is another interesting direction for further exploration, based on the techniques
and observations from Chapters~\ref{c:sums} and~\ref{c:sums_apps}.
As an individual Fourier coefficient is roughly of size $a(n) \sim n^{\frac{k-1}{2}}$ and
the partial sum appears to satisfy $S_f(n) \ll n^{\frac{k-1}{2} + \frac{1}{4}}$, there is
a large amount of cancellation among individual coefficients.


But I ask the following question: What if we consider partial sums formed from the partial
sums $S_f(n)$?
That is, what should we expect from the sizes of
\begin{equation}
  \sum_{n \leq X} S_f(n)?
\end{equation}
Further, what if we iterate this process and consider sums of sums of sums, and so on?
Initial numerical investigation suggests that there continues to be extreme amounts of
cancellation, far more than would occur by merely random chance.


In Chapter~\ref{c:hyperboloid}, we considered the question of how many points lie within
the $d$-dimensional sphere of radius $\sqrt R$ and on the one-sheeted hyperboloid
\begin{equation}
  \mathcal{H}_{h,d} = X_1^2 + \cdots + X_{d-1}^2 = X_d^2 + h,
\end{equation}
which is essentially a constrained Gauss Sphere problem.
We were able to prove improved bounds and asymptotics for this number of points.


On comparing the main techniques of Chapter~\ref{c:hyperboloid} with those of
Chapter~\ref{c:sums}, it is clear that there are many similarities.
In both, we reduce the study towards sufficient understanding of a carefully chosen
shifted convolution sum.
And in both, we understand the convolution sum by translating the properties into
properties of functions associated to the spectrum of the hyperbolic Laplacian.


The analysis and proof of the main theorem of Chapter~\ref{c:hyperboloid} is not
completely optimized.
In particular, it is possible to perform a very close and detailed analytic argument,
similar to the argument appearing in~\cite{hkldwShort}, in order to further improve the
bound on the main error term of the lattice point estimate.
It may even be possible to improve estimates for the divisor sum $\sum d(n^2 + 1)$, whose
connection with lattice points on hyperboloids is explained in
Chapter~\ref{c:hyperboloid_apps}.


More generally, the ideas and techniques of Chapter~\ref{c:hyperboloid} can be extended to
more general products of theta functions.
By replacing the Jacobi theta function with theta functions associated to different
quadratic forms, it should be possible to understand a wide variety of quadratic surfaces.

% vim: tw=90
