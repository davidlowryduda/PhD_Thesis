

\section{Introduction}\label{sec:hyperboloid_introduction}
\index{hyperboloid, one-sheeted}
\index{H@$\mathcal{H}_{d,h}$}
\index{N@$N_{d,h}(R)$}



A one-sheeted $d$-dimensional hyperboloid $\mathcal{H}_{d,h}$ is a surface satisfying the
equation
\begin{equation}
  X_1^2 + \cdots + X_{d-1}^2 = X_d^2 + h
\end{equation}
for some fixed positive integer $h$.
In this chapter, we investigate the number of integer points lying on the hyperboloid
$\mathcal{H}_{d,h}$.
In particular, we investigate the asymptotics for the number $N_{d,h}(R)$ of integer
points $\bm{m} = (m_1, \ldots, m_d) \in \mathbb{Z}^d$ lying on $\mathcal{H}_{d,h}$ and
within the ball $\| \bm{m} \|^2 \leq R$ for large $R$.
Stated differently, if $B(\sqrt{R})$ is the ball of radius $\sqrt R$ in $\mathbb{R}^d$,
centered at the origin, then
\begin{equation}
  N_{d,h}(R) = \# \big(\mathbb{Z}^d \cap \mathcal{H}_{d,h} \cap B(\sqrt{R})\big).
\end{equation}


Heuristically, one should expect to be capable of determining the leading term asymptotic
using the circle method on hyperboloids $\mathcal{H}_{d,h}$ of sufficiently high
dimension.
More recently, Oh and Shah~\cite{ohshah2014} used ergodic methods to study the
three-dimensional hyperboloid $\mathcal{H}_{3,h}$ when $h$ is a positive square.
They proved the following theorem.
\begin{theorem}{Oh and Shah}
  Suppose that $h$ is a square.
  On $\mathcal{H}_{3,h}$, as $X \to \infty$,
  \begin{equation}
    N_{d,h}(X) = c X^{\frac{1}{2}}\log X + O \big( X^{\frac{1}{2}} (\log X)^{\frac{3}{4}}
    \big)
  \end{equation}
  for some constant $c > 0$.
\end{theorem}


In this chapter, we sharpen and extend this theorem to any dimension $d \geq 3$ and any
integral $h \geq 1$.
We also prove a smoothed analogue, including smaller-order growth terms.
The primary result is the following theorem.


\begin{theorem}\label{theorem:hyperboloid_intro_sharp}
  Let $d \geq 3$ and $h \geq 1$ be integers.
  Let $N_{d,h}(R)$ denote the number of integer points $\bm{m}$ on the hyperboloid
  $\mathcal{H}_{d,h}$ with $\| \bm{m} \|^2 \leq R$.
  Then for any $\epsilon > 0$,
  \begin{align}
    N_{d,h}(R) &= \delta_{[d = 3]} \delta_{[h = a^2]} C'_3 R^{\frac{1}{2}} \log R + C_d
    R^{\frac{d}{2} - 1} + O(R^{\frac{d}{2} - 1 - \lambda(d) + \epsilon}).
  \end{align}
  Here the Kronecker $\delta$ expressions indicate that the first term only occurs if $d =
  3$ and if $h$ is a square, and $\lambda(d) > 0$ is a constant depending only on the
  dimension $d$.
\end{theorem}


This Theorem is presented in greater detail as Theorem~\ref{thm:hyp:sharp_theorem_full},
including the description of $\lambda(d)$.
When $d = 3$, the power savings $\lambda(d)$ is exactly $\frac{1}{20}$.
As $d$ gets larger, $\lambda(d)$ grows and limits towards $\frac{1}{3}$.
Note that there is an error term with polynomial savings, which is a significant
improvement over previous results.
As a corollary, one recovers the Theorem of Oh and Shah.

In addition to the sharp estimate of Theorem~\ref{theorem:hyperboloid_intro_sharp}, we
consider smoothed approximations to $N_{d,h}(R)$.
In~\eqref{eq:hyp:points_equals_sum}, we show that $N_{d,h}(R) = \sum_{2m^2+h \leq R}
r_{d-1}(m^2 + h)$.
Then sums of the form
\begin{equation}
  \sum_{m \in \mathbb{Z}} r_{d-1}(m^2 + h) e^{-\frac{2m^2 + h}{R}}
\end{equation}
count the number of points $\bm{m}$ on $\mathcal{H}_{d,h}$ with exponential decay in $\|
\bm{m} \|$ once $\| \bm{m} \|^2 \geq R$.
This smoothed sum should be thought of as giving a smooth approximation to $N_{d,h}(R)$.
Through the methodology of this chapter, we prove the following smooth estimate.


\begin{theorem}\label{theorem:hyperboloid_intro_smooth}
  Let $d \geq 3$ and $h \geq 1$ be integers.
  Then for each $h$ and $d$, there exist constants $C'$ and $C_m$ such that for any
  $\epsilon > 0$,
  \begin{align}
    &\sum_{m \in \mathbb{Z}} r_{d-1}(m^2 + h) e^{-(2m^2 + h)/X} \\
    &\qquad = \delta_{[d=3]} \delta_{[h = a^2]} C' X^{\frac{1}{2}} \log X + \sum_{0 \leq m
    < \lceil \frac{d}{2} - 1 \rceil} C_{m} X^{\frac{d}{2} - 1 -\frac{m}{2}}
    + O(X^{\frac{d}{4} - \frac{1}{2} + \epsilon}).
  \end{align}
  Here, $\delta_{[\text{condition}]}$ is a Kronecker $\delta$ and evaluates to $1$ if the
  condition is true and $0$ otherwise.
\end{theorem}



See Theorem~\ref{thm:hyperboloid:smooth_full} in \S\ref{sec:hyp:proof_main_theorems} for a
more complete statement.
This Theorem suggests that for dimensions greater than $4$, there may be secondary main
terms with lower power contributions.


Theorems~\ref{theorem:hyperboloid_intro_sharp} and~\ref{theorem:hyperboloid_intro_smooth}
can be thought of as average order estimates of the function $r_{d-1}(m^2+h)$.
In particular, for $2m^2 + h \leq R$, the average value of $r_{d-1}(m^2+h)$ is about
$R^{\frac{d-1}{2}-1}$.
In the process of proving~\ref{theorem:hyperboloid_intro_sharp}, we also prove
that this average order estimate holds on short-intervals, i.e.\ intervals
around $R$ of length much less than $R$.


\begin{theorem}\label{theorem:hyperboloid_intro_short}
  Let $k \geq \frac{1}{2}$ be a full or half-integer.
  Then for each dimension $d$, there is a constant $\lambda(d) > 0$ such that
  \begin{equation}
    \sum_{\lvert 2m^2 + h - X \rvert < X^{1 + \epsilon - \lambda(d)}} r_{d-1}(m^2 + h) \ll
    X^{\frac{d}{2} - 1 + \epsilon - \lambda(d)}.
  \end{equation}
  The constant $\lambda(d)$ is the same constant as in
  Theorem~\ref{theorem:hyperboloid_intro_sharp}.
\end{theorem}


This Theorem can be roughly interpreted to count the number of lattice points $\bm{m}$ on
$\mathcal{H}_{d,h}$ with $\| \bm{m} \|$ very near $X$, or equivalently counting the number
of lattice points within a sphere of radius slightly larger than $\sqrt X$ and outside of
a sphere of radius a slightly smaller than $\sqrt X$.
This Theorem can be compared to the main theorem in short intervals
in~\cite{hkldwShort}, as described in Chapter~\ref{c:sums_apps}.


\begin{remark}
  To make heuristic sense of Theorem~\ref{theorem:hyperboloid_intro_short}, note that
  there are on the order of $X^{\frac{1}{2} + \epsilon - \lambda(d)}$ integers $m$ such
  that $\lvert 2m^2 + h - X \rvert < X^{1 + \epsilon - \lambda(d)}$.
  Therefore, if each of these values of $r_{d-1}(2m^2 + h)$ is approximately the size we
  expect, $X^{\frac{d-1}{2} - 1}$, then the total size should be
  \begin{equation}
    X^{\frac{d-1}{2} - 1} \cdot X^{\frac{1}{2} + \epsilon - \lambda(d)} = X^{\frac{d}{2} -
    1 + \epsilon - \lambda(d)},
  \end{equation}
  which is exactly what is shown in Theorem~\ref{theorem:hyperboloid_intro_short}.
\end{remark}




\subsection*{Overview of Methodology}



In order to count points on hyperboloids, let $d = 2k + 2$ (where $k$ may be a
half-integer).
Then in
\begin{equation}
  X_1^2 + \cdots + X_{2k+1}^2 = X_{2k+2}^2 + h,
\end{equation}
notice that for a point $\bm{X}$ on the hyperboloid,
\begin{equation}
  (X_1^2 + \cdots + X_{2k+1}^2) + X_{2k+2}^2 \leq R \iff 2X_{2k+2}^2 + h \leq R.
\end{equation}
It suffices to consider those points on the hyperboloid with $2X_{2k+2}^2 + h \leq R$.
Recall the notation that $r_d(n)$ is the number of representations of $n$ as a sum of
$d$ squares.
Then, breaking the hyperboloid into each possible value of $X_{2k+2}^2 + h$ and summing
across the number of representations as sums of squares, we have that
\begin{equation}\label{eq:hyp:points_equals_sum}
  N_{d,h}(R) = \sum_{2X_{2k+2}^2 + h \leq R} r_{2k+1}(X_{2k+2}^2 + h) = \sum_{2m^2 + h
  \leq R} r_{2k+1}(m^2 + h).
\end{equation}
We will find the number of points on the hyperboloid by estimating this last sum.


Consider the automorphic function
\index{V@$\widetilde{V}(z)$}
\begin{equation}
  \widetilde{V}(z) = \theta^{2k+1}(z) \overline{\theta(z)}y^{\frac{k+1}{2}},
\end{equation}
where
\index{theta@$\theta(z)$}
\begin{equation}
  \theta(z) = \sum_{n \in \mathbb{Z}} e^{2\pi i n^2 z}
\end{equation}
is the classical Jacobi theta function.
Heuristically, the $h$th Fourier coefficient of $\widetilde{V}(z)$ is a weighted
version of the sum $\sum_m r_{2k+1}(m^2+h)$, and so proper analysis of the the $h$th
Fourier coefficient of $\widetilde{V}(z)$ will give an estimate for $N_{d,h}(R)$.


More completely, let $P_h(z,s)$ denote a Poincar\'e series that isolates the $h$th Fourier
coefficient.
Then we will have that
\begin{equation}
  \frac{(2\pi)^{s + \frac{k-1}{2}}}{\Gamma(s + \frac{k-1}{2})} \langle P_h(\cdot, s),
  \widetilde{V} \rangle = \sum_{m \in \mathbb{Z}}
  \frac{r_{2k+1}(m^2+h)}{(2m^2+h)^{s+\frac{k-1}{2}}}.
\end{equation}
In order to understand the meromorphic properties of this Dirichlet series, we
will use a spectral expansion of the Poincar\'e series and understand each of the terms in
the spectral decomposition.
As $\widetilde{V}(z)$ is not square integrable, it is necessary to modify
$\widetilde{V}(z)$ by cancelling out the growth.
We do that in the next section by subtracting carefully chosen Eisenstein series.


Once the meromorphic properties of this Dirichlet series are understood, it only remains
to perform some classical cutoff integral transforms.
In Section~\ref{sec:hyp:proof_main_theorems}, we apply three Mellin integral
transforms described in Chapter~\ref{c:background} and perform classical
integral analysis in order to prove our main theorems of this chapter.




\section{Altering $\widetilde{V}$ to be Square-Integrable}



From the transformation laws of $\theta(z)$, we see that $\widetilde{V}$ satisfies the
transformation law
\begin{equation}\label{eq:V_transformation_law}
  \widetilde{V}(\gamma z) = \frac{\varepsilon_d^{-2k} \kron{c}{d}^{2k} (cz+d)^k}{\lvert cz+d
  \rvert^k} \widetilde{V}(z)
\end{equation}
for $\gamma = \Big( \begin{smallmatrix} a&b\\c&d \end{smallmatrix} \Big) \in \Gamma_0(4)$,
and where
\begin{equation}
  \varepsilon_d = \begin{cases}
    1 & d \equiv 1 \pmod 4 \\
    i & d \equiv 3 \pmod 4
  \end{cases}
\end{equation}
is the sign of the $d$th Gauss sum.
Therefore when $k$ is an integer, $\widetilde{V}$ is a modular form of full-integral
weight $k$ of nebentypus $\chi(\cdot) = \kron{-1}{\cdot}^k$ on $\Gamma_0(4)$.
When $k$ is a half-integer, $\widetilde{V}$ is a modular form of half-integral weight $k$
on $\Gamma_0(4)$ with a normalized theta multiplier system as described
in~\eqref{eq:V_transformation_law}.


Under the action of $\Gamma_0(4)$, the quotient $\Gamma_0(4)\backslash\mathcal{H}$
has three cusps: at $0, \frac{1}{2}$, and $\infty$.
We use $E_\mathfrak{a}^k(z,w)$ to denote the Eisenstein series of weight $k$ associated to
the cusp $\mathfrak{a}$, as detailed extensively in \S\ref{sec:Eisen_summary}.
We will soon see that $\widetilde{V}$ is non-cuspidal, and we will analyze the behavior of
$\widetilde{V}$ at each of the cusps.
In doing so, we will prove the following.


\begin{proposition}
  For $k \geq 1$, define $V(z)$ as
  \begin{equation}
    V(z) := \widetilde{V}(z) - E_\infty^k(z, \tfrac{k+1}{2}) - E_0^k(z, \tfrac{k+1}{2}).
  \end{equation}
  Then $V(z)$ is in $L^2(\Gamma_0(4)\backslash \mathcal{H}, k)$.

  In the case when $k = \frac{1}{2}$, we define
  \begin{equation}
    V(z) := \widetilde{V}(z) - \const_{w = \frac{3}{4}} E_\infty^k(z, w) - \const_{w =
    \frac{3}{4}} E_0^k(z,w),
  \end{equation}
  where $\const_{w=c} f(w)$ refers to the constant term in the Laurent expansion of $f(w)$
  expanded at $w=c$.
  Then $V(z)$ is in $L^2(\Gamma_0(4)\backslash \mathcal{H}, \frac{1}{2})$.
\end{proposition}



\begin{proof}

Writing $\widetilde{V}$ directly as
\begin{equation}
  \widetilde{V}(z) = \sum_{m_1, \ldots, m_{2k+2} \in \mathbb{Z}} y^{\frac{k+1}{2}} e^{2\pi
  i x(m_1^2 + \cdots m_{2k+1}^2 - m_{2k+2}^2)} e^{-2\pi y(m_1^2 + \cdots + m_{2k+2}^2)}
\end{equation}
shows that all terms have significant exponential decay in $y$, except when $m_1 = \cdots
= m_{2k+2} = 0$, in which case there is the term $y^{\frac{k+1}{2}}$.
Correspondingly, at the $\infty$ cusp, $\widetilde{V}(z)$ grows like $y^{\frac{k+1}{2}}$.
However the function $\widetilde{V}(z) - y^{\frac{k+1}{2}}$ has exponential decay as $y
\to \infty$.


At the $0$ cusp, we use $\sigma_0 = \Big(\begin{smallmatrix} 0&-\frac{1}{2} \\ 2&0
\end{smallmatrix}\Big)$, a matrix in $\SL(2, \mathbb{R})$ taking $0$ to $\infty$, and
directly compute
\begin{equation}
  \widetilde{V}\big|_{\sigma_0}(z) = \theta^{2k+1}\Big(\frac{-1}{4z}\Big)
  \overline{\theta\Big(\frac{-1}{4z}\Big)} \Im^{\frac{k+1}{2}} \Big(\frac{-1}{4z}\Big)
  \frac{\lvert -2iz \rvert^k}{(-2iz)^k}
\end{equation}


At the $\tfrac{1}{2}$ cusp, $\widetilde{V}$ has exponential decay because each $\theta(z)$
factor has exponential decay there.


Thus $\widetilde{V}$ grows like $y^{\frac{k+1}{2}}$ at the $\infty$ and $0$ cusps, and has
exponential decay at the $\frac{1}{2}$ cusp.
To cancel and better understand these growth terms, we subtract spectral Eisenstein series
associated to the cusps $0$ and $\infty$ with spectral parameter chosen so that the
leading growth of the Eisenstein series perfectly cancels the growth of $\widetilde{V}$.
We will use the properties of the full and half-integral weight Eisenstein series
associated to the cusp $\mathfrak{a}$, $E_\mathfrak{a}^k(z,w)$, as described more fully in
\S\ref{sec:Eisen_summary}.
In particular, it is shown in \S\ref{sec:Eisen_summary} that the constant terms in the
Fourier series of the Eisenstein series $E_\mathfrak{a}^k(z,w)$, expanded at the cusp
$\mathfrak{a}$, is of the shape
\begin{equation}
  y^w + c(w)y^{1-w}
\end{equation}
for a constant $c(w)$ depending on $w$.
Therefore, specializing the parameter $w = \frac{k+1}{2}$, the leading term from the
constant term of each Eisenstein series perfectly cancels the growth of $\widetilde{V}$ at
each cusp.
Further, each Eisenstein series is small at each cusp other than its associated cusp, so
for instance $E^k_\infty(z, \tfrac{k+1}{2})$ cancels the $y^{\frac{k+1}{2}}$ at the
$\infty$ cusp and is otherwise small at each other cusp (see~\cite{Iwaniec97} for more).



However, when $k$ is half-integral weight, the Eisenstein series
$E_\mathfrak{a}(z,w)$ has a pole at $w = \frac{3}{4}$.
When $k = \frac{1}{2}$, corresponding to the dimension $3$ hyperboloid,
the two Eisenstein series $E^{\frac{1}{2}}_\infty(z,w)$ and
$E_0^{\frac{1}{2}}(z,w)$ each have poles at
$w = \frac{k+1}{2} = \frac{3}{4}$, and so we cannot subtract
them from $\widetilde{V}$ directly.
Referring again to \S\ref{sec:Eisen_summary}, it is clear that the constant term of the
Laurent expansion at $w = \frac{3}{4}$ of each Eisenstein series contains the leading
growth terms $y^{\frac{3}{4}}$.
Since the constant term in the Laurent expansion is also modular, we conclude the $k =
\frac{1}{2}$ case.
%
\end{proof}




\section{Analytic Behavior}



Let $P_h^k(z,s)$ denote the weight $k$ Poincar\'e series
\begin{equation}
  P_h^k(z,s) = \sum_{\gamma \in \Gamma_\infty \backslash \Gamma_0(4)} \Im(\gamma z)^s
  e^{2\pi i h \gamma z} J(\gamma, z)^{-2k}
\end{equation}
where
\begin{equation}
  J(\gamma, z) = \frac{j(\gamma, z)}{\lvert j(\gamma, z) \rvert}
\end{equation}
and $j(\gamma, z) = \theta(\gamma z)/\theta(z) = \varepsilon^{-1} \kron{c}{d}
(cz+d)^{\frac{1}{2}}$, exactly as for the Eisenstein series defined in
Chapter~\ref{c:background}.


Our basic strategy is to understand the Petersson inner product $\langle P_h^k(\cdot, s),
V(z) \rangle$ in two different ways.
On the one hand, we will compute it directly, giving the Dirichlet series $D_h(s)$ with
coefficients $r_{2k+1}(m^2 + h)$.
On the other hand, we will take a spectral expansion of $P_h^k$ and understand the
meromorphic properties of each part of the spectral expansion.



\subsection{Direct Expansion}\label{ssec:direct_expansion}



We first understand $\langle P_h^k(\cdot, s), V \rangle$ directly, using the method of
unfolding:
\begin{align}
  \langle P_h^k(\cdot, s), V \rangle &= \int \int_{\Gamma_0(4) \backslash \mathcal{H}}
  P_h(z,s) \overline{V(z)} \frac{dx dy}{y^2} \\
  &= \int_0^\infty \int_0^1 y^{s-1} e^{2\pi i h z} \overline{V(z)} dx \frac{dy}{y}.
\end{align}
Initially, we consider the case when $k > \frac{1}{2}$.
We'll consider the three dimensional case, when $k = \frac{1}{2}$, afterwards.



\subsubsection*{Dimension $\geq 4$}

Writing $V = \widetilde{V} - E_\infty^k(z, \frac{k+1}{2}) - E_0^k(z, \frac{k+1}{2})$, we
compute
\begin{align}
  &\langle P_h^k(\cdot, s), E_\infty^k(z, \tfrac{k+1}{2}) + E_0^k(z, \tfrac{k+1}{2})
\rangle \\
  &\quad = \frac{\overline{\rho_\infty^k(h, \frac{k+1}{2}) + \rho_0^k(h,
  \frac{k+1}{2})}}{(4\pi h)^{s-1}} \frac{\Gamma(s + \frac{k}{2} - \frac{1}{2}) \Gamma(s -
\frac{k}{2} - \frac{1}{2})}{\Gamma(s - \frac{k}{2})}.
\end{align}
Expanding $\widetilde{V}$, we can compute the remaining $x$ integral as
\begin{align}
  &\int_0^1 \overline{\widetilde{V}(z)} e^{2\pi i h x} dx = \int_0^1
  \overline{\theta^{2k+1}(z)}\theta(z)y^{\frac{k+1}{2}} e^{2\pi i h x} dx \\
  &\qquad = \sum_{m_1, \ldots, m_{2k+2}} y^{\frac{k+1}{2}} e^{2\pi y(m_1^2 + \cdots +
m_{2k+2}^2)} \int_0^1 e^{-2\pi i x(m_1^2 + \cdots + m_{2k+1}^2 - m_{2k+2}^2 - h)} dx \\
  &\qquad = \quad y^{\frac{k+1}{2}}
  \sum_{\mathclap{\substack{\bm{m} \in \mathbb{Z}^{2k+2} \\
  m_1^2 + \cdots + m_{2k+1}^2 = m_{2k+2}^2 + h}}} e^{-2\pi y(m_1^2 + \cdots m_{2k+2}^2)} =
  \quad y^{\frac{k+1}{2}} \sum_{\mathclap{\substack{\bm{m} \in \mathbb{Z}^{2k+2} \\
  m_1^2 + \cdots + m_{2k+1}^2 = m_{2k+2}^2 + h}}} e^{-2\pi y(2m_{2k+2}^2 + h)} \\
  &\qquad = y^{\frac{k+1}{2}} \sum_{m \in \mathbb{Z}} r_{2k+1}(m^2 + h)
  e^{-2\pi y (2m^2 + h)}.
\end{align}
To go from the penultimate line to the last line, we write $m = m_{2k+2}$ and count the
number of representations of $m^2 + h$.
We compute the remaining $y$ integral
\begin{equation}
  \int_0^\infty y^{s + \frac{k-1}{2}} \sum_{m \in \mathbb{Z}} r_{2k+1}(m^2 + h) e^{-2\pi
  y(2m^2 + h)} \frac{dy}{y} = \sum_{m \in \mathbb{Z}} \frac{r_{2k+1}(m^2 + h)}{(2m^2 +
  h)^{s + \frac{k-1}{2}}} \frac{\Gamma(s + \frac{k-1}{2})}{(2\pi)^{s + \frac{k-1}{2}}}.
\end{equation}



Define
\begin{equation}\label{eq:hyp:Dh_def}
  D^k_h(s) := \frac{(2\pi)^{s + \frac{k-1}{2}}}{\Gamma(s + \frac{k-1}{2})} \langle
  P_h^k(\cdot, s), V\rangle.
\end{equation}
Then our computation above shows that for $k \geq 1$, $D^k_h(s)$ can be written as
\begin{equation}\label{eq:basic_expansion_final}
  \sum_{m \in \mathbb{Z}}\frac{r_{2k+1}(m^2 + h)}{(2m^2 + h)^{s + \frac{k-1}{2}}} -
  \mathfrak{E}_h^k(s)
\end{equation}
where
\begin{equation}
  \mathfrak{E}_h^k(s) =  \frac{(2\pi)^{\frac{k+1}{2}} (\overline{\rho_\infty^k(h,
  \frac{k+1}{2}) + \rho_0^k(h, \frac{k+1}{2})})}{(2 h)^{s-1}} \frac{\Gamma(s - \frac{k}{2}
- \frac{1}{2})}{\Gamma(s - \frac{k}{2})}.
\end{equation}
Notice that $\mathfrak{E}_h^k(s)$ has poles at $s = \frac{k+1}{2} - m$ for $m \in
\mathbb{Z}_{\geq 0}$, coming from the Gamma function in the
numerator, and clear meromorphic continuation.\footnote{$\mathfrak{E}$ is an E in an old
  German font, which many mathematicians would pronounce as ``fraktur E'' or ``mathfrak
E.'' We use it because those terms come from Eisenstein series.}



After applying Stirling's approximation to estimate the Gamma functions, we have the
following proposition.


\begin{proposition}\label{prop:nonspectral_analytic_props_large_dim}
  With the notation above and with $k \geq 1$, we have
  \begin{equation}
    D^k_h(s) := \frac{(2\pi)^{s + \frac{k-1}{2}}}{\Gamma(s + \frac{k-1}{2})} \langle
    P_h^k(\cdot, s), V\rangle = \sum_{m \in \mathbb{Z}}\frac{r_{2k+1}(m^2 + h)}{(2m^2 +
  h)^{s + \frac{k-1}{2}}} - \mathfrak{E}_h^k(s).
  \end{equation}
  The function $\mathfrak{E}_h^k(s)$ is analytic for $\Re s > \frac{1}{2}$ except for
  simple poles at $s = \frac{k+1}{2} - m$ for $m \in \mathbb{Z}_{\geq 0}$, and has
  meromorphic continuation to the plane.
  For $\Re s > \frac{1}{2}$, we have the bound
  \begin{equation}
    \mathfrak{E}_h^k(s) \ll (1 + \lvert s \rvert)^{-\frac{1}{2}}.
  \end{equation}
\end{proposition}




\subsubsection*{Dimension $3$}



We proceed analogously, and write
$V = \widetilde{V} - \const_{w = \frac{3}{4}} E_\infty^{\frac{1}{2}}(z, w) -
\const_{w = \frac{3}{4}} E_0^{\frac{1}{2}} (z, w)$, initially for $\Re s \gg 1$.
We now need to compute
\begin{equation}
  \langle P_h^{\frac{1}{2}} (\cdot, s), \const_{w = \frac{3}{4}}
  E_\mathfrak{a}^{\frac{1}{2}} (\cdot, w) \rangle
  =
  \const_{w = \frac{3}{4}} \; \langle P_h^{\frac{1}{2}} (\cdot, s),
  E_\mathfrak{a}^{\frac{1}{2}} (\cdot, w) \rangle.
\end{equation}
Computing the expression for the inner product
$\langle P_h^{\frac{1}{2}} , E_\mathfrak{a}^{\frac{1}{2}}  \rangle$
for $\Re s \gg 1$ directly and then taking the constant term in $w$ gives that
\begin{equation}\label{eq:dim_3_constant_laurent}
  \const_{w = \frac{3}{4}} \;
  \langle P_h^{\frac{1}{2}} (\cdot, s), E_\mathfrak{a}^{\frac{1}{2}} (\cdot, w) \rangle
  =
  \const_{w = \frac{3}{4}}
  \frac{\overline{\rho_{\mathfrak{a}}^{\frac{1}{2}} (h,w)}}{(4\pi h)^{s-1}}
  \frac{\Gamma(s+\overline{w}-1)\Gamma(s-\overline{w})}{\Gamma(s - \frac{1}{4})}.
\end{equation}
We must now make sense of this constant term.



In \S\ref{sec:Eisen_summary}, it is shown that $\rho_\mathfrak{a}^{\frac{1}{2}}(h,w)$
has a simple pole at $w = \frac{3}{4}$ if and only if $h$ is a positive square,
and otherwise is analytic.
Thus the constant term~\eqref{eq:dim_3_constant_laurent} manifests in three ways:
\begin{enumerate}
  \item The constant terms of $\rho_{\mathfrak{a}}^{\frac{1}{2}}(h,w)$,
    $\Gamma(s + \overline{w} - 1)$, and $\Gamma(s - \overline{w})$
  \item The residue term of $\rho_{\mathfrak{a}}^{\frac{1}{2}}(h,w)$,
    the constant term of $\Gamma(s + \overline{w} - 1)$,
    and the linear term of $\Gamma(s - \overline{w})$
  \item The residue term of $\rho_{\mathfrak{a}}^{\frac{1}{2}}(h,w)$,
    the linear term of $\Gamma(s + \overline{w} - 1)$,
    and the constant term of $\Gamma(s - \overline{w})$.
\end{enumerate}
Together, these mean that~\eqref{eq:dim_3_constant_laurent} can be written as
\begin{equation} \label{eq:dim_3_laurent_simplify_I}
  \begin{split}
    &\const_{w = \frac{3}{4}} \; \langle P_h^{\frac{1}{2}} (\cdot, s),
    E_\mathfrak{a}^{\frac{1}{2}} (\cdot, w) \rangle = \\
    &\quad = \frac{\const_{w = \frac{3}{4}}
    \rho_\mathfrak{a}^{\frac{1}{2}}(h,w)
    \Gamma(s - \frac{1}{4})\Gamma(s - \frac{3}{4})}
    {(4\pi h)^{s-1} \Gamma(s - \frac{1}{4})}
    \\
    &\qquad + \frac{\Res_{w = \frac{3}{4}}
    \rho_{\mathfrak{a}}^{\frac{1}{2}}(h,w)}{(4\pi h)^{s-1}}
    \bigg( \frac{\Gamma'(s - \frac{1}{4}) \Gamma(s - \frac{3}{4})}
      {\Gamma(s - \frac{1}{4})}
      +
    \frac{\Gamma'(s - \frac{3}{4}) \Gamma(s - \frac{1}{4})}{\Gamma(s - \frac{1}{4})}\bigg)
    \\
    &\quad =\frac{\const_{w = \frac{3}{4}} \rho_\mathfrak{a}^{\frac{1}{2}}(h,w)
    \Gamma(s - \frac{3}{4})}{(4\pi h)^{s-1}}
    \\
    &\qquad + \frac{\Res_{w = \frac{3}{4}}
    \rho_{\mathfrak{a}}^{\frac{1}{2}}(h,w)}{(4\pi h)^{s-1}}
    \bigg( \frac{\Gamma'(s - \frac{1}{4}) \Gamma(s - \tfrac{3}{4})}
      {\Gamma(s - \frac{1}{4})}
      +
    \Gamma'(s - \tfrac{3}{4})\bigg).
  \end{split}
\end{equation}
This expression has clear meromorphic continuation to the plane, and the poles and
analytic behavior can be determined from the individual Gamma functions.
This expression has a simple pole at $s = \frac{3}{4}$, and when
$\rho_\mathfrak{a}^{\frac{1}{2}}(s,w)$ has a pole at $w = \frac{3}{4}$, this
expression has a double pole in $s$ at $s = \frac{3}{4}$ coming
from $\Gamma'(s - \frac{3}{4})$.
These are the only poles in this expression when $\Re s > \frac{1}{2}$.




As in the case when $k \geq 1$, we define
\begin{equation}\label{eq:hyp:Dh_def_half}
  D_h^{\frac{1}{2}}(s) :=
  \frac{(2\pi)^{s - \frac{1}{4}}}{\Gamma(s - \frac{1}{4})}
  \langle P_h^{\frac{1}{2}}(\cdot, s), V \rangle.
\end{equation}
At each cusp $\mathfrak{a}$, we also define
\begin{equation}
  \mathfrak{E}_{h, \mathfrak{a}}^{\frac{1}{2}}(s)
  :=
  \frac{(2\pi)^{s - \frac{1}{4}}}{\Gamma(s - \frac{1}{4})} \const_{w = \frac{3}{4}}
  \langle P_h^{\frac{1}{2}}(\cdot, s), E_\mathfrak{a}^{\frac{1}{2}}(\cdot, w) \rangle.
\end{equation}
Notice that this is $(2\pi)^{s-\frac{1}{4}} \Gamma(s - \frac{1}{4})^{-1}$ times
the expression in~\eqref{eq:dim_3_laurent_simplify_I}.
Finally, define
\begin{equation}
  \mathfrak{E}_h^{\frac{1}{2}}(s)
  :=
  \mathfrak{E}_{h, \infty}^{\frac{1}{2}}(s) + \mathfrak{E}_{h, 0}^{\frac{1}{2}}(s).
\end{equation}
Then when $k \geq 1$, we have that
\begin{equation}
  D^{\frac{1}{2}}_h(s) = \frac{(2\pi)^{s - \frac{1}{4}}}{\Gamma(s - \frac{1}{4})}
  \langle P_h^{\frac{1}{2}}(\cdot, s), V\rangle
  =
  \sum_{m \in \mathbb{Z}}
  \frac{r_{3}(m^2 + h)}{(2m^2 + h)^{s - \frac{1}{4}}}
  -
  \mathfrak{E}_h^{\frac{1}{2}}(s).
\end{equation}
Although the intermediate steps are different, this final notation agrees with the
notation for $k \geq 1$.



To roughly understand the growth of $\Gamma'(s)$, it suffices to use Cauchy's
Integral Formula by examining (for $\lvert \Im s \rvert \gg 1$)
\begin{equation}
  \Gamma'(s) = \frac{1}{2\pi i} \int_{B_1(s)} \frac{\Gamma(z)}{(z - s)^2}dz \ll \max_{0
  \leq \theta \leq 2\pi} \lvert \Gamma(s + e^{i \theta}) \rvert,
\end{equation}
where $B_1(s)$ is the circle of radius $1$ around $s$.
It is then straightforward to get rough bounds on $\Gamma'(s)$ through
Stirling's Approximation.


\begin{remark}
  It is possible to get much stronger bounds, but it will turn out that this
  rough bound suffices.
\end{remark}


We gather the relevant details from this section into the following proposition.


\begin{proposition}\label{prop:nonspectral_analytic_props_dim_3}
  With the notation above,
  \begin{equation}
    \begin{split}
      D^{\frac{1}{2}}_h(s) = \frac{(2\pi)^{s - \frac{1}{4}}}{\Gamma(s - \frac{1}{4})}
      \langle P_h^{\frac{1}{2}}(\cdot, s), V\rangle
      =
      \sum_{m \in \mathbb{Z}}
      \frac{r_{3}(m^2 + h)}{(2m^2 + h)^{s - \frac{1}{4}}}
      -
      \mathfrak{E}_h^{\frac{1}{2}}(s).
    \end{split}
  \end{equation}
  Further, $\mathfrak{E}_h^{\frac{1}{2}}(s)$ has meromorphic continuation to the plane,
  and is analytic for $\Re s > \frac{1}{2}$ except for a pole at $s = \frac{3}{4}$.
  If $h$ is a square, this is a double pole.
  If $h$ is not a square, then this is a simple pole.

  For $\Re s > \frac{1}{2}$, we have the bound
  \begin{equation}
    \mathfrak{E}_h^{\frac{1}{2}}(s) \ll (1 + \lvert s \rvert)^{\frac{1}{2}}.
  \end{equation}
\end{proposition}




\subsection{Spectral Expansion}

Fix an orthonormal basis of Maass forms of weight $k$ for $L^2(\Gamma_0(4)\backslash
\mathcal{H}, k)$.
This basis consists of Maass forms $\{\mu_j(z)\}$ of types $\frac{1}{2} + it_j$ and
corresponding eigenvalues $\frac{1}{4} + t_j^2$, each with expansion
\begin{equation}
  \mu_j(z) = \sum_{n \neq 0} \rho_j(n) W_{\frac{n}{\lvert n \rvert} \frac{k}{2},
  it_j}(4\pi \lvert n \rvert y) e^{2\pi i n x},
\end{equation}
as well as a finite number of Maass forms $\mu_{j,\ell}(z)$ with eigenvalues
$\frac{\ell}{2}(1 - \frac{\ell}{2})$ with $1 \leq \ell \leq k$ for $\ell$ a (possibly
half-integer) satisfying $\ell \equiv k \bmod 2$, each with expansion
\begin{equation}
  \mu_{j, \ell}(z) = \sum_{n \neq 0} \rho_{j,\ell}(n) W_{\frac{n}{\lvert n \rvert}
  \frac{k}{2}, \frac{\ell-1}{2}}(4\pi \lvert n \rvert y) e^{2\pi i n x}.
\end{equation}
Note that for each $\ell$, there are finitely many such Maass forms, as these come from
holomorphic cusp forms of weight $\ell$.
These Maass forms contribute the so-called \emph{bottom of the spectrum}, as described
in~\cite[Chapter 3]{GoldfeldHundleyI}.



Then $P_h^k$ has a Selberg Spectral decomposition~\cite{IwaniecKowalski04,
Goldfeld2006automorphic} of the form
\begin{align}
  P_h^k(z,s) &= \sum_j \langle P_h^k(\cdot, s), \mu_j \rangle \mu_j(z) + \sum_{\frac{1}{2}
  \leq \ell \leq k} \sum_j \langle P_h^k(\cdot, s), \mu_{j, \ell}\rangle \mu_{j, \ell}(z)
  \label{line:1:Ph_discrete} \\
  &\quad + \sum_{\mathfrak{a}} \langle P_h^k(\cdot, s), R_{\mathfrak{a}} \rangle
  R_{\mathfrak{a}}(z) \label{line:2:Ph_residual} \\
  &\quad + \frac{1}{4\pi i} \sum_{\mathfrak{a}} \int_{(\frac{1}{2})} \langle P_h^k(\cdot,
  s), E_{\mathfrak{a}}^k (\cdot, u)\rangle E_{\mathfrak{a}}^k (z, u) du.
  \label{line:3:Ph_continuous}
\end{align}
In this expansion, line~\eqref{line:1:Ph_discrete} is the \emph{discrete part of the
spectrum}, line~\eqref{line:2:Ph_residual} is the \emph{residual part of the spectrum},
and line~\eqref{line:3:Ph_continuous} is the \emph{continuous part of the spectrum}.
\index{spectrum!discrete part}
\index{spectrum!residual part}
\index{spectrum!continuous part}
The sums over $\mathfrak{a}$ are sums over the three cusps of $\Gamma_0(4)$.
Note that the residual part of the spectrum exists only when $k$ is a half-integer, in
which case
\begin{equation}\label{eq:hyp:residual_spec_def}
  R_{\mathfrak{a}}^k(z) = \Res_{w = \frac{3}{4}} E_{\mathfrak{a}}^k(z, w).
\end{equation}


Each of the inner products against $P_h^k$ can be directly evaluated.
These computations are very similar to those computations in \S\ref{sec:Eisen_summary} and
\S\ref{ssec:direct_expansion}, and we omit them.
We collect these together in the following lemma.


\begin{lemma}\label{lem:inner_product_list}
  Maintaining the notation above, we have
  \begin{align}
    \langle P_h^k(\cdot, s), \mu_j \rangle &= \frac{\overline{\rho_j(h)}}{(4\pi h)^{s-1}}
    \frac{\Gamma(s - \tfrac{1}{2} + it_j) \Gamma(s - \tfrac{1}{2} - it_j)}{\Gamma(s -
    \frac{k}{2})} \\
    \langle P_h^k(\cdot, s), \mu_{j,\ell} \rangle &=
    \frac{\overline{\rho_{j,\ell}(h)}}{(4\pi h)^{s-1}} \frac{\Gamma(s + \frac{\ell}{2} -
    1) \Gamma(s - \frac{\ell}{2})}{\Gamma(s - \frac{k}{2})} \\
    \langle P_h(\cdot, s), R^k_{\mathfrak{a}} \rangle &= \frac{\Res_{w =
    \frac{3}{4}}\overline{\rho_{\mathfrak{a}}^k(h,w)}}{(4\pi h)^{s - 1}}  \frac{\Gamma(s -
    \frac{1}{4}) \Gamma(s - \frac{3}{4})}{\Gamma(s - \frac{k}{2})} \\
    \langle P_h(\cdot, s), E_{\mathfrak{a}}^k(\cdot, \tfrac{1}{2} + it)\rangle &=
    \frac{\overline{\rho_\mathfrak{a}(h, \tfrac{1}{2} + it)}}{(4\pi h)^{s-1}}
    \frac{\Gamma(s - \tfrac{1}{2} + it) \Gamma(s - \tfrac{1}{2} - it)}{\Gamma(s -
    \frac{k}{2})}.
  \end{align}
  Here, $\rho_j(h)$ is the $h$th coefficient of $\mu_j$, $\rho_{j, \ell}(h)$ is the $h$th
  coefficient of $\mu_{j,\ell}$, $\rho_{\mathfrak{a}}(h, \tfrac{1}{2} + it)$ is the
  $h$th coefficient of $E_{\mathfrak{a}}^k(z, \tfrac{1}{2} + it)$, and $R_\mathfrak{a}^k$
  is as in~\eqref{eq:hyp:residual_spec_def}.
\end{lemma}



\subsection{Meromorphic Continuation of $\langle P_h^k, V \rangle$}



In order to provide a meromorphic continuation for $D_h^k(s)$ (defined
in~\eqref{eq:hyp:Dh_def} and~\eqref{eq:hyp:Dh_def_half}), we provide a meromorphic
continuation for the expression coming from the spectral decomposition of $P_h^k(z,s)$.
Inserting the spectral decomposition of $P_h^k(z,s)$ into $\langle P_h^k(\cdot, s), V
\rangle$, we get
\begin{align}
  \langle P_h^k(\cdot, s), V \rangle &= \sum_j \langle P_h^k(\cdot, s), \mu_j \rangle
  \langle \mu_j, V \rangle + \sum_{\frac{1}{2} \leq \ell \leq k} \sum_j \langle
  P_h^k(\cdot, s), \mu_{j,\ell}\rangle \langle \mu_{j,\ell}, V \rangle
  \label{line:Ph_spectral_discrete} \\
  &\quad + \sum_{\mathfrak{a}} \langle P_h^k(\cdot, s), R_\mathfrak{a}\rangle \langle
  R_\mathfrak{a}, V \rangle \label{line:Ph_spectral_residual} \\
  &\quad + \frac{1}{4\pi i} \sum_{\mathfrak{a}} \int_{(\frac{1}{2})} \langle P_h^k(\cdot,
  s), E^k_{\mathfrak{a}}(\cdot, u) \rangle \langle E^k_{\mathfrak{a}}(\cdot, u), V \rangle
  du, \label{line:Ph_spectral_continuous}
\end{align}
where we have again separated the expressions into separate lines for the discrete
spectrum, the residual spectrum, and the continuous spectrum.
To study the meromorphic continuation, we provide separate meromorphic continuations for
the discrete spectrum, the residual spectrum, and the continuous spectrum in turn.



\subsubsection*{Discrete Spectrum}
\index{spectrum!discrete part}

Consider the discrete spectrum appearing in line~\eqref{line:Ph_spectral_discrete}, which
we rewrite for convenience:
\begin{equation}
  \langle P_h^k(\cdot, s), V \rangle = \sum_j \langle P_h^k(\cdot, s), \mu_j \rangle
  \langle \mu_j, V \rangle + \sum_{\frac{1}{2} \leq \ell \leq k} \sum_j \langle
  P_h^k(\cdot, s), \mu_{j,\ell}\rangle \langle \mu_{j,\ell}, V \rangle.
\end{equation}
Analysis of the discrete spectrum naturally breaks into two categories: analysis of the
finitely many Maass forms $\{\mu_{j, \ell}\}_{j, \ell}$ coming from holomorphic cusp
forms of weight $\ell$, and analysis of the infinitely many Maass forms $\{\mu_j\}_{j}$
with corresponding types $\tfrac{1}{2} = it_j$.
Taking inspiration from~\cite[\S3.10]{GoldfeldHundleyI}, we refer to the Maass
forms $\{\mu_{j, \ell}\}_{j, \ell}$ as the \emph{old discrete spectrum}, and
the Maass forms $\{\mu_j\}_j$ as the \emph{new discrete spectrum}.
\index{discrete spectrum!old}
\index{discrete spectrum!new}
We will prove the following proposition in this section.


\begin{proposition}\label{prop:hyperboloid_discrete_props}
  Write $s = \sigma + it$.
  The discrete spectrum component has meromorphic continuation to the plane, and
  \begin{enumerate}

    \item For $\Re s > \frac{1}{2}$, the new discrete spectrum is analytic and satisfies
      the bound
      \begin{equation}
        \sum_j \langle P_h^k(\cdot, s), \mu_j\rangle \langle \mu_j, V \rangle \ll_{h,
        \epsilon} (1 + \lvert t \rvert)^{\frac{3k}{2} + \frac{5}{2} + \epsilon}
        e^{-\frac{\pi}{2} \lvert t \rvert}. \qquad (\tfrac{1}{2} < \Re s < 1)
      \end{equation}
      The new discrete spectrum has a line of poles on $\Re s = \frac{1}{2}$.

    \item For $\Re s > \frac{k}{2} - 1$, the old discrete spectrum is analytic and
      satisfies the bound
      \begin{equation}
        \sum_{\frac{1}{2} \leq \ell \leq k} \sum_j \langle P_h^k(\cdot, s),
        \mu_{j,\ell}\rangle \langle \mu_{j,\ell}, V \rangle \ll (1 + \lvert t
      \rvert)^{\frac{k}{2} - \frac{3}{2} + \sigma} e^{-\frac{\pi}{2}\lvert t \rvert}.
      \end{equation}
      In the region $\Re s > 0$, the old discrete spectrum has simple poles at $s =
      \frac{k}{2} - 1 - m$ for $m \in \mathbb{Z}_{\geq 0}$.

  \end{enumerate}
\end{proposition}


In order to prove this proposition, we prove a sequence of lemmata.
%Yep, still an awesome word
We first bound the infinite sum in the new discrete spectrum.


\begin{lemma}\label{lem:discrete_newspectrum_innerproduct_bound}
  With $\mu_j$ coming from the new discrete spectrum, and with the same notation as above,
  we have
  \begin{equation}
    \sum_{T \leq \lvert t_j \rvert \leq 2T} \rho_j(h) \langle \mu_j, V \rangle = \sum_{T
    \leq \lvert t_j \rvert \leq 2T} \rho_j(h) \langle \mu_j, \theta^{2k+1}
    \overline{\theta} y^{\frac{k+1}{2}} \rangle \ll T^{k + 3 + \epsilon}.
  \end{equation}
\end{lemma}

\begin{proof}
  First note that $E^k_\mathfrak{a}(z, \tfrac{k+1}{2})$ is orthogonal to the cusp form
  $\mu_j$, which gives the first equality.
  We recognize $\theta(z) y^{1/4}$ as a constant times $\Res_{u = \frac{3}{4}}
  E_\infty^{\frac{1}{2}}(z,u)$.
  Performing this on $\overline{\theta}$ transforms each inner product into
  \begin{equation}
    \Res_{u = \frac{3}{4}} \langle \mu_j, \theta^{2k+1}
    \overline{E_\infty^{\frac{1}{2}}(\cdot,u)} y^{\frac{2k+1}{4}}\rangle.
  \end{equation}
  Using the standard unfolding argument on the Eisenstein series (which uses similar
  methodology to the computations of $\langle P, V \rangle$ in
  \S\ref{ssec:direct_expansion}), taking the residue at $u = \frac{3}{4}$, and using
  Stirling's approximation shows that
  \begin{equation}
    \langle \mu_j, \theta^{2k+1} \overline{\theta} y^{\frac{k+1}{2}} \rangle \ll \rho_j(h)
    (1 + \lvert t_j \rvert)^k e^{-\pi \lvert t_j \rvert}.
  \end{equation}
  As noted in~\cite{hulseCountingSquare}, it is possible to understand bound $\rho_j(h)$
  on average over $j$.
  Through the standard Kuznetsof Trace Formula and arguing as in~\cite{HoffsteinHulse13},
  one can show that $\rho_j(h) \ll_\epsilon e^{\frac{\pi}{2}\lvert t_j \rvert}(1 + \lvert
  t_j \rvert)^\epsilon$ (on average) for full integral weight $k$.
  By using Proskurin's generalization of the Kuznetsof Trace Formula, described
  in~\cite{Duke88}, one can show the same for half integral weight $k$.
  (This is closely related to an argument in~\cite{hulseCountingSquare}).


  Therefore we have that
  \begin{equation}
    \langle \mu_j, \theta^{2k+1} \overline{\theta} y^{\frac{k+1}{2}} \rangle \ll (1 +
    \lvert t_j \rvert)^{k + 1 + \epsilon} e^{-\frac{\pi}{2} \lvert t_j \rvert}
  \end{equation}
  on average.
  Recalling that there are $O(T^2 \log T)$ Maass forms with $T \leq \lvert t_j \rvert \leq
  2T$, summing these terms together gives
  \begin{equation}
    \sum_{T \leq \lvert t_j \rvert \leq 2T} \rho_j(h) \langle \mu_j, \theta^{2k+1}
    \overline{\theta} y^{\frac{k+1}{2}} \rangle \ll (1 + \lvert t_j \rvert)^{k + 3 +
    \epsilon}.
  \end{equation}
  This concludes the proof.
\end{proof}


It is also necessary to note that Selberg's Eigenvalue Conjecture is known for weight $k$
Maass forms on $\Gamma_0(4)$.


\begin{lemma}
  Suppose $\mu_j$ is a Maass form appearing in the old discrete spectrum described above.
  Denote the type of $\mu_j$ by $\frac{1}{2} + it_j$, so that $\mu_j$ is an eigenfunction
  of the Laplacian with eigenvalue $\lambda = \frac{1}{4} + t_j^2$.
  Then $\lambda > \frac{1}{4}$.
  That is, there are no exceptional eigenvalues.
\end{lemma}


\begin{proof}
  This is an argument given by Gergely Harcos at the Alfr\'ed R\'enyi Institute of
  Mathematics in an answer at MathOverflow~\cite{GHfromMO}.
  We sketch the argument here.
  A half-integral weight Maass form with eigenvalue $(1 - t^2)/4$ has a Shimura lift to an
  integral weight Maass form on $\Gamma_0(1)$ with eigenvalue $\frac{1}{4} - t^2$.
  As Selberg's Eigenvalue Conjecture is known in this case (see~\cite{Blomer2013}), this
  completes the proof.
\end{proof}


Now fix $s = \sigma + it$ with $\frac{1}{2} < \sigma < 1$.
We write $\lvert t_j \rvert \sim T$ to mean $T \leq \lvert t_j \rvert \leq 2T$ for the
rest of this section.
Then
\begin{equation}
  \sum_{\lvert t_j \rvert \sim T} \langle P_h^k(\cdot, s), \mu_j \rangle \langle \mu_j, V
  \rangle = \sum_{\lvert t_j \rvert \sim T} \frac{\Gamma(s - \frac{1}{2} + it_j) \Gamma(s
- \frac{1}{2} - it_j)}{(4\pi h)^{s - 1} \Gamma(s - \frac{k}{2})} \rho_j(h) \langle \mu_j,
V \rangle.
\end{equation}
In this expression it is clear that the rightmost poles in $s$ are at $s = \frac{1}{2} \pm
it_j$, which occur on the line $\Re s = \frac{1}{2}$.
By Stirling's Approximation, this is asymptotically
\begin{equation}\label{eq:discrete_newspectrum_stepI}
  \sum_{\lvert t_j \rvert \sim T} \frac{(1 + \lvert t + t_j \rvert)^{\sigma - 1} (1 +
  \lvert t - t_j \rvert)^{\sigma - 1}}{(4\pi h)^{\sigma - 1 + it}(1 + \lvert t
\rvert)^{\sigma - \frac{1}{2} - \frac{k}{2}}} e^{-\frac{\pi}{2}(\lvert t + t_j \rvert +
\lvert t-t_j \rvert - \lvert t \rvert)} \rho_j(h) \langle \mu_j, V \rangle.
\end{equation}
Examination of the exponential contribution shows that there is large exponential decay in
$t_j$ when $\lvert t_j \rvert > \lvert t \rvert$, so we only need to investigate the
convergence when $\lvert t_j \rvert \leq \lvert t \rvert$.
Then~\eqref{eq:discrete_newspectrum_stepI} is bounded by
\begin{equation}
  O_h\Big( \sum_{\lvert t_j \rvert \sim T} (1 + \lvert t \rvert)^{\frac{k}{2} -
  \frac{1}{2}} e^{-\frac{\pi}{2} \lvert t \rvert} \rho_j(h) \langle \mu_j, V \rangle
\Big),
\end{equation}
which, by Lemma~\ref{lem:discrete_newspectrum_innerproduct_bound}, is bounded by
\begin{equation}
  O_{h, \epsilon} \Big( (1 + \lvert t \rvert)^{\frac{3}{2}k + \frac{5}{2} + \epsilon}
  e^{-\frac{\pi}{2}\lvert t \rvert}\Big).
\end{equation}
Summing dyadically gives the first part of the proposition.


For the second part of the proposition, recall the inner product from
Lemma~\ref{lem:inner_product_list}
\begin{equation}
  \langle P_h^k(\cdot, s), \mu_{j, \ell} \rangle =
  \frac{\overline{\rho_{j,\ell}(h)}}{(4\pi h)}^{s-1} \frac{\Gamma(s + \frac{\ell}{2} - 1)
  \Gamma(s - \frac{\ell}{2})}{\Gamma(s - \frac{k}{2})}.
\end{equation}
As the sum over $j$ and $\ell$ are each finite for the old discrete spectrum, the analytic
properties in $s$ can be read directly from the inner products $\langle P_h^k, \mu_{j,
\ell} \rangle$.
The leading poles all come from the Gamma function $\Gamma(s - \frac{\ell}{2})$ in the
numerator.
Note that the first apparent pole is at $s = \frac{k}{2}$ is cancelled by the Gamma
function in the denominator, but there are poles at $s = \frac{k}{2} - 1 - m$ for $m \in
\mathbb{Z}_{\geq 0}$.



\subsubsection*{Residual Spectrum}
\index{spectrum!residual part}



Consider the residual spectrum appearing in line~\eqref{line:Ph_spectral_residual}, which
we rewrite for convenience:
\begin{equation}
  \sum_{\mathfrak{a}} \langle P_h^k(\cdot, s), R_{\mathfrak{a}} \rangle \langle
  R_{\mathfrak{a}}, V \rangle.
\end{equation}
The residual spectrum only occurs when $k$ is a half-integer.
For each cusp, $\langle R_{\mathfrak{a}}, V \rangle$ evaluates to some constant and
doesn't affect the analysis in $s$.
Referring to Lemma~\ref{lem:inner_product_list}, we see that
\begin{equation}
  \langle P_h^k(\cdot, s), R_{\mathfrak{a}} \rangle = \frac{\Res_{w = \frac{3}{4}}
  \overline{\rho_\mathfrak{a}^k(h,w)}}{(4\pi h)^{s-1}} \frac{\Gamma(s - \frac{1}{4})
  \Gamma(s - \frac{3}{4})}{\Gamma(s - \frac{k}{2})}.
\end{equation}
As described in~\eqref{eq:Eisenstein_halfweight_generalshape_coeffs}, the residue in
$\rho_\mathfrak{a}^k(h,w)$ comes from a potential pole in
\begin{equation}
  L\Big(2w - \tfrac{1}{2}, \big( \tfrac{h (-1)^{k - \frac{1}{2}}}{\cdot}\big) \Big)
\end{equation}
at $w = \frac{3}{4}$.
This $L$-function has a pole if and only if the character is trivial, which occurs if and
only if $h$ is a square, and $k = \frac{1}{2} + 2m$ for some $m \in \mathbb{Z}_{\geq 0}$.


Therefore, if $h$ is not a square or if $k$ is not of the form $\frac{1}{2} + 2m$, then
the residual spectrum vanishes.
If $h$ is a square and $k = \frac{1}{2} + 2m$, then the residue is nonzero, and the
analytic properties of the residual spectrum can be read directly from
\begin{equation}\label{eq:hyperboloid:residual_analytic_descriptor}
  \frac{\Gamma(s - \frac{1}{4}) \Gamma(s - \frac{3}{4})}{(4\pi h)^{s-1} \Gamma(s -
  \frac{k}{2})}.
\end{equation}
We codify this in a proposition.


\begin{proposition}\label{prop:hyperboloid_residual_props}
  The residual spectrum in line~\eqref{line:Ph_spectral_residual} vanishes unless $h$ is a
  square and $k = \frac{1}{2} + 2m$ for some $m \in \mathbb{Z}_{\geq 0}$, in which case
  the residual spectrum has meromorphic continuation to the plane and is analytic for $\Re
  s > 0$, except possibly for poles at $s = \frac{3}{4}$ and $s = \frac{1}{4}$.
\end{proposition}



\subsubsection*{Continuous Spectrum}
\index{spectrum!continuous part}



Consider the continuous spectrum appearing in line~\eqref{line:Ph_spectral_continuous},
which we rewrite for convenience:
\begin{equation}
  \frac{1}{4\pi i} \sum_{\mathfrak{a}} \int_{(\frac{1}{2})} \langle P_h^k(\cdot, s),
  E^k_{\mathfrak{a}}(\cdot, u) \rangle \langle E^k_{\mathfrak{a}}(\cdot, u), V \rangle du.
\end{equation}
Referring to Lemma~\ref{lem:inner_product_list}, we see that
\begin{equation}
  \langle P_h(\cdot, s), E_{\mathfrak{a}}^k(\cdot, \tfrac{1}{2} + it)\rangle =
  \frac{\overline{\rho_\mathfrak{a}(h, \tfrac{1}{2} + it)}}{(4\pi h)^{s-1}} \frac{\Gamma(s
  - \tfrac{1}{2} + it) \Gamma(s - \tfrac{1}{2} - it)}{\Gamma(s - \frac{k}{2})}.
\end{equation}
The Gamma functions can be approximated through Stirling's Approximation.
Each $\rho_\mathfrak{a}(h, \tfrac{1}{2} + it)$ can understood through the
Phragm\'en-Lindel\"of Principle to satisfy the bound $\rho_\mathfrak{a}(h, \frac{1}{2} +
it) \ll (1 + \lvert t \rvert)^{\frac{1}{4} - \frac{k}{2} + \epsilon}$.


The first apparent poles in $s$ can be read from the Gamma functions, and it is clear that
there are no poles in $s$ for $\Re s > \frac{1}{2}$.


The other inner products, $\langle E_\mathfrak{a}^k(\cdot, u), V \rangle$, can be
understood through Zagier normalization~\cite{ZagierRankinSelberg}.
In particular, since $V = c y^{\frac{1-k}{2}} + O(y^{-N})$ for some constant $c$ and any
$N \geq 0$ as $y \to \infty$ (and more generally, at each cusp), Zagier normalization
allows us to identify
\begin{equation}\label{eq:continuous_stepI}
  \langle E_\infty^k(\cdot, u), V \rangle = \int_0^\infty \widetilde{V}_0(y) y^{u-1}
  \frac{dy}{y} = \frac{\Gamma(s + \frac{k+1}{2} - 1)}{(4\pi)^{u + \frac{k+1}{2} - 1}}
  \sum_{n \geq 1} \frac{r_{2k+1}(n)r_1(n)}{n^{u + \frac{k+1}{2} - 1}}
\end{equation}
for $\frac{1-k}{2} < \Re u < \frac{k+1}{2}$ and give meromorphic continuation to the plane.
Here, $\widetilde{V}_0$ is the $0$th Fourier coefficient of $\widetilde{V} = \theta^{2k+1}
\overline{\theta}$, which was first defined in \S\ref{sec:hyperboloid_introduction}.
Notice that the region of identification includes $\Re u = \frac{1}{2}$, as is necessary
for this application.
Similar expressions exist at the cusps $0$ and $\frac{1}{2}$.


\begin{remark}
  Zagier normalization also gives that the Dirichlet series at the right
  in~\eqref{eq:continuous_stepI} has a potential pole at $u = \frac{k+1}{2}$, which agrees
  with on-average estimates.
  The function $r_1(n)$ is essentially a square indicator function, so the Dirichlet
  series can be rewritten as
  \begin{equation}
    2\sum_{n \geq 1} \frac{r_{2k+1}(n^2)}{n^{2(u + \frac{k+1}{2} - 1)}},
  \end{equation}
  in which it is straightforward to use the Gaussian heuristic to confirm the pole from
  Zagier normalization.
\end{remark}


Using the functional equation of the Eisenstein series to give the functional equation of
the DIrichlet series in~\eqref{eq:continuous_stepI} and applying the Phragm\'en-Lindel\"of
Convexity Principle guarantees that
\begin{equation}
  \langle E_\mathfrak{a}^k(\cdot, \tfrac{1}{2} + it), V \rangle \ll (1 + \lvert t
  \rvert)^{2k - 1 + \epsilon}.
\end{equation}
Denote $\Re s = \sigma$, and suppose $\tfrac{1}{2} < \sigma < 1$.
Applying Stirling's Approximation and the bounds above, we can now estimate
\begin{align}
  &\frac{1}{2\pi i} \int_{(\frac{1}{2})} \langle P_h^k(\cdot, s),
E^k_{\mathfrak{a}}(\cdot, u) \rangle \langle E^k_{\mathfrak{a}}(\cdot, u), V \rangle du \\
  &\quad \ll \int_{-\infty}^\infty
  \frac{(1 + \lvert t \rvert)^{(\frac{1}{4} - \frac{k}{2} + \epsilon) + (2k - 1 +
  \epsilon)}}{(4\pi h)^{\sigma - 1}}
  \frac{(1 + \lvert s + t \rvert)^{\sigma - 1} (1 + \lvert s - t \rvert)^{\sigma - 1}}{(1
  + \lvert s \rvert)^{\sigma - \frac{1}{2} - \frac{k}{2}}}
  e^{-\frac{\pi}{2}(\lvert s - t \rvert + \lvert s + t \rvert - \lvert s \rvert)}dt.
\end{align}
When $\lvert t \rvert > \lvert s \rvert$, there is significant exponential decay in $t$,
effectively cutting off the integral to the interval $\lvert t \rvert \leq \lvert s
\rvert$.
Within this interval, the integral can be bounded by
\begin{equation}
  \int_{-\lvert s \rvert}^{\lvert s \rvert} (1 + \lvert t \rvert)^{\frac{3}{2}k -
  \frac{3}{4} + \epsilon} (1 + \lvert s \rvert)^{\frac{k}{2} - \frac{1}{2}}
  e^{-\frac{\pi}{2}\lvert s \rvert} dt \ll (1 + \lvert s \rvert)^{2k - \frac{1}{4} +
  \epsilon} e^{-\frac{\pi}{2}\lvert s \rvert}.
\end{equation}


\begin{proposition}\label{prop:hyperboloid_continuous_props}
  The continuous spectrum in line~\eqref{line:Ph_spectral_continuous} has meromorphic
  continuation tot he plane and is analytic for $\Re s > \frac{1}{2}$ and has apparent
  poles at $\Re s = \frac{1}{2}$.
  For $\frac{1}{2} < \Re s < 1$, the continuous spectrum satisfies the bound
  \begin{equation}
    \frac{1}{4\pi i} \sum_{\mathfrak{a}} \int_{(\frac{1}{2})} \langle P_h^k(\cdot, s),
    E^k_{\mathfrak{a}}(\cdot, u) \rangle \langle E^k_{\mathfrak{a}}(\cdot, u), V \rangle
    du \ll (1 + \lvert s \rvert)^{2k - \frac{1}{4} + \epsilon} e^{-\frac{\pi}{2}\lvert s
    \rvert}.
  \end{equation}
\end{proposition}



\subsection{Analytic Behavior of $D_h^k(s)$}


We are now ready to describe the analytic behavior of $D_h^k(s)$ for each $k \geq
\frac{1}{2}$, for $\Re s > \frac{1}{2}$.
Recall that $D_h^k(s)$ is defined in~\eqref{eq:hyp:Dh_def} and~\eqref{eq:hyp:Dh_def_half}
as
\begin{equation}
  D_h^k(s) = \frac{(2\pi)^{s + \frac{k-1}{2}}}{\Gamma(s + \frac{k-1}{2})} \langle
  P_h^{\frac{1}{2}}(\cdot, s), V \rangle.
\end{equation}
In Propositions~\ref{prop:nonspectral_analytic_props_large_dim} (for $k \geq 1$)
and~\ref{prop:nonspectral_analytic_props_dim_3} (for $k = \frac{1}{2}$), it was shown that
\begin{equation}\label{eq:hyperboloid_analytic_middlestep_I}
  D_h^k(s) = \sum_{m \in \mathbb{Z}} \frac{r_{2k+1}(m^2 + h)}{(2m^2 + h)^{s +
  \frac{k-1}{2}}} - \mathfrak{E}_h^k(s)
\end{equation}
and the analytic properties of $\mathfrak{E}_h^k(s)$ are described.
On the other hand, through the Spectral Expansion of $P_h^k$, we also have an expression
for $\langle P_h^k(\cdot, s), V \rangle$, given
in~\eqref{line:Ph_spectral_discrete}--\eqref{line:Ph_spectral_continuous}.
Multiplying by $(2\pi)^{s + \frac{k-1}{2}} \Gamma(s + \frac{k-1}{2})^{-1}$ and
rearranging~\eqref{eq:hyperboloid_analytic_middlestep_I}, we have
\begin{equation}\label{eq:hyperboloid_dirichlet_decomposition}
  \begin{split}
    &\sum_{m \in \mathbb{Z}} \frac{r_{2k+1}(m^2 + h)}{(2m^2 + h)^{s + \frac{k-1}{2}}} =
    \mathfrak{E}_h^k(s) \\
    &\qquad + \frac{(2\pi)^{s + \frac{k-1}{2}}}{\Gamma(s + \frac{k-1}{2})} \bigg( \sum_j
  \langle P_h^k, \mu_j \rangle \langle \mu_j, V \rangle + \sum_{\ell,j} \langle P_h^k,
\mu_{j,\ell} \rangle \langle \mu_{j, \ell}, V \rangle \bigg) \\
    &\qquad + \frac{(2\pi)^{s + \frac{k-1}{2}}}{\Gamma(s + \frac{k-1}{2})}
\sum_\mathfrak{a} \langle P_h^k, R_\mathfrak{a}\rangle \langle R_\mathfrak{a}, V \rangle
\\
    &\qquad + \frac{(2\pi)^{s + \frac{k-1}{2}}}{\Gamma(s + \frac{k-1}{2})} \frac{1}{4\pi
i} \sum_{\mathfrak{a}} \int_{(\frac{1}{2})} \langle P_h^k, E_\mathfrak{a}^k\rangle \langle
E_\mathfrak{a}^k, V\rangle,
  \end{split}
\end{equation}
where the lines are separated into the Eisenstein correction factors in $V$, the discrete
spectrum, the residual spectrum, and the continuous spectrum, respectively.
The analytic properties of the discrete, residual, and continuous spectra are described in
Propositions~\ref{prop:hyperboloid_discrete_props},~\ref{prop:hyperboloid_residual_props},
and~\ref{prop:hyperboloid_continuous_props}, respectively.
Assembling these propositions together, we have proved the following theorem.


\begin{theorem}\label{thm:hyperboloid:mero_summary}
  Let $h \geq 1$ be an integer and $k \geq \frac{1}{2}$ be either an integer or a
  half-integer.
  The Dirichlet series
  \begin{equation}
    \sum_{m \in \mathbb{Z}} \frac{r_{2k+1}(m^2 + h)}{(2m^2 + h)^{s + \frac{k-1}{2}}}
  \end{equation}
  has meromorphic continuation to the plane, and is analytic for $\Re s > \frac{1}{2}$
  except for
  \begin{itemize}
    \item simple poles at $s = \frac{k}{2} - 1 - m$ for $m \in \mathbb{Z}_{\geq 0}$,
      coming from the discrete spectrum,
    \item simple poles at $s = \frac{k+1}{2} - m$ for $m \in \mathbb{Z}_{\geq 0}$, coming
      from the Eisenstein correction factors $\mathfrak{E}_h^k(s)$, and
    \item a double pole at $s = \frac{3}{4}$ when $k = \frac{1}{2}$ and $h$ is a square,
      also coming from $\mathfrak{E}_h^k(s)$.
  \end{itemize}
\end{theorem}



\section{Integral Analysis}\label{sec:hyp:integral_analysis}

We are now ready to perform the main integral analysis on $\langle P_h^k(\cdot, s),
V\rangle$.
In this section, we handle the $k > \frac{1}{2}$ case.
We will examine
\begin{equation}
  \frac{1}{2\pi i} \int_{(\sigma)} \sum_{n \in \mathbb{Z}} \frac{r_{2k+1}(n^2 + h)}{(2n^2 +
  h)^{s + \frac{k-1}{2}}} X^{s+\frac{k-1}{2}} V_Y(s) ds,
\end{equation}
which is closely related to studying
\begin{equation}
  \frac{1}{2\pi i} \int_{(\sigma)} \langle P_h^k(\cdot, s), V \rangle \frac{(2\pi)^{s +
  \frac{k-1}{2}}}{\Gamma(s + \frac{k-1}{2})} X^{s+\frac{k-1}{2}} V_Y(s) ds.
\end{equation}
We will use three integral kernels $V_Y(s)$ described in Chapter~\ref{c:background}.
We denote the Mellin transform of $V_Y(s)$ by $v_Y(x)$ when appropriate.


From the decomposition in~\eqref{eq:hyperboloid_dirichlet_decomposition}, it will suffice
to consider the integral transforms applied to $\mathfrak{E}_h^k$, the discrete spectrum,
the residual spectrum, and the continuous spectrum separately.
In each integral, we will shift the line of integration to $\frac{1}{2} + \epsilon$ for a
small $\epsilon > 0$ and analyze the poles and residues.



\subsection{Integral Analysis of $\mathfrak{E}_h^k$}


We first study
\begin{equation}
  \frac{1}{2\pi i} \int_{(\sigma)} \mathfrak{E}_h^k(s) X^{s + \frac{k-1}{2}} V_Y(s) ds.
\end{equation}
From Theorem~\ref{thm:hyperboloid:mero_summary} and
Propositions~\ref{prop:nonspectral_analytic_props_large_dim}
and~\ref{prop:nonspectral_analytic_props_dim_3}, we recognize that $\mathfrak{E}_h^k$ has
poles at $s = \frac{k+1}{2} - m$ for $m \in \mathbb{Z}_{\geq 0}$.
All of these poles are simple, except when $k = \frac{1}{2}$ and $h$ is a square, in which
case the leading pole at $s = \frac{k+1}{2} = \frac{3}{4}$ is a double pole.
As $\mathfrak{E}_h^k$ is of moderate growth for $\Re s > \frac{1}{2}$ and each kernel
$V_Y(s)$ is of rapid decay, we may shift the line of integration to $\frac{1}{2} +
\epsilon$ by Cauchy's Theorem we have
\begin{equation}\label{eq:hyp:integral_MT}
  \begin{split}
    &\frac{1}{2\pi i} \int_{(\sigma)} \mathfrak{E}_h^k(s) X^{s + \frac{k-1}{2}} V_Y(s) ds \\
    &\quad = \frac{1}{2\pi i} \int_{(\frac{1}{2} + \epsilon)} \mathfrak{E}_h^k(s) X^{s +
    \frac{k-1}{2}} V_Y(s) ds + \sum_{0 \leq m < \frac{k}{2}} R_{k-m,h}^k X^{k - m}
    V_Y(\tfrac{k+1}{2} - m) \\
    &\qquad + \delta_{[k=\frac{1}{2}]} \delta_{[h = a^2]} \bigg( R'_h X^{\frac{1}{2}} \log X
  V_Y(\tfrac{3}{4}) + R'_h X^{\frac{1}{2}} V'_Y(\tfrac{3}{4}) \bigg)
  \end{split}
\end{equation}
where
\begin{equation}
  R_{k-m,h}^k = \Res_{s = \frac{k+1}{2} - m} \mathfrak{E}_h^k(s)
\end{equation}
and where $R'_h$ is the coefficient of $(s - \tfrac{3}{4})^{-2}$ in the Laurent expansion
of $\mathfrak{E}_h^{\frac{1}{2}}(s)$.
The Kronecker $\delta$ symbol is used here to mean
\begin{equation}
  \delta_{[k = \frac{1}{2}]} \delta_{[h = a^2]} = \begin{cases}
    1 & \text{if } k = \frac{1}{2} \; \text{and } h \; \text{is a square}, \\
    0 & \text{otherwise}.
  \end{cases}
\end{equation}



To estimate the shifted integral, recall from
Propositions~\ref{prop:nonspectral_analytic_props_large_dim}
and~\ref{prop:nonspectral_analytic_props_dim_3} that for $\Re s > \frac{1}{2}$, we know
the bound $\mathfrak{E}_h^k(s) \ll (1 + \lvert s \rvert)^{\frac{1}{2}}$.
Therefore as long as $V_Y(s) \ll (1 + \lvert s \rvert)^{-\frac{3}{2} - \epsilon}$, the
integral converges absolutely.
With respect to the three integral transforms, this means that
\begin{align}
  \frac{1}{2\pi i} &\int_{(\frac{1}{2} + \epsilon)} \mathfrak{E}_h^k(s) X^{s +
  \frac{k-1}{2}} \Gamma(s + \tfrac{k-1}{2}) ds \ll X^{\frac{k}{2} + \epsilon}
  \label{eq:hyperboloid:nonspectral_gamma_integral_bound} \\
  \frac{1}{2\pi i} &\int_{(\frac{1}{2} + \epsilon)} \mathfrak{E}_h^k(s) X^{s +
  \frac{k-1}{2}} \frac{e^{\pi s^2 / y^2}}{y} ds  \ll X^{\frac{k}{2} + \epsilon}
  Y^{\frac{1}{2}} \label{eq:hyperboloid:nonspectral_concentrating_integral_bound}\\
  \frac{1}{2\pi i} &\int_{(\frac{1}{2} + \epsilon)} \mathfrak{E}_h^k(s) X^{s +
  \frac{k-1}{2}} \Phi_Y(s) ds  \ll X^{\frac{k}{2} + \epsilon} Y^{\frac{1}{2}+\epsilon}.
  \label{eq:hyperboloid:nonspectral_compact_integral_bound}
\end{align}



\subsection{Integral analysis of the discrete spectrum}


We now study the integral of the discrete spectrum.
To condense notation, we introduce the notation
\begin{equation}
  \discrete_h^k(s) := \frac{(2\pi)^{s + \frac{k-1}{2}}}{\Gamma(s + \frac{k-1}{2})} \bigg(
  \sum_j \langle P_h^k, \mu_j \rangle \langle \mu_j, V \rangle + \sum_{\ell,j} \langle
P_h^k, \mu_{j,\ell} \rangle \langle \mu_{j, \ell}, V \rangle \bigg).
\end{equation}
Then the integral of the discrete spectrum can be written as
\begin{equation}\label{eq:integral_analysis_discrete_I}
  \frac{1}{2\pi i} \int_{(\sigma)} \discrete_h^k(s) X^{s + \frac{k-1}{2}} V_Y(s) ds.
\end{equation}
From Theorem~\ref{thm:hyperboloid:mero_summary} we recognize that the integrand has simple
poles at $s = \frac{k}{2} - m$ for $m \in \mathbb{Z}_{\geq 0}$, coming from the finitely
many terms of the ``old'' discrete spectrum.
Therefore, shifting the line of integration to $\frac{1}{2} + \epsilon$ and applying
Cauchy's Theorem shows that~\eqref{eq:integral_analysis_discrete_I} is equal to
\begin{equation}
  \sum_{0 \leq m \leq \frac{k-1}{2}} R^k_{k - \frac{1}{2} - m, h} X^{k - \frac{1}{2} - m}
  V(\tfrac{k+1}{2} - m - \tfrac{1}{2}) + \frac{1}{2\pi i} \int_{(\frac{1}{2} + \epsilon)}
  \discrete_h^k(s) X^{s + \frac{k-1}{2}} V_Y(s) ds,
\end{equation}
where $R^k_{k - \frac{1}{2} - m}$ are the collected residues of the old discrete spectrum
at $s = \frac{k}{2} - m$.



To estimate the shifted integral, note that from applying Stirling's Approximation to
$\Gamma(s+\frac{k-1}{2})^{-1}$ and using the approximations from
Proposition~\ref{prop:hyperboloid_discrete_props} that $\discrete_h^k(s) \ll (1 + \lvert s
\rvert)^{k + \frac{7}{2} +\epsilon}$.
Therefore as long as $V_Y(s) \ll (1 + \lvert s \rvert)^{-k - \frac{9}{2} - 2\epsilon}$,
the integral converges absolutely.
The three integral transforms then give
\begin{align}
  &\frac{1}{2\pi i} \int_{(\frac{1}{2} + \epsilon)} \discrete_h^k(s) X^{s + \frac{k-1}{2}}
  \Gamma(s + \frac{k-1}{2})  ds \ll X^{\frac{k}{2} + \epsilon}
  \label{eq:hyperboloid:discrete_gamma_integral_bound}\\
  &\frac{1}{2\pi i} \int_{(\frac{1}{2} + \epsilon)} \discrete_h^k(s) X^{s + \frac{k-1}{2}}
  \frac{e^\pi s^2/Y^2}{Y} ds \ll X^{\frac{k}{2} + \epsilon} Y^{k + \frac{7}{2} + \epsilon}
  \label{eq:hyperboloid:discrete_concentrating_integral_bound} \\
  &\frac{1}{2\pi i} \int_{(\frac{1}{2} + \epsilon)} \discrete_h^k(s) X^{s + \frac{k-1}{2}}
  \Phi_Y(s) ds \ll X^{\frac{k}{2}+ \epsilon} Y^{k + \frac{7}{2} + 2\epsilon}.
  \label{eq:hyperboloid:discrete_compact_integral_bound}
\end{align}


\subsection{Integral analysis of the residual spectrum}


We now study the integral of the residual spectrum,
\begin{equation}\label{eq:hyperboloid:integral_analysis_residual_I}
  \frac{1}{2\pi i} \int_{(\sigma)} \frac{(2\pi)^{s + \frac{k-1}{2}}}{\Gamma(s +
  \frac{k-1}{2})} \sum_\mathfrak{a} \langle P_h^k, R_\mathfrak{a}\rangle \langle
R_\mathfrak{a}, V \rangle X^{s + \frac{k-1}{2}} V_Y(s) ds.
\end{equation}
From Theorem~\ref{thm:hyperboloid:mero_summary},
Proposition~\ref{prop:hyperboloid_residual_props}, and the analytic properties of the
residual spectrum as described in~\eqref{eq:hyperboloid:residual_analytic_descriptor}, the
residual spectrum is analytic for $\Re s > \frac{1}{2} + \epsilon$ and is bounded by $O((1
+ \lvert s \rvert)^{\frac{1}{2}})$.
By Cauchy's Theorem the integral~\eqref{eq:hyperboloid:integral_analysis_residual_I} is
equal to the shifted integral
\begin{equation}
  \frac{1}{2\pi i} \int_{(\frac{1}{2} + \epsilon)} \frac{(2\pi)^{s +
  \frac{k-1}{2}}}{\Gamma(s + \frac{k-1}{2})} \sum_\mathfrak{a} \langle P_h^k,
  R_\mathfrak{a}\rangle \langle R_\mathfrak{a}, V \rangle X^{s + \frac{k-1}{2}} V_Y(s) ds.
\end{equation}
As long as $V_Y(s) \ll (1 + \lvert s \rvert)^{-\frac{3}{2} + \epsilon}$, this converges
absolutely.
The three integral transforms then satisfy the bounds
\begin{align}
  &\frac{1}{2\pi i} \int_{(\frac{1}{2} + \epsilon)} \frac{(2\pi)^{s +
  \frac{k-1}{2}}}{\Gamma(s + \frac{k-1}{2})} \sum_\mathfrak{a} \langle P_h^k,
    R_\mathfrak{a}\rangle \langle R_\mathfrak{a}, V \rangle X^{s + \frac{k-1}{2}} \Gamma(s
    + \tfrac{k-1}{2}) ds \ll X^{\frac{k}{2} + \epsilon}
  \label{eq:hyperboloid:residual_gamma_integral_bound} \\
  &\frac{1}{2\pi i} \int_{(\frac{1}{2} + \epsilon)} \frac{(2\pi)^{s +
  \frac{k-1}{2}}}{\Gamma(s + \frac{k-1}{2})} \sum_\mathfrak{a} \langle P_h^k,
  R_\mathfrak{a}\rangle \langle R_\mathfrak{a}, V \rangle X^{s + \frac{k-1}{2}}
  \frac{e^{\pi s^2/Y^2}}{Y} ds \ll X^{\frac{k}{2} + \epsilon} Y^{\frac{1}{2} + \epsilon}
  \label{eq:hyperboloid:residual_concentrating_integral_bound} \\
  &\frac{1}{2\pi i} \int_{(\frac{1}{2} + \epsilon)} \frac{(2\pi)^{s +
  \frac{k-1}{2}}}{\Gamma(s + \frac{k-1}{2})} \sum_\mathfrak{a} \langle P_h^k,
  R_\mathfrak{a}\rangle \langle R_\mathfrak{a}, V \rangle X^{s + \frac{k-1}{2}} \Phi_Y(s)
  ds \ll X^{\frac{k}{2} + \epsilon} Y^{\frac{1}{2} + 2\epsilon}.
  \label{eq:hyperboloid:residual_compact_integral_bound}
\end{align}



\subsection{Integral analysis of the continuous spectrum}


We now consider the last integral, the integral of the continuous spectrum:
\begin{equation}\label{eq:hyperboloid:integral_analysis_continuous_I}
  \frac{1}{2\pi i} \int_{(\sigma)} \frac{(2\pi)^{s + \frac{k-1}{2}}}{\Gamma(s +
  \frac{k-1}{2})} \frac{1}{4\pi i} \sum_{\mathfrak{a}} \int_{(\frac{1}{2})} \langle P_h^k,
  E_\mathfrak{a}^k\rangle \langle E_\mathfrak{a}^k, V\rangle X^{s+\frac{k-1}{2}} V_Y(s)
  du \, ds.
\end{equation}
(Recall that we use $u$ to denote the variable within the Eisenstein series
$E_\mathfrak{a}^k(z,u)$, though we omit this from the notation).
By Theorem~\ref{thm:hyperboloid:mero_summary}, we know that the continuous spectrum is
analytic for $\Re s > \frac{1}{2}$.
By Cauchy's Theorem, the integral~\eqref{eq:hyperboloid:integral_analysis_continuous_I}
is equal to the shifted integral
\begin{equation}
  \frac{1}{2\pi i} \int_{(\frac{1}{2} + \epsilon)} \frac{(2\pi)^{s +
  \frac{k-1}{2}}}{\Gamma(s + \frac{k-1}{2})} \frac{1}{4\pi i} \sum_{\mathfrak{a}}
  \int_{(\frac{1}{2})} \langle P_h^k, E_\mathfrak{a}^k\rangle \langle E_\mathfrak{a}^k,
  V\rangle X^{s+\frac{k-1}{2}} V_Y(s) du \, ds.
\end{equation}
From Stirling's Approximation applied to $\Gamma(s + \frac{k-1}{2})^{-1}$ and the bounds
from Proposition~\ref{prop:hyperboloid_continuous_props}, as long as $V_Y(s) \ll (1 +
\lvert s \rvert)^{-\frac{3}{2}k - \frac{7}{4} - 2\epsilon}$, the shifted integral
converges absolutely.
The three integral transforms thus satisfy the bounds
\begin{align}
  \begin{split}
    &\frac{1}{2\pi i} \int_{(\frac{1}{2} + \epsilon)} \frac{(2\pi)^{s +
    \frac{k-1}{2}}}{\Gamma(s + \frac{k-1}{2})} \frac{1}{4\pi i} \\
    &\quad \times \sum_{\mathfrak{a}} \int_{(\frac{1}{2})} \langle P_h^k,
    E_\mathfrak{a}^k\rangle \langle
    E_\mathfrak{a}^k, V\rangle X^{s+\frac{k-1}{2}} \Gamma(s + \tfrac{k-1}{2}) du \, ds \ll
    X^{\frac{k}{2} + \epsilon} \label{eq:hyperboloid:continuous_gamma_integral_bound}
  \end{split}
  \\
  \begin{split}
    &\frac{1}{2\pi i} \int_{(\frac{1}{2} + \epsilon)} \frac{(2\pi)^{s +
    \frac{k-1}{2}}}{\Gamma(s + \frac{k-1}{2})} \frac{1}{4\pi i} \\
    &\quad \times \sum_{\mathfrak{a}}
    \int_{(\frac{1}{2})} \langle P_h^k, E_\mathfrak{a}^k\rangle \langle E_\mathfrak{a}^k,
    V\rangle X^{s+\frac{k-1}{2}} \frac{e^{\pi s^2 / Y^2}}{Y} du\, ds \ll X^{\frac{k}{2} +
    \epsilon} Y^{\frac{3}{2}k + \frac{3}{4} + \epsilon}
    \label{eq:hyperboloid:continuous_concentrating_integral_bound}
  \end{split}
  \\
  \begin{split}
    &\frac{1}{2\pi i} \int_{(\frac{1}{2} + \epsilon)} \frac{(2\pi)^{s +
    \frac{k-1}{2}}}{\Gamma(s + \frac{k-1}{2})} \frac{1}{4\pi i} \\
    &\quad \times \sum_{\mathfrak{a}}
    \int_{(\frac{1}{2})} \langle P_h^k, E_\mathfrak{a}^k\rangle \langle E_\mathfrak{a}^k,
    V\rangle X^{s+\frac{k-1}{2}} \Phi_Y(s) du \, ds \ll X^{\frac{k}{2} + \epsilon}
    Y^{\frac{3}{2}k + \frac{3}{4} + 2\epsilon}.
    \label{eq:hyperboloid:continuous_compact_integral_bound}
  \end{split}
\end{align}



\section{Proof of Main Theorems}\label{sec:hyp:proof_main_theorems}


Using the integral analysis from the previous section, we now prove the main theorems of
this chapter.


\subsection{Smoothed Main Theorem}


Consider the integral transform
\begin{equation}
  \frac{1}{2\pi i} \int_{(\sigma)} \sum_{m \in \mathbb{Z}} \frac{r_{2k+1}{(m^s +
  h)}}{(2m^2 + h)^{s + \frac{k-1}{2}}} X^{s + \frac{k-1}{2}} \Gamma(s + \tfrac{k-1}{2})
  ds.
\end{equation}
On the one hand, by the standard properties of the transform (in
\S\ref{sec:cutoff_integrals}), this is exactly the sum
\begin{equation}
  \sum_{m \in \mathbb{Z}} r_{2k+1}(m^2 + h) e^{-(2m^2 + h)/X}.
\end{equation}
On the other hand, by~\eqref{eq:hyperboloid_dirichlet_decomposition} and the analysis in
the previous section (and in particular the bounds from
lines~\eqref{eq:hyperboloid:nonspectral_gamma_integral_bound},
\eqref{eq:hyperboloid:discrete_gamma_integral_bound},
\eqref{eq:hyperboloid:residual_gamma_integral_bound},
and~\eqref{eq:hyperboloid:continuous_gamma_integral_bound}, as well as the residual
expression in~\eqref{eq:hyp:integral_MT}), this transform is equal to
\begin{align}
  &\sum_{0 \leq m < \frac{k}{2}} R_{k-m, h}^k \Gamma(k-m) X^{k-m} + \delta_{[k =
\frac{1}{2}]} \delta_{[h = a^2]} \bigg( R'_h X^{\frac{1}{2}} \log X \Gamma(\tfrac{1}{2}) +
R'_h \Gamma'(\tfrac{1}{2}) X^{\frac{1}{2}} \bigg) \\
  &\qquad + \sum_{0 \leq m < \frac{k-1}{2}} R_{k - \frac{1}{2} - m,h}^k \Gamma(k -
  \tfrac{1}{2} - m) X^{k - \frac{1}{2} - m} + O(X^{\frac{k}{2} + \epsilon}).
\end{align}
This proves the following theorem.
\begin{theorem}\label{thm:hyperboloid:smooth_full}
  Let $k \geq \frac{1}{2}$ be a full or half-integer.
  Then
  \begin{align}
    &\sum_{m \in \mathbb{Z}} r_{2k+1}(m^2 + h) e^{-(2m^2 + h)/X} =\\
    &\quad =\delta_{[k = \frac{1}{2}]} \delta_{[h = a^2]} \bigg( R'_h X^{\frac{1}{2}} \log
    X \Gamma(\tfrac{1}{2}) + R'_h \Gamma'(\tfrac{1}{2}) X^{\frac{1}{2}} \bigg) + \sum_{0
    \leq m < \frac{k}{2}} R_{k-m, h}^k \Gamma(k-m) X^{k-m} \\
    &\qquad + \sum_{0 \leq m < \frac{k-1}{2}} R_{k - \frac{1}{2} - m,h}^k \Gamma(k -
    \tfrac{1}{2} - m) X^{k - \frac{1}{2} - m} + O(X^{\frac{k}{2} + \epsilon}).
  \end{align}
  Here, $\delta_{[\text{condition}]}$ is a Kronecker $\delta$ and evaluates to $1$ if the
  condition is true and $0$ otherwise, and the constants $R^k_{\ell, h}$ and $R'_h$ are
  residues as defined in \S\ref{sec:hyp:integral_analysis}.
\end{theorem}



\subsection{Main Theorem in Short-Intervals}


Consider the integral transform
\begin{equation}
  \frac{1}{2\pi i} \int_{(\sigma)} \sum_{m \in \mathbb{Z}} \frac{r_{2k+1}{(m^s +
  h)}}{(2m^2 + h)^{s + \frac{k-1}{2}}} X^{s + \frac{k-1}{2}} \frac{e^{\pi s^2/Y^2}}{Y} ds.
\end{equation}
On the one hand, by the properties of this integral transform as described in
\S\ref{sec:cutoff_integrals}, this is exactly the sum
\begin{equation}
  \sum_{m \in \mathbb{Z}} r_{2k+1}(m^2 + h) \exp\bigg( -\frac{Y^2
  \log^2(X/(2m^2+h))}{4\pi} \bigg).
\end{equation}
When $2m^2 + h \in [X - X/Y, X + X/Y]$, the exponential damping term is almost constant.
But for $m$ with $\lvert 2m^2 + h - X \rvert > X/Y^{1-\epsilon}$, the exponential term
contributes significant exponential decay.
Further, by the positivity of $r_{2k+1}$, we have that
\begin{equation}
  \sum_{\lvert 2m^2 + h - X \rvert < \frac{X}{Y}} r_{2k+1}(m^2 + h) \ll \frac{1}{2\pi i}
  \int_{(\sigma)} \sum_{m \in \mathbb{Z}} \frac{r_{2k+1}{(m^s + h)}}{(2m^2 + h)^{s +
  \frac{k-1}{2}}} X^{s + \frac{k-1}{2}} \frac{e^{\pi s^2/Y^2}}{Y} ds.
\end{equation}


\begin{remark}
  Roughly speaking, this should be interpreted to mean that this integral transform
  concentrates the mass of the integral on those $m$ such that $2m^2 + h$ is within a
  short interval around $X$.
\end{remark}


And on the other hand, by~\eqref{eq:hyperboloid_dirichlet_decomposition} and the analysis
in the previous section (in particular the bounds from
lines~\eqref{eq:hyperboloid:nonspectral_concentrating_integral_bound},
\eqref{eq:hyperboloid:discrete_concentrating_integral_bound},
\eqref{eq:hyperboloid:residual_concentrating_integral_bound},
and~\eqref{eq:hyperboloid:continuous_concentrating_integral_bound}, as well as the
residual expression in~\eqref{eq:hyp:integral_MT}), this transform is
equal to
\begin{align}
  &\delta_{[k = \frac{1}{2}]} \delta_{[h = a^2]} \bigg(R'_h X^{\frac{1}{2}}\log X +R'_h
X^{\frac{1}{2}} \frac{\pi}{Y} \bigg)\frac{\exp(\frac{\pi}{4Y})}{Y} + R_k^k X^k
\frac{\exp(\frac{\pi k^2}{Y})}{Y} \\
  &\quad + O(\frac{X^{k-\frac{1}{2}}}{Y}) + O(X^{\frac{k}{2}+\epsilon} Y^{k + \frac{7}{2}
  + \epsilon}) + O(X^{\frac{k}{2} + \epsilon} Y^{\frac{3}{2}k + \frac{3}{4} + \epsilon}).
\end{align}
In this expression, we only kept the leading poles.
As we are only seeking to create an upper bound, we simplify the above expression into the
bound
\begin{equation}\label{eq:hyp:short_interval_step_I}
  O(\frac{X^{k+\epsilon}}{Y}) + O(X^{\frac{k}{2}+\epsilon} Y^{k + \frac{7}{2} + \epsilon})
  + O(X^{\frac{k}{2} + \epsilon} Y^{\frac{3}{2}k + \frac{3}{4} + \epsilon}).
\end{equation}


We choose $Y$ to balance the expressions in~\eqref{eq:hyp:short_interval_step_I}.
When $k \leq \frac{11}{2}$, the two terms
\begin{equation}
  O(\frac{X^{k+\epsilon}}{Y}) + O(X^{\frac{k}{2} + \epsilon} Y^{k + \frac{7}{2} +
  \epsilon})
\end{equation}
dominate, and are balanced when $Y = X^{1/(2 + \frac{9}{k} + \frac{2\epsilon}{k})}$.
This gives the overall bound
\begin{equation}\label{eq:hyp:short_interval_bound_smallk}
  O(X^{k + \epsilon - \lambda(k)}), \qquad \text{where } \lambda(k) = \frac{1}{2 +
  \frac{9}{k}}.
\end{equation}
When $k \geq \frac{11}{2}$, the two terms
\begin{equation}
  O(\frac{X^{k+\epsilon}}{Y}) + O(X^{\frac{k}{2} + \epsilon} Y^{\frac{3}{2} k +
  \frac{3}{4} + \epsilon})
\end{equation}
dominate, and are balanced when $Y = X^{1/(3 + \frac{7}{2k} + \frac{2\epsilon}{k})}$.
This gives the overall bound
\begin{equation}\label{eq:hyp:short_interval_bound_largek}
  O(X^{k + \epsilon - \lambda(k)}), \qquad \text{where } \lambda(k) = \frac{1}{3 +
  \frac{7}{2k}}.
\end{equation}


We have now shown that
\begin{equation}
  \sum_{\lvert 2m^2 + h - X \rvert < X^{1 + \epsilon - \lambda(k)}} r_{2k+1}(m^2 + h) \ll
  X^{k + \epsilon - \lambda(k)},
\end{equation}
where $\lambda(k)$ is defined by~\eqref{eq:hyp:short_interval_bound_smallk} or
by~\eqref{eq:hyp:short_interval_bound_largek} depending on the size of $k$.
This is the content of the second main theorem in this chapter.
\begin{theorem}\label{thm:hyp:concentrating_theorem_full}
  Let $k \geq \frac{1}{2}$ be a full or half-integer.
  Then
  \begin{equation}
    \sum_{\lvert 2m^2 + h - X \rvert < X^{1 + \epsilon - \lambda(k)}} r_{2k+1}(m^2 + h)
    \ll X^{k + \epsilon - \lambda(k)},
  \end{equation}
  where $\lambda(k)$ is defined as
  \begin{equation}
    \lambda(k) =
    \begin{cases}
      \frac{1}{2 + \frac{9}{k}} & \text{if } k \leq \frac{11}{2} \\
      \frac{1}{3 + \frac{7}{2k}} & \text{if } k \geq \frac{11}{2}.
    \end{cases}
  \end{equation}
\end{theorem}



\subsection{Sharp Main Theorem}


Finally, consider the integral transform
\begin{equation}
  \frac{1}{2\pi i} \int_{(\sigma)} \sum_{m \in \mathbb{Z}} \frac{r_{2k+1}{(m^s +
  h)}}{(2m^2 + h)^{s + \frac{k-1}{2}}} X^{s + \frac{k-1}{2}} \Phi_Y(s) ds.
\end{equation}
On the one hand, by the description of this integral transform in
\S\ref{sec:cutoff_integrals}, this is exactly the sum
\begin{equation}\label{eq:hyp:sharp_main_estimate_I}
  \sum_{\lvert 2m^2 + h \rvert \leq X} r_{2k+1}(m^2 + h) + \sum_{X \leq \lvert 2m^2+h
  \rvert \leq X+\frac{X}{Y}} r_{2k+1}(m^2+h) \phi_Y(\tfrac{2m^2+h}{X}).
\end{equation}
As $r_{2k+1}$ is always positive, we can bound the second term above by
\begin{equation}\label{eq:hyp:sharp_main_estimate_II}
  \sum_{X \leq \lvert 2m^2+h \rvert \leq X+\frac{X}{Y}} r_{2k+1}(m^2+h)
  \phi_Y(\tfrac{2m^2+h}{X}) \ll \sum_{\lvert 2m^2 + h - X \rvert \leq \frac{X}{Y}}
  r_{2k+1}(m^2 + h).
\end{equation}
Notice that this is a short-interval type estimate, exactly as considered in
Theorem~\ref{thm:hyp:concentrating_theorem_full}.


On the other hand, by~\eqref{eq:hyperboloid_dirichlet_decomposition} and the analysis in
the previous section (and in particular the bounds from
lines~\eqref{eq:hyperboloid:nonspectral_compact_integral_bound},
\eqref{eq:hyperboloid:discrete_compact_integral_bound},
\eqref{eq:hyperboloid:residual_compact_integral_bound},
and~\eqref{eq:hyperboloid:continuous_compact_integral_bound}, as well as the residual
expression in~\eqref{eq:hyp:integral_MT}), this integral
transform is equal to
\begin{align}
  &\sum_{0 \leq m < \frac{k}{2}} R_{k-m, h}^k X^{k-m} \big( \tfrac{1}{k-m} +
O(\tfrac{1}{Y}) \big) \\
  &\quad + \delta_{[k = \frac{1}{2}]} \delta_{[h = a^2]} \bigg( R'_h X^{\frac{1}{2}} \log
X \big(2 + O(\tfrac{1}{Y}) \big) + R'_h X^{\frac{1}{2}} \big( -4 + O(\tfrac{1}{Y}) \big)
\bigg) \\
  &\quad + \sum_{0 \leq m < \frac{k-1}{2}} R_{k - \frac{1}{2} - m,h}^k X^{k - \frac{1}{2}
- m} \big( \tfrac{1}{k - \frac{1}{2} - m} + O(\tfrac{1}{Y})\big) \\
  &\quad + O(X^{\frac{k}{2} + \epsilon} Y^{\frac{1}{2} + \epsilon}) + O(X^{\frac{k}{2} +
  \epsilon} Y^{k + \frac{7}{2} + 2\epsilon}) + O(X^{\frac{k}{2} + \epsilon}
Y^{\frac{3}{2}k + \frac{3}{4} + 2\epsilon}).
\end{align}
We note that we have used that $\Phi_Y(s) = \frac{1}{s} + O(\frac{1}{Y})$ and
$\Phi_Y'(s) = -\frac{1}{s^2} + O(\frac{1}{Y})$ from \S\ref{sec:cutoff_integrals}
to simplify the residual terms involving the weight function $\Phi_Y$, giving
the various big Oh terms in this expression.




Keeping only the leading terms of growth, we rewrite this as
\begin{align}
  &\delta_{[k = \frac{1}{2}]} \delta_{[h = a^2]} \bigg( 2 R'_h X^{\frac{1}{2}} \log X -4
R'_h X^{\frac{1}{2}} \bigg) + \tfrac{1}{k} R_{k, h}^k X^{k} +  O\big(\frac{X^{\frac{1}{2}}
\log X}{Y}\big) \\
  &\quad + \big(\frac{X^{k}}{Y}\big) + O(X^{\frac{k-1}{2}}) + O(X^{\frac{k}{2} + \epsilon}
  Y^{k + \frac{7}{2} + 2\epsilon}) + O(X^{\frac{k}{2} + \epsilon} Y^{\frac{3}{2}k +
\frac{3}{4} + 2\epsilon}).
\end{align}
The collected error terms can be written as
\begin{equation}
  O(X^{\frac{k-1}{2}}) + \big(\frac{X^{k+\epsilon}}{Y}\big) + O(X^{\frac{k}{2} + \epsilon}
  Y^{k + \frac{7}{2} + 2\epsilon}) + O(X^{\frac{k}{2} + \epsilon} Y^{\frac{3}{2}k +
\frac{3}{4} + 2\epsilon}).
\end{equation}
Notice that this is the exact same expression as in~\eqref{eq:hyp:short_interval_step_I},
and so the choice of $Y$ that optimizes the error bound is the same!
That is, we choose $Y = X^{\lambda(k)}$ where $\lambda(k)$ is defined as in
Theorem~\ref{thm:hyp:concentrating_theorem_full}.


By combining this optimal error term and choice of $Y = X^{\lambda(k)}$ with the
expression~\eqref{eq:hyp:sharp_main_estimate_I} and the
bound~\eqref{eq:hyp:sharp_main_estimate_II}, we have shown that
\begin{align}
  &\frac{1}{2\pi i} \int_{(\sigma)} \sum_{m \in \mathbb{Z}} \frac{r_{2k+1}{(m^s +
h)}}{(2m^2 + h)^{s + \frac{k-1}{2}}} X^{s + \frac{k-1}{2}} \Phi_Y(s) ds \\
  &\quad =\sum_{\lvert 2m^2 + h \rvert \leq X} r_{2k+1}(m^2 + h) +O \bigg( \sum_{\lvert
  2m^2+h - X \rvert \leq X^{1 + \epsilon - \lambda(k)}} r_{2k+1}(m^2+h)\bigg)  \\
  &\quad = \delta_{[k = \frac{1}{2}]} \delta_{[h = a^2]} \bigg( 2 R'_h X^{\frac{1}{2}}
\log X -4 R'_h X^{\frac{1}{2}} \bigg) + \tfrac{1}{k} R_{k, h}^k X^{k} + O(X^{k + \epsilon
- \lambda(k)}).
\end{align}
Rearranging, this shows that
\begin{align}
  &\sum_{\lvert 2m^2 + h \rvert \leq X} r_{2k+1}(m^2 + h) \\
  &\quad = \delta_{[k = \frac{1}{2}]} \delta_{[h = a^2]} \bigg( 2 R'_h X^{\frac{1}{2}}
\log X -4 R'_h X^{\frac{1}{2}} \bigg) + \tfrac{1}{k} R_{k, h}^k X^{k} + O(X^{k + \epsilon
- \lambda(k)}) \\
  &\quad + O \bigg( \sum_{\lvert 2m^2+h - X \rvert \leq X^{1 + \epsilon - \lambda(k)}}
  r_{2k+1}(m^2+h)\bigg).
\end{align}
By Theorem~\ref{thm:hyp:concentrating_theorem_full}, the last big Oh term is bounded by
$O(X^{k + \epsilon - \lambda(k)})$.
This concludes the proof of the following theorem.


\begin{theorem}\label{thm:hyp:sharp_theorem_full}
  Let $k \geq \frac{1}{2}$ be a full or half-integer.
  Then
  \begin{align}
    &\sum_{\lvert 2m^2 + h - X \rvert \leq X} r_{2k+1}(m^2 + h) \\
    &\quad= \delta_{[k = \frac{1}{2}]} \delta_{[h = a^2]} \bigg( 2 R'_h X^{\frac{1}{2}}
  \log X -4 R'_h X^{\frac{1}{2}} \bigg) + \tfrac{1}{k} R_{k, h}^k X^{k} + O(X^{k +
\epsilon - \lambda(k)}),
  \end{align}
  where $\lambda(k)$ is defined as
  \begin{equation}
    \lambda(k) =
    \begin{cases}
      \frac{1}{2 + \frac{9}{k}} & \text{if } k \leq \frac{11}{2} \\
      \frac{1}{3 + \frac{7}{2k}} & \text{if } k \geq \frac{11}{2}.
    \end{cases}
  \end{equation}
  In particular, if $k = \frac{1}{2}$ then
  \begin{align}
    &\sum_{\lvert 2m^2 + h - X \rvert \leq X} r_{2k+1}(m^2 + h) \\
    &\quad = \delta_{[h = a^2]} \bigg( 2 R'_h X^{\frac{1}{2}} \log X -4 R'_h
  X^{\frac{1}{2}} \bigg) + 2R_{1/2, h}^{1/2} X^{\frac{1}{2}} + O(X^{\frac{1}{2} -
\frac{1}{20} + \epsilon}),
  \end{align}
  and for $k \geq 1$
  \begin{equation}
    \sum_{\lvert 2m^2 + h - X \rvert \leq X} r_{2k+1}(m^2 + h)  = \tfrac{1}{k} R_{k, h}^k
    X^{k} + O(X^{k + \epsilon - \lambda(k)}).
  \end{equation}
\end{theorem}


\begin{remark}
  It is interesting that the shape of the main term is different in the dimension $3$ case
  (when $k = \frac{1}{2}$) compared to all higher dimensions.
  There is a rough heuristic argument that explains this.
  Counting solutions to $X^2 + Y^2 = Z^2 + h$ is the same as counting solutions to
  $(Z-X)(Z+X) = Y^2 - h$, which can be thought of as counting the number of factorizations
  of $Y^2 - h$ as $Y$ varies.
  The number of factorizations of $Y^2 - h$ depends heavily on whether or not $h$ is a
  square.
  If $h$ is not a square, then there are expected to be relatively few factorizations.
  If $h$ is a square, then there are expected to be logarithmically many factorizations in
  $Y$.
  Thus if $Y$ varies up to size $\sqrt X$ and there are $\log Y$ many factorizations on
  average, we should expect $X^{\frac{1}{2}} \log X$ terms.


  In higher dimensions, this factorization heuristic doesn't apply.
  And in higher dimensions, the regularity of representation of integers as sums of many
  squares should smooth away irregularities present in low dimension.
\end{remark}





% vim: tw=90 cc=90
