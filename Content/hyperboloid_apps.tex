


In this Chapter, we highlight one particular application of counting lattice points on
one-sheeted hyperboloids.
In particular, we describe how to understand sums of the form
\begin{equation}
  \sum_{n \leq X} d (n^2 + 1),
\end{equation}
where $d(n)$ denotes the number of divisors of $n$.



\section*{Connection to the Divisor Sum $\sum d(n^2 + 1)$}


The special three dimensional hyperboloid
\begin{equation}
  X^2 + Y^2 = Z^2 + 1
\end{equation}
is closely related to the divisor sum
\begin{equation}
  \sum_{n \leq R} d(n^2 + 1),
\end{equation}
which has been heavily studied by Hooley~\cite{Hooley63}.
This relationship is visible through the following theorem.


\begin{theorem}
  Let $d_o(n)$ denote the number of positive odd divisors of $n$.
  Then
  \begin{equation}
    \#\{\text{Integer points on } X^2 + Y^2 = Z^2 + 1 \; \text{with } \lvert Z \rvert \leq
    R \} = \sum_{n \leq R} 4 d_o(n^2 + 1).
  \end{equation}
\end{theorem}



To prove this theorem, we first prove this lemma.



\begin{lemma}
  Given an integer $Z$, we first show that $Z^2 + 1$ is not divisible by any prime
  congruent to $3 \bmod 4$.
\end{lemma}


\begin{proof}
  Indeed, factorize $Z + i$ as a product of Gaussian primes
  \begin{equation}
    Z + i = \pi_1^{k_1} \cdots \pi_r^{k_r}.
  \end{equation}
  Taking norms and letting $N(\pi_i) = p_i$, we have
  \begin{equation}
    Z^2 + 1 = p_1^{k_1} \cdots p_r^{k_r}.
  \end{equation}
  A Gaussian prime $\pi_j$ satisfies exactly one of the following:
  \begin{enumerate}
    \item $\pi_j = 1 + i$ or $\pi_j = 1 - i$ (in which case $\pi_j \mid 2$),
    \item $N(\pi_j) \equiv 1 \bmod 4$, or
    \item $\pi_j$ is inert, and is a prime congruent to $3 \bmod 4$ in $\mathbb{Z}$.
  \end{enumerate}
  Therefore if $Z^2+1$ is divisible by a prime congruent to $3 \bmod 4$, then there is an
  inert prime $\pi_j$ dividing $Z+i$ and (by conjugation) also $Z-i$.
  But then $\pi_j$ divides $Z+i - (Z-i) = 2i$, which is impossible as $\pi_j$ is inert and
  thus doesn't divide $2$.
  Therefore $Z^2+1$ is not divisible by a prime congruent to $3 \bmod 4$, and this
  concludes the proof of the sublemma.
\end{proof}



Returning to the proof of the theorem, it is a classical result that the number of ways of
writing a non-square $n$ as a sum of two squares is given by
$\frac{1}{2}(e_1+1)(e_2+1)\cdots(e_r+1)$, where $e_j$ is the multiplicity of the prime
$p_j$ congruent to $3 \bmod 4$ dividing $n$.
(This is under the assumption that $n$ can be written as the sum of two squares, which is
the case we are interested in).
In this formulation, note that we consider $X^2 + Y^2$ and $(-X)^2 + Y^2$ to be the same
representation.
Therefore the number of ways of writing $Z^2 + 1$ is $\frac{1}{2}(k_1+1)(k_2+1) \cdots
(k_r+1)$, excluding $2$ and its exponent from the list.
This is exactly half the number of odd divisors of $Z^2 + 1$, which we denote as
$\frac{1}{2}d_o(Z^2 + 1)$.



As each individual representation $X^2 + Y^2 = Z^2 + 1$ comes with the eight lattice
points $(\pm X, \pm Y, \pm Z)$, we see that the number of lattice point solutions to $X^2
+ Y^2 = Z^2 + 1$ with $\lvert Z \rvert \leq R$ is given by
\begin{equation}
  8 \sum_{Z \leq R}\tfrac{1}{2}d_o(Z^2+1).
\end{equation}
This completes the proof of the theorem. \qed{}




Note that when $n$ is even, all the divisors of $n^2 + 1$ are odd and $d(n^2 + 1) =
d_o(n^2 + 1)$.
On the other hand, when $n$ is odd, then $n^2 + 1$ is divisible by $2$ exactly once, and
$d(n^2 + 1)=2d_o(n^2 + 1)$.
Therefore it is possible to convert the summation to $\sum d(n^2 + 1)$.



As a corollary to the proof of the above theorem, one can prove the following.
\begin{corollary}
  \begin{align}
    \#\{\text{Integer points on } X^2 + Y^2 = (2Z)^2 + 1 \; \text{with } \lvert Z \rvert
    \leq R \} &= \sum_{n \leq R} 4 d_o(4n^2 + 1) \\
    &= \sum_{n \leq R} 4 d(4n^2 + 1).
  \end{align}
\end{corollary}
While this hyperboloid is not studied in this thesis, the methodology still applies and
it is possible to obtain asymptotics with error term for this lattice counting problem as
well.


Further, if we denote
\begin{align}
  N_1(R) &:= \#\{\text{Integer points on } X^2 + Y^2 = Z^2 + 1 \; \text{with } \lvert Z
  \rvert \leq R \}
  \\
  N_2(R) &:= \#\{\text{Integer points on } X^2 + Y^2 = (2Z)^2 + 1 \; \text{with } \lvert Z
  \rvert \leq R \},
\end{align}
then one can now easily compute that
\begin{equation}
  \sum_{n \leq R} d(n^2 + 1) = \frac{N_1(R)}{2} - \frac{N_2(R/2)}{4}.
\end{equation}
In this way, we convert a classical, and still somewhat mysterious, divisor sum into two
lattice counting problems.


In the future, it would be a good idea to optimize the arguments in
Chapter~\ref{c:hyperboloid} in order to try to improve estimates for divisor sums.
In particular, in the integral analysis for proving the main theorems, it is possible to
further shift lines of integration and handle resulting residue terms (although one would
also need to get a deeper understanding of the underlying analytic behavior).



\begin{remark}
  Similar techniques may be applied to study
  \begin{equation}
    \sum_{n \leq R} d(n^2 + h)
  \end{equation}
  for any positive $h$, although the computations look progressively messier.
\end{remark}





% vim: tw=90 cc=90
