

\hfill
\begin{minipage}{4in}
  \small
  \singlespacing{}
  Charles Darwin had a theory that once in a while one should perform a damn-fool
  experiment.
  It almost always fails, but when it does come off is terrific.\\

  Darwin played the trombone to his tulips.
  The result of this particular experiment was negative.
  \vspace{.3ex}
  \begin{flushright}
    {Littlewood, \textit{A Mathematician's Miscellany}}
  \end{flushright}
\end{minipage}





In this thesis, we study and introduce new methods to study problems that are closely
related to the Gauss Circle problem.
In this introductory chapter, we motivate and explain the Gauss Circle problem and how it
relates to the other problems described in later chapters.
In Section~\ref{sec:intro:GaussCircleExplanation}, we give a historical overview of the
classical Gauss Circle and Dirichlet Hyperbola problems.




\section{Gauss Circle Problem}\label{sec:intro:GaussCircleExplanation}




The Gauss Circle problem is the following seemingly innocent question:
\index{Gauss Circle problem}
\begin{quote}
  How many integer lattice points lie inside or on the circle of radius $\sqrt R$ centered
  at the origin? That is, how many points $(x,y) \in \mathbb{Z}^2$ satisfy $x^2 + y^2 \leq
  R$?
\end{quote}
We will use $S_2(R)$ to denote the number of lattice points inside or on the circle of
radius $\sqrt R$ centered at the origin.
Intuitively, it is clear that $S_2(R)$ should be approximately the area of the circle,
$\Vol B(\sqrt R) = \pi R$.
This can be made rigorous by thinking of each lattice point as being the center of a $1
\times 1$ square in the plane and counting those squares fully contained within the circle
and those squares lying on the boundary of the circle.
Using this line of thought, Gauss proved (in 1798) that
\begin{equation}
  \lvert S_2(R) - \Vol B(\sqrt R) \rvert \ll \sqrt R,
\end{equation}
so that the \emph{lattice point discrepancy}\index{lattice point discrepancy} between the
number of lattice points within the circle and the area of the circle was bounded by some
constant times the circumference.


It may appear intuitive that this is the best one could hope for.\footnote{Indeed, I am
  not currently aware of an intuitive heuristic that explains \emph{why} the actual error
term is so much better.}
No further progress was made towards Gauss Circle problem until 1906, when
Sierpi\'{n}ski~\cite{sierpinski1906pewnem} showed that
\begin{equation}
  \lvert S_2(R) - \Vol B(\sqrt R) \rvert \ll R^{\frac{1}{3}},
\end{equation}
a significant improvement over the bound from Gauss.
This is already remarkable --- somehow lattice points in the plane are distributed in the
plane in a way that a growing circle includes and omits points in a way that offsets each
other.


It is natural to ask: What is the correct exponent of growth on the error term?
Hardy, Littlewood, and Cram\'{e}r~\cite{cramer1922} proved that, \emph{on average}, the
correct exponent is $\frac{1}{4}$ when they proved
\begin{equation}\label{eq:intro:hardy_littlewood_meansquare}
  \int_0^X \lvert S_2(r) - \Vol B(\sqrt r) \rvert^2 dr = c X^{\frac{3}{2}} +
  O(X^{\frac{5}{4} + \epsilon})
\end{equation}
for some constant $c$.
At approximately the same time, Hardy~\cite{Hardy1917} showed that
\begin{equation}
  \lvert S_2(R) - \Vol B(\sqrt T) \rvert = \Omega(R^{\frac{1}{4}}),
\end{equation}
so that $\frac{1}{4}$ is both a lower bound and the average bound.





At the same time (and often, even within the same works, such as in Hardy's work on the
average and minimum values of the error in~\cite{Hardy1917}), mathematicians were studying
Dirichlet's Divisor problem.
Let $d(n)$ denote the number of positive divisors of $n$.
Then Dirichlet's Divisor problem is to determine the average size of $d(n)$ on integers
$n$ up to $R$.\index{Dirichlet Divisor problem}\index{Dirichlet Hyperbola problem | \\see
{Dirichlet Divisor problem} }
Noting that
\begin{equation}
  \sum_{n \leq R} d(n) =
  \sum_{n \leq R} \sum_{d \mid n} 1 =
  \sum_{d \leq R} \Big\lfloor \frac{R}{d} \Big\rfloor,
\end{equation}
we see that this is equivalent to counting the number of positive integer lattice points
under the hyperbola $xy = R$.
(For this reason, some refer to this as Dirichlet's Hyperbola problem).


As noted already by Hardy~\cite{Hardy1917}, it's known that
\begin{equation}
  \sum_{n \leq R} d(n) = cR \log R + c' R + O(R^{\frac{1}{2}}),
\end{equation}
and that
\begin{equation}
  \int_1^X \Big\lvert \sum_{n \leq r} d(n) - c r \log r - c' r \Big\rvert^2 dr = c'' X^{\frac{3}{2}} +
  O(X^{\frac{5}{4} + \epsilon}),
\end{equation}
exhibiting analogous to the Gauss Circle problem.
Attempts to further understand the error terms in the Gauss Circle and Dirichlet Hyperbola
problems have indicated that there is a strong connection between the two, and often an
improvement to one of the problems yields an improvement to the other.


It is interesting to note that much of the early work of Hardy and Littlewood on
the Gauss Circle and Dirichlet Hyperbola problems occurred from 1914 to 1919, which are
the years when Ramanujan studied and worked with them at Cambridge.
Of particular import is a specific identity inspired by Ramanujan (as noted by Hardy
in~\cite{Hardy1959onRamanujan}) that is now sometimes called ``Hardy's
Identity''\footnote{This is another instance of Stigler's Law of Eponymy.},
which states that
\begin{equation}\label{eq:intro:HardysIdentity}
  S_2(R) - \Vol B(\sqrt R) =  \sqrt{R} \sum_{n \geq 1} \frac{r_2(n)}{n^{1/2}} J_1(2\pi
  \sqrt{n R}),
\end{equation}
in which $r_2(n)$ is the number of ways of writing $n$ as a sum of $2$ squares and $J_\nu$ is
the ordinary Bessel function
\begin{equation}
  J_\nu(z) := \sum_{n \geq 0} \frac{(-1)^n}{\Gamma(n+1) \Gamma(\nu + n + 1)}
  (z/2)^{\nu + 2n}.
\end{equation}
Ivi\'c has repeatedly noted in~\cite{ivic2004lattice} and~\cite{ivic1996} that  almost all
significant progress towards both the Gauss Circle and Dirichlet Hyperbola problems have
come from identities and approaches similar to~\eqref{eq:intro:HardysIdentity}, though
sometimes obscured through technical details.





\subsection{An Early Connection to Modular Forms}


One of the topics that Ramanujan devoted himself to was what we now call the ``Ramanujan
tau function,''\index{Ramanujan $\tau$}
which can be defined by matching coefficients in
\begin{equation}
  \sum_{n \geq 1} \tau(n) q^n = q \prod_{n \geq 1} (1 - q^n)^{24}.
\end{equation}
Ramanujan noticed or conjectured that this function satisfies many nice properties,
such as being multiplicative.
The individual $\tau(n)$ satisfy the bound
\begin{equation}
  \tau(n) \ll n^{\frac{11}{2} + \epsilon},
\end{equation}
and numerical experimentation might lead one to conjecture that
\begin{equation}
  \sum_{n \leq R} \tau(n) \ll R^{\frac{11}{2} + \frac{1}{4} + \epsilon}.
\end{equation}
Just like the Gauss Circle and Dirichlet Hyperbola problem, it appears that summing over
$R$ many terms contributes only $\frac{1}{4}$ to the exponent in the size.
Further, Chandrasekharan and Narasimhan~\cite{chandrasekharan1962functional,
chandrasekharan1964mean} showed a result analogous to the average estimate of Hardy and
Littlewood, proving that
\begin{equation}
  \int_0^X \big\lvert \sum_{n \leq r} \tau(n) \big\rvert^2 dr = c X^{11 + \frac{3}{2}} + O(X^{12 +
  \epsilon}).
\end{equation}
This also indicates that the average additional exponent is $\frac{1}{4}$.



We now recognize that this is another analogy to the Gauss Circle problem (except that in
this case there is no main term).
This analogy readily generalizes.


The Ramanujan $\tau$ function appears as the coefficients of the unique weight $12$
holomorphic cusp form on $\SL(2, \mathbb{Z})$, usually written
\begin{equation}
  \Delta(z) = \sum_{n \geq 1} \tau(n) e(nz),
\end{equation}
where here and throughout this thesis, $e(nz) = e^{2\pi i n z}$.
Generally, we can consider a holomorphic cusp form $f$ on a congruence subgroup $\Gamma
\subseteq \SL(2, \mathbb{Z})$ of weight $k$, with Fourier expansion
\begin{equation}
  f(z) = \sum_{n \geq 1} a(n) e(nz).
\end{equation}
Let $S_f(R)$ denote the partial sum of the first $R$ Fourier coefficients,
\begin{equation}
  S_f(R) := \sum_{n \leq R} a(n).
\end{equation}
Then Chandrasekharan and Narasimhan also show that
\begin{equation}
  \int_0^X \lvert S_f(r) \rvert^2 dr = c X^{k + \frac{3}{2}} + O(X^{k + \epsilon}),
\end{equation}
indicating that \emph{on average}, the partial sums satisfy
\begin{equation}
  S_f(R) \ll R^{\frac{k-1}{2} + \frac{1}{4} + \epsilon}.
\end{equation}
In other words, partial sums of coefficients of cusp forms appear to satisfy a Gauss
Circle problem type growth bound.





Further, the Gauss Circle problem can be restated as a problem estimating
\begin{equation}
  S_2(R) = \sum_{n \leq R} r_2(n),
\end{equation}
where $r_2(n)$ denotes the number of representation of $n$ as a sum of two squares as
above.
The coefficients $r_2(n)$ appear as the coefficients of a the modular form $\theta^2(z)$,
so the analogy between $S_f$ and $S_2$ is very strong.
(However $\theta^2$ is not cuspidal, so there are some differences).



In Chapter~\ref{c:sums}, we consider this ``Cusp Form Analogy'' and study $S_f(n)$.
We introduce new techniques that are fundamentally different than most techniques employed
in the past.
Of particular interest is the introduction of Dirichlet series of the form
\begin{equation}
  \sum_{n \geq 1} \frac{S_f(n)^2}{n^s},
\end{equation}
including their meromorphic continuation and analysis.
In Chapter~\ref{c:sums_apps}, we give an overview of the completed and currently planned
applications of the analysis and techniques for studying $S_f(n)$ in Chapter~\ref{c:sums}.
This includes an overview of several recent papers of the author and his collaborators
where these techniques have been used successfully.






\subsection{Further Generalizations of the Gauss Circle Problem}



A very natural generalization of the Gauss Circle problem is to higher dimensions.
We will call the following the Gauss $d$-Sphere problem:


\begin{quote}
  How many integer lattice points lie inside or on the $d$-dimensional sphere of radius
  $\sqrt R$ centered at the origin? That is, how many points $\bm x \in \mathbb{Z}^d$
  satisfy $\| \bm x \|^2 \leq R$?
\end{quote}\index{Gauss $d$-Sphere problem}

We use $S_d(R)$ to denote the number of lattice points inside or on the $d$-sphere of
radius $\sqrt R$, as occurs in the Gauss $d$-Sphere problem.
As with the Gauss Circle problem, it is intuitively clear that $S_d(R) \approx \Vol
B_d(\sqrt R)$, where we use $B_d(\sqrt R)$ to denote a $d$-dimensional ball of radius
$\sqrt R$ centered at the origin.
Just as with the Gauss Circle problem, the true goal of the Gauss $d$-Sphere problem
is to understand the lattice point discrepancy $S_d(R) - \Vol B_d(\sqrt R)$.


The Gauss $d$-Sphere problem is most mysterious for low dimensions, but perhaps the
dimension $3$ Gauss Sphere problem is the most mysterious.
For an excellent survey on the status of this problem, see~\cite{ivic2004lattice}.


The Gauss $d$-Sphere problem is not considered in detail in this thesis, but in
Chapter~\ref{c:sums_apps} a new approach on aspects of the Gauss $d$-Sphere problem is
outlined which builds on the techniques of Chapter~\ref{c:sums}.
This is a project under current investigation by the author and his collaborators.




Closely related to the Gauss $d$-Sphere problem is the problem of determining the number
of lattice points that lie within $B_d(\sqrt R)$ \textbf{and} which lie on a one-sheeted
hyperboloid\index{one-sheeted hyperboloid problem}
\begin{equation}
  \mathcal{H}_{d,h} := X_1^2 + \cdots + X_{d-1}^2 = X_d^2 + h
\end{equation}
for some $h \in \mathbb{Z}_{\geq 1}$.
We use $N_{d,h}(R)$ to denote the number of lattice points within $B_d(\sqrt R)$ and on
$\mathcal{H}_{d,h}$.
This is a problem within the larger class of constrained lattice counting problems.


The dimension $3$ one-sheeted hyperboloid problem is the first nontrivial dimension, and
just as with the standard Gauss $d$-Sphere problem, the dimension $3$ hyperboloid is the
most enigmatic.
Very little is currently known.
A significant reason for the mystery comes from the fact that this is too small of a
dimension to apply the Circle Method of Hardy and Littlewood, and there doesn't seem to be
an immediate analogue of Hardy's Identify~\eqref{eq:intro:HardysIdentity}.
In fact, it's not quite clear what the correct separation between the main term and error
term should be.
The best known result is the recent result of Oh and Shah~\cite{ohshah2014}, which uses
ergodic methods to prove that
\begin{equation}\label{eq:intro:oh_shah}
  N_{3,h}(R) = c R^{\frac{1}{2}} \log R + O\big( R^{\frac{1}{2}} (\log R)^{\frac{3}{4}}
  \big)
\end{equation}
when $h$ is a positive square.



From the perspective of modular forms, the underlying modular object is the modular form
$\theta^{d-1}(z) \overline{\theta(z)}$.
But in contrast to the variants of the Gauss Circle problem discussed above, $N_{d,h}(R)$
is encoded within the $h$th Fourier coefficient of $\theta^{d-1}\overline{\theta}$, which
affects the shape of the analysis significantly.


In Chapter~\ref{c:hyperboloid}, we consider the problem of estimating the number
$N_{d,h}(R)$ of points on one-sheeted hyperboloids.
Along the way, we study the Dirichlet series
\begin{equation}
  \sum_{n \geq 1} \frac{r_{d-1}(n^2 + h)}{(2n^2 + h)^s}
\end{equation}
and its meromorphic properties, and use it to get improved estimates for $N_{d,h}(R)$.




\section{Outline and Statements of Main Results}



\subsubsection*{Chapter 2.}

Chapter~\ref{c:background} consists of background information used in later sections.
In particular, a variety of properties concerning Eisenstein series are discussed and
referenced from the literature.
The second half of Chapter~\ref{c:background} concerns a complete description of three
Mellin integral transforms.
It is shown that two smooth integral transforms can be used in tandem to establish a sharp
cutoff result, which is employed in Chapter~\ref{c:hyperboloid} and in the applications
described in Chapter~\ref{c:sums_apps}.



\subsubsection*{Chapters 3 and 4.}

Chapter~\ref{c:sums} is closely related to the journal article~\cite{hkldw}, which was
published in 2017.
In this Chapter, we introduce and study the meromorphic properties of the Dirichlet series
\begin{equation}\label{eq:intro:sums_dirichlet_series}
  \sum_{n \geq 1} \frac{S_f(n)}{n^s}, \quad \sum_{n \geq 1} \frac{S_f(n)
  \overline{S_g(n)}}{n^s}, \quad \text{and} \quad \sum_{n \geq 1}
  \frac{S_f(n)S_g(n)}{n^s},
\end{equation}
where $f$ and $g$ are weight $k$ cuspforms on a congruence subgroup $\Gamma \subseteq
\SL(2, \mathbb{Z})$ and where $S_f(n)$ and $S_g(n)$ denote the partial sums of the first
$n$ Fourier coefficients of $f$ and $g$, respectively.


The first major result is in Theorem~\ref{thm:Wsfgmero} and
Corollary~\ref{cor:DsSfSg_has_meromorphic}, which show that the series
in~\eqref{eq:intro:sums_dirichlet_series} have meromorphic continuation to the plane with
understandable analytic properties.


As a first application of this meromorphic continuation, we prove the major analytic
result of the chapter in Theorem~\ref{thm:second_moment_Sf}, which states that
\begin{equation}
  \sum_{n \geq 1} \frac{S_f(n) \overline{S_g(n)}}{n^{k-1}} e^{-n/X} = C X^{\frac{3}{2}} +
  O(X^{\frac{1}{2} + \epsilon}),
\end{equation}
and related results, for an explicit constant $C$.
In terms of the analogy to the Gauss Circle problem, this is a smoothed second moment and
is comparable in nature to the mean-square moment bound of Hardy and
Littlewood~\eqref{eq:intro:hardy_littlewood_meansquare}.
As the Dirichlet series in~\eqref{eq:intro:sums_dirichlet_series} are new, this smoothed
result is a new type of result.


Chapter~\ref{c:sums_apps} is a description of further work using the Dirichlet
series~\eqref{eq:intro:sums_dirichlet_series} and its meromorphic continuation.
This includes the work in the papers~\cite{hkldwShort} and~\cite{hkldwSigns}.
Further applications are described, as well as preliminary results from the author and his
collaborators on this topic.




\subsubsection*{Chapters 5 and 6}

Chapter~\ref{c:hyperboloid} focuses on studying the One-Sheeted Hyperboloid problem.
The Dirichlet series
\begin{equation}
  \sum_{n \in \mathbb{Z}} \frac{r_{d-1}(n^2 + h)}{(2n^2 + h)^s}
\end{equation}
is studied, and the first major result of the Chapter is in
Theorem~\ref{thm:hyperboloid:mero_summary}, which states that this Dirichlet series has
meromorphic continuation to the plane with understandable analytic behavior.

Using this meromorphic continuation, three applications are given.
Firstly, there is Theorem~\ref{thm:hyperboloid:smooth_full}, which gives a smoothed
analogue of the number of lattice points $N_{d,h}(R)$ on the one-sheeted hyperboloid
$\mathcal{H}_{d,h}$ and within $B_d(\sqrt R)$, including many second order main terms.
This Theorem can be interpreted as a long-interval smoothed average, with many secondary
main terms.


Secondly, there is Theorem~\ref{thm:hyp:concentrating_theorem_full}, which proves a
short-interval sharp average result.
But the most important result of the chapter is Theorem~\ref{thm:hyp:sharp_theorem_full},
which improves the state of the art estimate of Oh and Shah in~\eqref{eq:intro:oh_shah} by
showing that
\begin{equation}
  N_{3,h}(R) = C' R^{\frac{1}{2}} \log R + C R^{\frac{1}{2}} + O(X^{\frac{1}{2} -
  \frac{1}{20} + \epsilon})
\end{equation}
when $h$ is a positive square.
(When $h$ is not a square, the first term does not appear).
In fact, the theorem is more general and gives results for any dimension $d \geq 3$.


In Chapter~\ref{c:hyperboloid_apps}, we describe an application of the methods and
techniques of Chapter~\ref{c:hyperboloid} to asymptotics of sums of the form
\begin{equation}
  \sum_{n \leq X} d(n^2 + h),
\end{equation}
where $d(m)$ denotes the number of positive divisors of $m$.


% vim: tw=90
