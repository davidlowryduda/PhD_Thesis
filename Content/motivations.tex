
Let me describe briefly what we set out to prove and how different the final results
actually are.\footnote{Jeff Hoffstein once told me that far too few published works
  describe the difference between the original intent and the final version.
I include that here, in this very informal section.}


Chapters~\ref{c:sums} and~\ref{c:sums_apps} were inspired from a single question that Jeff
Hoffstein asked after a talk from Winfried Kohnen during Jeff's Birthday Conference.
Kohnen described new results on sign changes of cusp forms, and Jeff asked whether or not
it was possible to prove that sums of coefficients of cusp forms change sign frequently.


Very na{\"\i}vely, I thought that if I could understand this question, I would probably do
is using the series $D(s, S_f) = \sum_{n \geq 1} S_f(n) n^{-s}$ and $D(s, S_f \times
\overline{S_f}) = \sum_{n \geq 1} \lvert S_f(n) \rvert^s n^{-s}$.


The reason for this is simple: one way to determine that a sequence changes signs often is
to show that partial sums of squares are large while partial sums are small (indicating
lots of cancellation).
This fundamental idea was included within Kohnen's talk.
As for the use of these Dirichlet series --- it's in some sense the first thing that a
multiplicative number theorist might try.


I doodled on this question for a few talks, and shared my doodles with Tom Hulse.
Tom had two quick ideas: firstly, he knew of an exact set of conditions that guarantee
sign changes in intervals; secondly, he knew a Mellin-Barnes transform to decouple
denominators. This seemed like a short, quick project, so we set out to prove sign
changes.


We were wrong.
This was not short nor quick, and as we delved into the problem it became apparent that
there were significant obstacles in the way of understanding the spectral
analysis.
In some sense, we knew that these were understandable from the general philosophy
of~\cite{HoffsteinHulse13}.
Yet actually demonstrating the extent of cancellation required attention to different
subtleties.
And I engrossed myself into these details.


Several months later, the question had totally shifted away from sign changes, and instead
focused on the Cusp Form analogy to the Gauss Circle problem in various aspects, leading
ultimately to the current line of research.
In hindsight, it turns out that our analysis of $D(s, S_f)$ and $D(s, S_f \times
\overline{S_f})$ have had only limited success in actually proving sign changes --- but
they are great tools otherwise.



Chapters~\ref{c:hyperboloid} and~\ref{c:hyperboloid_apps} accomplished almost exactly what
was originally intended.
In~\cite{hulseCountingSquare}, Hulse, K{\i}ral, Kuan, and Lim studied a problem
inspired from the same work of Oh and Shah~\cite{ohshah2014} that studied lattice points
on hyperboloids through ergodic methods.
Shortly afterwards, a K{\i}ral, Kuan, and I began to look at lattice points on
hyperboloids.
Our initial investigations stalled, though I first learned much about the general
techniques involving shifted convolution sums from those first forays into this problem.


It is interesting to note that the combination of integral transforms that leads to the
main theorem of Chapter~\ref{c:hyperboloid} was noticed by a combination of Kuan, Walker,
and I originally while we were writing our sign changes paper~\cite{hkldwSigns}.
We then forgot about this technique until we began to struggle with our forthcoming
work on the Gauss Sphere problem, until Walker and I slowly realized that we were
vastly overcomplicating a particular difficulty.


Originally, the intention of Chapter~\ref{c:hyperboloid} was simply to prove the
meromorphic continuation of the underlying Dirichlet series and to prove the smoothed sum
result.
This is a rare case of proving exactly what I had set out to prove, and then a little
bit more.\footnote{As opposed to the normal pattern of failing in the original goal, and
  then proving something else entirely.
  As Jeff likes to say, it is important to be able to love the theorem that you can prove,
as rarely can we prove the theorems that we love.}





% vim: tw=90
