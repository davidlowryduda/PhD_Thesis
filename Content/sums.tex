

\section{Introduction}


Continuing with same notation as before, let $f$ be a holomorphic cusp form of positive
weight $k$ on a congruence subgroup $\Gamma \subseteq \SL(2, \mathbb{Z})$, where $k \in
\mathbb{Z} \cup ( \mathbb{Z} + \tfrac{1}{2})$.
Denote the Fourier expansion of $f$ at $\infty$ by
\begin{equation}
  f(z) = \sum_{n \geq 1} a(n) e(nz).
\end{equation}
The individual coefficients $a(n)$ have long been of interest since the coefficients
contain interesting arithmetic data.
For example, the major insight leading to the resolution of Fermat's Last Theorem involved
showing that for an elliptic curve $E$ there exists a corresponding modular form $f_E$
whose coefficients (at prime indices) satisfy
\begin{equation}
  a(p) = p + 1 - \# E(\mathbb{F}_p),
\end{equation}
or rather that the $a(p)$ counted the number of points on the elliptic curve over finite
fields.


The first cusp form to be studied in depth was the Delta Function (as described in
Chapter~\ref{c:introduction}), whose coefficients are the Ramanujan $\tau$ function,
\index{Ramanujan tau}
\begin{equation}
  \Delta(z) = \sum_{n \geq 1} \tau(n) e(nz).
\end{equation}
Ramanujan conjectured that the coefficients of $\Delta$ should satisfy the bound
\begin{equation}
  \lvert \tau(n) \rvert \ll d(n) n^{\frac{11}{2}},
\end{equation}
where $d(n)$ is the number of positive divisors of $n$.
This conjecture initiated an exploration that included a much wider set of objects than
Ramanujan could have dreamt of.



Ramanujan's Conjecture has been extended to include all modular and automorphic forms.
For a cusp form $f$ of weight $k$ in $\GL(2)$, the conjecture states that
\begin{equation}\label{eq:ramanujan_conjecture_an}
  a(n) \ll n^{\frac{k-1}{2} + \epsilon}.
\end{equation}
This is now known as the Ramanujan-Petersson conjecture, and it is now known to be true
for full integral weight $k$ holomorphic cusp forms on $\GL(2)$ as a consequence of
Deligne's proof of the Weil Conjecture~\cite{Deligne}.


It is an interesting coincidence that Hardy and Littlewood were investigating averaged
estimates for the Gauss Circle and Dirichlet Divisor problems when Ramanujan was arriving
in England, thinking about the Delta function.
It is easy to look at the first several $\tau$ coefficients and believe that the sign of
$\tau(n)$ changes approximately uniformly random in $n$.
Under this assumption, and assuming Ramanujan's Conjecture that $\tau(n) \ll
n^{\frac{11}{2} + \epsilon}$, it is very natural to conjecture that the summatory function
of $\tau(n)$ satisfies the square-root cancellation phenomenon,
\begin{equation}
  \sum_{n \leq X} \tau(n) \ll X^{\frac{11}{2} + \frac{1}{2} + \epsilon}.
\end{equation}
As described in Chapter~\ref{c:introduction}, this is analogous to the error terms $E(R)$
in the Gauss Circle or Dirichlet Divisor Problems.
We similarly expect the even better bound,
\begin{equation}
  \sum_{n \leq X} \tau(n) \ll X^{\frac{11}{2} + \frac{1}{4} + \epsilon}.
\end{equation}
Although there is a clear qualitative connection with the Circle and Divisor problems, it
seems unlikely that this common thread was recognized by Hardy, Littlewood, or Ramanujan
at the time.


For our general cusp form $f$ of weight $k$ in $\GL(2)$, we expect an analogous conjecture
to hold, which we refer to as the ``Classical Conjecture.''
\index{Classical Conjecture}


\begin{conjecture}[Classical Conjecture]
  Let $f(z) = \sum_{n \geq 1} a(n) e(nz)$ be a holomorphic cusp form of weight $k$ on
  $\GL(2)$, where $k \in \mathbb{Z}\cup(\mathbb{Z}+\frac{1}{2})$ and $k > 1$.
  Then\index{S@$S_f(n)$}
  \begin{equation}
    S_f(n) := \sum_{n \leq X} a(n) \ll X^{\frac{k-1}{2} + \frac{1}{4} + \epsilon}.
  \end{equation}
\end{conjecture}


In a seminal pair of works, Chandrasekharan and
Narasimhan~\cite{chandrasekharan1962functional, chandrasekharan1964mean} showed that the
Classical Conjecture is true \emph{on average} by showing that
\begin{equation*}
  \sum_{n \leq X} \lvert S_f(n) \rvert^2 = C X^{k-1 + \frac{3}{2}} + B(X)
\end{equation*}
where $B(X)$ is an error term satisfying
\begin{equation}
  B(X) = \begin{cases}
    O(X^k \log^2 X) \\
    \Omega(X^{k - \frac{1}{4}} \frac{(\log \log \log X)^2}{\log X}),
  \end{cases}
\end{equation}
and where $C$ is an explicitly known constant.


This should be thought of as a Classical Conjecture \emph{on average} due to the following
classical argument:
\begin{align}
  \left( \sum_{n \leq X} \lvert S_f(n) \rvert \right)^2 &= \left( \sum_{n \leq X} \lvert
  S_f(n) \rvert \cdot 1 \right)^2
  \\
  &\leq \sum_{n \leq X} \lvert S_f(n) \rvert^2 \sum_{m \leq X} 1 = X \sum_{n \leq X}
  \lvert S_f(n) \rvert^2
  \\
  &\ll X^{k-1 + \frac{5}{2}}.
\end{align}
The Cauchy-Schwarz-Bunyakovsky inequality is used to pass from the first line to the
second, and the bound of Chandrasekharan and Narasimhan is used to pass from the second to
the third.
Taking the square root of each side and dividing by $X$ gives
\begin{equation}
  \frac{1}{X} \sum_{n \leq X} \lvert S_f(n) \rvert \ll X^{\frac{k-1}{2} + \frac{1}{4}},
\end{equation}
which is precisely the statement that the Classical Conjecture holds \emph{on average}.
Their result is described more completely in \S\ref{ssec:CN}.



Building on this result, Hafner and Ivi\'c were able to show that for holomorphic cusp
forms of full integral weight on $\SL(2, \mathbb{Z})$, we know
\begin{equation}
  S_f(n) \ll X^{\frac{k-1}{2} + \frac{1}{3}}.
\end{equation}
The argument of Hafner and Ivi\'c only applies for holomorphic forms of full-integral
eight and of level one, but it is possible to provide some extension to their result using
their methodology.


In the rest of this chapter, we will examine a new method for examining the behavior of $S_f(n)$.
We will be able to study a slightly more general object.
Let $g = \sum_{n \geq 1} b(n) e(nz)$ be another modular form of weight $k$ and of the same
level as $f$.
The fundamental idea is to study the Dirichlet series
\index{Ds@$D(s, S_f \times S_f)$}
\begin{align}
  D(s, S_f) &= \sum_{n \geq 1} \frac{S_f(n)}{n^{s + \frac{k-1}{2}}} \\
  D(s, S_f \times S_g) &= \sum_{n \geq 1} \frac{S_f(n) S_g(n)}{n^{s + k-1}} \\
  D(s, S_f \times \overline{S_g}) &= \sum_{n \geq 1}
  \frac{S_f(n) \overline{S_g(n)}}{n^{s + k-1}}.
\end{align}
In the sequel, we show that these three Dirichlet series have meromorphic continuation to
$\mathbb{C}$.
In \S\ref{sec:secondmoment}, we show how to analyze these Dirichlet series
to prove results concerning average sizes of the partial sums $S_f(n)$.


\begin{remark}
  Note that the notation used in this thesis is different than the notation used in the
  series of papers~\cite{hkldw, hkldwShort, hkldwSigns}.
  In this thesis, we adopt the convention that has risen amidst the representation
  theoretic point of view on automorphic forms.
  Therefore $L(s, f \times f)$ in this thesis is the same as $L(s, f\times \overline{f})$
  in the papers.
  This difference is ultimately very minor, and does not change any aspect of the
  analysis.
\end{remark}




\section{Useful Tools and Notation Reference}

For ease of reference, we give a notational reference and a brief description of some of
the tools necessary for the analysis.



\subsection{The Rankin--Selberg $L$-function}\label{ssec:rankinselberg_lfunction}% chktex 8
\index{Rankin--Selberg $L$-function} % chktex 8
\index{L@$L(s, f\times g)$}


The Rankin--Selberg convolution $L$-function is described in detail % chktex 8
in~\cite{Goldfeld2006automorphic, Bump98}, but we summarize its construction and
properties.
Note that there is a choice of convention concerning notation for conjugation.
The more common convention is changing due to influence from more general lines of
inquiry.


Let $f(z) = \sum a(n) e(nz)$ and $g(z) = \sum b(n) e(nz)$ be modular forms of weight $k$
on a congruence subgroup $\Gamma \subseteq \SL(2, \mathbb{Z})$, where at least one is
cuspidal.
Let $\Gamma \backslash \mathcal{H}$ denote the upper half plane modulo the group action
due to $\Gamma$, and let $\langle f, g \rangle$ denote the Petersson inner product
\index{Petersson inner product}
\begin{equation}
  \langle f, g \rangle = \iint_{\Gamma \backslash \mathcal{H}} f(z) \overline{g(z)}
  \frac{dx dy}{y^2}.
\end{equation}
The Rankin--Selberg $L$-function is given by the Dirichlet series % chktex 8
\begin{equation}
  L(s, f\times \overline{g}) := \zeta(2s) \sum_{n \geq 1} \frac{a(n) \overline{b(n)}}{n^{s
  + k - 1}},
\end{equation}
which is absolutely convergent for $\Re s > 1$.
This $L$-function has a meromorphic continuation to all $s \in \mathbb{C}$ via the
identity
\begin{equation}\label{eq:Lsfgbar_equals_eisenstein}
  L(s, f\times \overline{g}) := \frac{(4\pi)^{s + k - 1} \zeta(2s)}{\Gamma(s + k - 1)}
  \langle \Im(\cdot)^k f \overline{g}, E(\cdot, \overline{s})\rangle,
\end{equation}
where $E(z,s)$ is the real-analytic Eisenstein series\index{E@$E(z,s)$}
\begin{equation}\label{eq:eisenstein_def}
  E(z,s) = \sum_{\gamma \in \Gamma_\infty \backslash \Gamma} \Im(\gamma z)^s.
\end{equation}



If, in~\eqref{eq:Lsfgbar_equals_eisenstein}, we replace $f \overline{g}$ with $f T_{-1}g$,
where $T_{-1}$ is the Hecke operator giving the action
\begin{equation}
  T_{-1} F(x + iy) = F(-x + iy),
\end{equation}
then one gets a meromorphic continuation of
\begin{equation}
  L(s, f\times g) = \zeta(2s) \sum_{n \geq 1} \frac{a(n) b(n)}{n^{s + k - 1}}.
\end{equation}
More details on the $T_{-1}$ Hecke operator can be found in the discussion leading up to
Theorem~3.12.6 of~\cite{Goldfeld2006automorphic}.
As the meromorphic properties of both are determined by the zeta function, Gamma function,
and Eisenstein series, we see that complex analytic arguments are very similar on either
$L(s, f\times \overline{g})$ or $L(s, f\times g)$.
We will only carry out the argument for $L(s, f\times \overline{g})$, and
describe any changes necessary to perform the argument on the other.


These Rankin--Selberg $L$-functions are holomorphic except for, % chktex 8
at most, a simple pole at $s = 1$ with residue proportional to $\langle f, g \rangle$.
When $\Gamma = \SL(2, \mathbb{Z})$, there is the functional equation
\begin{equation}
  (2\pi)^{-2s} \Gamma(s) \Gamma(s + k - 1) L(s, f\times \overline g) =: \Lambda(s, f\times
  \overline{g}) = \Lambda(1 - s, f\times \overline{g}),
\end{equation}
coming from the functional equation of the completed Eisenstein series
\begin{equation}
  \pi^{-s} \Gamma(s) \zeta(2s) E(z,s) =: E^*(z,s) = E^*(z, 1-s).
\end{equation}
There are analogous transformations for higher levels, but their formulation is a bit more
complicated due to the existence of other cusps.



\subsection{Selberg spectral expansion}\label{ssec:selberg-spectral}
\index{spectral expansion}
\index{Maass forms}
\index{mu@$\mu_j$ | see {Maass forms} }
\index{Selberg Eigenvalue Conjecture}


Let $L^2(\Gamma \backslash \mathcal{H})$ denote the space of square integrable functions
on $\Gamma \backslash \mathcal{H}$ with respect to the Petersson norm.
There is a complete orthonormal system for the residual and cuspidal spaces of $\Gamma
\backslash \mathcal{H}$, which we denote by $\{\mu_j(z): j \geq 0\}$, consisting of the
constant function $\mu_0(z)$ and infinitely many Maass cusp forms $\mu_j(z)$ for $j \geq
1$ with associated eigenvalues $\tfrac{1}{4} + t_j^2$ with respect to the hyperbolic
Laplacian.
Without loss of generality, we assume that the $\mu_j$ are also simultaneous
eigenfunctions of the standard Hecke operators, as well as the $T_{-1}$ operator.
Then for any $f \in L^2(\Gamma \backslash \mathcal{H})$, we have the \emph{spectral
decomposition} of $f$ given by
\begin{equation}
  f(z) = \sum_j \langle f, \mu_j \rangle \mu_j(z) + \sum_{\mathfrak{a}} \frac{1}{4\pi}
  \int_\mathbb{R} \langle f, E_\mathfrak{a}(\cdot, \tfrac{1}{2} + it)\rangle
  E_\mathfrak{a}(a, \tfrac{1}{2} + it) \; dt,
\end{equation}
where $\mathfrak{a}$ ranges of the cusps of $\Gamma \backslash \mathcal{H}$.  Throughout
this thesis, we will refer to the first sum as the \emph{discrete spectrum} and the sums
of integrals as the \emph{continuous spectrum}.
\index{discrete spectrum}
\index{continuous spectrum}
The spectral decomposition as presented here is a consequence of Selberg's Spectral
Theorem, as presented in Theorem~15.5 of~\cite{IwaniecKowalski04}.


To each Maass form $\mu_j$ is associated a type $\tfrac{1}{2} + it_j$, and these $it_j$
are expected to satisfy Selberg's Eigenvalue Conjecture, which says that all $t_j$ are
real.
In complete generality, it is known that $t_j$ is either purely real or purely imaginary.
Selberg's Eigenvalue Conjecture has been proved for many congruence subgroups, including
$\SL(2, \mathbb{Z})$, but it is not known in general.
\index{Selberg's Eigenvalue Conjecture}
We let $\theta = \sup_j \{ \lvert \Im(t_j) \rvert \}$ denote the best known progress
towards Selberg's Eigenvalue Conjecture for $\Gamma$.
\index{theta@$\theta$ in Selberg's Eigenvalue Conjecture}
The current best result for $\theta$ in all congruence subgroups that $\theta \leq
\tfrac{7}{64}$, due to Kim and Sarnak~\cite{KimSarnak03}.



\subsection{Decoupling integral transform}\label{ssec:mellinbarnes_decouple}
\index{Mellin Barnes transform}


We will use an integral analogue of the binomial theorem, originally considered by
Barnes~\cite{Barnes}, also presented in 6.422(3) of~\cite{GradshteynRyzhik07}.
\begin{lemma}[Barnes, 1908]
  If $0 > \gamma > - \Re s$ and $\lvert \arg t \rvert < \pi$, then
  \begin{equation}\label{eq:mellinbarnes_base}
    \frac{1}{2\pi i} \int_{(\gamma)} \Gamma(-z) \Gamma(s + z) t^z \; dz = \Gamma(s) (1 +
    t)^{-s}.
  \end{equation}
\end{lemma}
This is a corollary to an integral representation of the beta function,
\begin{equation}
  B(z, s) = \int_0^\infty \frac{x^z}{(1 + x)^{z + s}} \; \frac{dx}{x},
\end{equation}
which implies that
\begin{equation}\label{eq:mellinbarnes_beta_is_mellin_transform}
  B(z, s-z) = \int_0^\infty \frac{x^z}{(1+x)^{s}} \; \frac{dx}{x}.
\end{equation}
The right hand side is a Mellin transform,
so~\eqref{eq:mellinbarnes_beta_is_mellin_transform} indicates that $B(z, s-z)$ is the
Mellin transform of $(1+x)^{-s}$ (with auxiliary variable $z$).
Applying the Mellin Inversion Theorem (as shown in~\cite{titchmarshfourier}) and the
representation of the Beta function in terms of Gamma functions, $B(s,t) =
\Gamma(s)\Gamma(t) / \Gamma(s+t)$, we recover a proof of the Lemma.


We will apply this integral transform to decouple $m,n$ in $(m+n)^{-s} = m^{-s} (1 +
\tfrac{n}{m})^{-s}$.
It is easy to check that an application of the lemma (followed by a change of variables $z
\mapsto -z$) gives
\begin{equation}
  \frac{1}{(n+m)^s} = \frac{1}{2\pi i} \int_{(\gamma)}
  \frac{\Gamma(z)\Gamma(s-z)}{\Gamma(s)} \frac{1}{n^{s-z} m^z} \; dz.
\end{equation}



\subsection{Chandrasekharan and Narasimhan}\label{ssec:CN}
\index{Chandrasekharan and Narasimhan}


We will refer to a result of Chandrasekharan and Narasimhan through this thesis.
We combine~\cite[Theorem~4.1]{chandrasekharan1962functional}
and~\cite[Theorem~1]{chandrasekharan1964mean} to state the following
theorem of theirs.


\begin{theorem}[Chandrasekharan and Narasimhan, 1962 and 1964]
  Let $f$ and $g$ be objects with meromorphic Dirichlet series
  \begin{equation*}
    L(s,f) = \sum_{n \geq 1} \frac{a(n)}{n^s}, \qquad L(s,g)=\sum_{n\geq 1}
    \frac{b(n)}{n^s}.
  \end{equation*}
  Suppose $G(s) = Q^s\prod_{i=1}^\ell \Gamma(\alpha_i s + \beta_i)$ is a product of Gamma
  factors with $Q>0$ and $\alpha_i > 0$.
  Define $A = \sum_{i=1}^\ell \alpha_i$.
  Let $w$ and $w'$ be numbers such that $\sum_{n \leq X} \lvert b(n) \rvert^2 \ll X^{2w -
  1} \log^{w'} X$.
  Let
  \begin{equation*}
    Q(X) = \frac{1}{2\pi i} \int_{\mathcal{C}} \frac{L(s,f)}{s} X^s \ ds,
  \end{equation*}
  where $\mathcal{C}$ is a smooth closed contour enclosing all the singularities of the
  integrand.
  Let $q$ be the maximum of the real parts of the singularities of $L(s,f)$ and let $r$ be
  the maximum order of a pole of $L(s,f)$ with real part $q$.
  Suppose the functional equation
  \begin{equation}\label{eq:basic_feq}
    G(s)L(s,f) = \epsilon(f) G(\delta-s)L(\delta-s,g)
  \end{equation}
  is satisfied for some $\lvert \epsilon(f) \rvert = 1$ and $\delta >0$.
  Then we have that
  \begin{equation}
    \begin{split}\label{eq:CN_L1}
      S_f(X) =& \sum_{n\leq X} a(n) =
      Q(X) + O(X^{\frac{\delta}{2}-\frac{1}{4A}+2A(w-\frac{\delta}{2}-\frac{1}{4A})\eta +
      \epsilon}) \\
      & \quad + O(X^{q-\frac{1}{2A}-\eta}\log(X)^{r-1})+O\bigg( \sum_{X \leq n \leq X'}
    |a(n)| \bigg)
    \end{split}
  \end{equation}
  for any $\eta \geq 0$, and where $X' =X+O(X^{1-\frac{1}{2A}-\eta})$.
  If all $a(n) \geq 0$, the final $O$-error term above does not contribute.

  Suppose further that $A \geq 1$ and that $2w - \delta - \frac{1}{A} \leq 0$.
  Then
  \begin{equation*}
    \sum_{n \leq X} \lvert S_f(n) - Q(n) \rvert^2 = cX^{\delta - \frac{1}{2A} + 1} +
    O(X^{\delta} \log^{w' + 2} X)
  \end{equation*}
  for a constant $c$ that can be made explicit.
\end{theorem}

Thus from little more than a functional equation with understood Gamma factors, one can
produce nontrivial bounds on first and second moments.



\section{Meromorphic Continuation}
\index{D@$D(s, S_f \times S_f)$}


We will now produce the meromorphic continuations of $D(s, S_f)$ and $D(s, S_f \times
\overline{S_g})$.
The meromorphic continuation of $D(s, S_f \times S_g)$ follows from applying the exact
same methodology to $T_{-1} g$ in place of $g$, so we only write down the corresponding
results.
In this section, we explicitly show the results in the case when $\Gamma = \SL(2,
\mathbb{Z})$.
For higher level congruence subgroups, the same methodology will work.
We remark on this in \S\ref{sec:higherlevel}.


Throughout this section, let $f(z) = \sum a(n) e(nz)$ and $g(z) = \sum b(n) e(nz)$ be
weight $k$ cusp forms on $\SL(2, \mathbb{Z})$.
Define $S_f(n) := \sum_{m \leq n} a(m)$ to be the partial sum of the first $n$ Fourier
coefficients of $f$.
Define $S_g(n)$ similarly.


We first describe the meromorphic continuation of $D(s, S_f)$.
The main ideas of this continuation are very similar to those in the continuation of $D(s,
S_f \times \overline{S_g})$, but the details are much simpler.
We then produce the meromorphic continuation of $D(s, S_f \times \overline{S_g})$ in
\S\ref{ssec:mero_DsSfSg}.
We shall see that the main obstacle is understanding the shifted convolution sum
\begin{equation}
  \sum_{n,h \geq 1} \frac{a(n)\overline{b(n-h)}}{n^{s+k-1} h^w}.
\end{equation}
We then use these meromorphic continuations to understand $D(s, S_f \times
\overline{S_g})$ in the next section.



\subsection{Meromorphic continuation of $D(s, S_f)$}
\index{D@$D(s, S_f)$}


The meromorphic continuation of $D(s, S_f)$ is very simple.
The pattern of the proof is similar to the pattern necessary for $D(s, S_f \times
\overline{S_g})$, so it is useful to be very clear.
The proof proceeds in two steps:
\begin{enumerate}
  \item Decompose the Dirichlet series into sums of Dirichlet series that are easier to
    understand, and
  \item Understand the reduced Dirichlet series by relating them to $L$-functions.
\end{enumerate}


\begin{proposition}
  With $f$ and $S_f(n)$ as defined above, the Dirichlet series associated to $S_f(n)$
  decomposes into
  \begin{equation}\label{eq:DsSf_decomposition}
    D(s, S_f) := \sum_{n \geq 1} \frac{S_f(n)}{n^{s + \frac{k-1}{2}}} = L(s, f) +
    \frac{1}{2\pi i} \int_{(2)} L(s-z, f) \zeta(z) \frac{\Gamma(z) \Gamma(s +
    \frac{k-1}{2} - z)}{\Gamma(s + \frac{k-1}{2})} \; dz,
  \end{equation}
  valid for $\Re s > 3$.
  Here, $L(s, f)$ denotes the standard $L$-function associated to $f$, given by
  \begin{equation}
    L(s, f) := \sum_{n \geq 1} \frac{a(n)}{n^{s + \frac{k-1}{2}}},
  \end{equation}
  normalized to have functional equation of the form $s \mapsto 1-s$.
\end{proposition}


\begin{proof}
  We directly manipulate the Dirichlet series.
  \begin{equation}
    D(s, S_f) = \sum_{n \geq 1} \frac{S_f(n)}{n^{s + \frac{k-1}{2}}} = \sum_{\substack{n
    \geq 1 \\ m \geq 0}} \frac{a(n-m)}{n^{s + \frac{k-1}{2}}}.
  \end{equation}
  In this last equality, we adopt the convention that $a(n) = 0$ for $n \leq 0$ to
  simplify notation.
  Separate the $m = 0$ case and reindex the remaining sum with $n \mapsto n+m$ to get
  \begin{equation}
    \sum_{n \geq 1} \frac{a(n)}{n^{s + \frac{k-1}{2}}} + \sum_{m,n \geq 1}
    \frac{a(n)}{(n+m)^{s + \frac{k-1}{2}}}.
  \end{equation}


  The first sum is exactly $L(s, f)$.
  In the second sum, decouple $(n+m)^{-s}$ through the use of the Mellin-Barnes transform
  detailed in \S\ref{ssec:mellinbarnes_decouple}.
  For $ \gamma > 1$ and $\Re s$ sufficiently large, the $m$ sum can be collected into
  $\zeta(z)$ and the $n$ sum can be collected into $L(s + \frac{k-1}{2})$.
  Simplification completes the proof.
\end{proof}


The meromorphic continuation of the $L$-function $L(s,f)$ is well-understood.
Note that in
\begin{equation}
  \frac{1}{2\pi i} \int_{(2)} L(s - z, f) \zeta(z) \frac{\Gamma(z) \Gamma(s +
  \frac{k-1}{2} - z)}{\Gamma(s + \frac{k-1}{2})} \; dz,
\end{equation}
the integrand is meromorphic in both $s$ and $z$, and has exponential decay in vertical
strips in $\Im z$ for any individual $s$.
Therefore one can use the meromorphic continuation of $L(s,f)$ to understand that this
integral is meromorphic for all $s \in \mathbb{C}$.


Note that it is also possible to shift the line of $z$ integration arbitrarily far in the
negative direction, passing poles at $z = 1, 0, -1, \ldots$ and picking up their residues.
Shifting the line of $z$ integration to $\epsilon$ for a small $\epsilon > 0$ passes
exactly one pole, coming from $\zeta(z)$ at $z = 1$, with residue
\begin{equation}
  \Res_{z = 1} = L(s - 1, f) \frac{\Gamma(s + \frac{k-1}{2} - 1)}{\Gamma(s +
  \frac{k-1}{2})} = \frac{L(s-1, f)}{s + \frac{k-1}{2} - 1}.
\end{equation}
Therefore, we have the equality
\begin{equation}
  D(s, S_f) = L(s, f) + \frac{L(s-1, f)}{s + \frac{k-1}{2} - 1} + \frac{1}{2\pi i}
  \int_{(\epsilon)} L(s - z, f) \zeta(z) \frac{\Gamma(z)\Gamma(s + \frac{k-1}{2} -
  z)}{\Gamma(s + \frac{k-1}{2})} \; dz.
\end{equation}
The first term is analytic, the third term is analytic in $s$ for $\Re s > \epsilon -
\frac{k-1}{2}$, and the middle term appears to have a simple pole at $s = 1 -
\frac{k-1}{2}$.
However, from the functional equation equation of $L(s,f)$,
%TODO maybe refer back to the % definition of a modular form
\begin{equation}
  \Lambda(s,f) := (2\pi)^{-(s + \tfrac{k-1}{2})} \Gamma(s + \tfrac{k-1}{2}) L(s,f)
  =
  \varepsilon \Lambda(1-s, f),
\end{equation}
we see that the residue $L(-\tfrac{k-1}{2})$ is a trivial zero of the $L$-function,
and $s = 1 - \frac{k-1}{2}$ is not a pole after all.


Further shifting the line of $z$ integration to $-M + \epsilon$ for small $\epsilon > 0$
passes $(M)$ further poles, coming from $\Gamma(z)$ at $z = -j$ for $0 \leq j < M$.
The $j$th pole, located at $z = -j, 0 \leq j < M$ has residue
\begin{equation}
  \Res_{z = -j} = L(s + j, f) \zeta(-j) \frac{\Gamma(s + \frac{k-1}{2} + j)}{\Gamma(s +
  \frac{k-1}{2})} \frac{(-1)^j}{j!}.
\end{equation}
In each residue, the $L$-function is analytic, and all apparent poles of the Gamma
function in the numerator are cancelled by poles of the Gamma function in the denominator.
Therefore each residue is analytic in $s$.


For any integer $M \geq 0$, we therefore have that
\begin{equation}
  D(s, S_f) = L(s, f) + \sum_{j = -1}^{M-1} \Res_j + \frac{1}{2\pi i} \int_{(-M +
  \epsilon)} L(s - z, f)\zeta(z) \frac{\Gamma(z) \Gamma(s + \frac{k-1}{2} - z)}{\Gamma(s +
\frac{k-1}{2})} dz.
\end{equation}
The $L$-function and residues are analytic in $s$.
The integral term is analytic in $s$ for $\Re s > -\frac{k-1}{2} - M + \epsilon$.
Since $M$ is arbitrary, we see that $D(s, S_f)$ is actually analytic for all $s \in
\mathbb{C}$.
We record this observation as a corollary to the decomposition of $D(s, S_f)$.


\begin{corollary}
  The Dirichlet series $D(s, S_f)$ has analytic continuation to the whole complex plane,
  given by~\eqref{eq:DsSf_decomposition}.
\end{corollary}


\begin{remark}
  I note that if $f$ is not cuspidal, then the decomposition and most of the analysis of
  $D(s, S_f)$ carries over verbatim, with one key difference: the value
  $L(-\tfrac{k-1}{2})$ is no longer a trivial zero.
  Therefore it is possible to show in general that $D(s, S_f)$ is meromorphic in the plane
  with at most several simple poles with residues given by special values of $L(s, f)$.

  This indicates a very strong parallel between the properties of $L(s, f)$ and
  $D(s, S_f)$.
  But it should be noted that $D(s, S_f)$ does not have a simple functional equation or an
  Euler product.
\end{remark}



\subsection{Meromorphic continuation of $D(s, S_f \times \overline{S_g})$}
\label{ssec:mero_DsSfSg}


The meromorphic continuation of $D(s, S_f \times \overline{S_g})$ is a bit involved, but
the approach is very similar to the approach for $D(s, S_f)$.
We proceed in three steps:
\begin{enumerate}
  \item Decompose $D(s, S_f \times \overline{S_g})$ into sums of Dirichlet series that are
    easier to understand,
  \item Related the reduced Dirichlet series to $L$-functions and convolution sums, and
  \item Combine the analytic properties of the $L$-functions and convolution sums.
\end{enumerate}
The first two steps are fundamentally the same as in$D(s, S_f)$, except that convolution
sums are necessary in the analysis.
As the meromorphic properties of convolution sums are significantly more delicate,
ascertaining the final analytic properties will take much more work.
The final step is deferred to \S\ref{sec:analyticbehavior}.


\index{Ds@$D(s, S_f \times S_f)$}
\index{Ws@$W(s; f, f)$}
\begin{proposition}\label{prop:SfSg_decomposition}
  With $f, g, S_f(n)$, and $S_g(n)$ as defined above, the Dirichlet series associated to
  $S_f(n)\overline{S_g(n)}$ decomposes into
  \begin{equation}
    \begin{split}
      D(s, &S_f \times \overline{S_g}) := \sum_{n \geq 1} \frac{S_f(n)
      \overline{S_g(n)}}{n^{s + k - 1}} \\
      &= W(s; f,\overline{g}) + \frac{1}{2\pi i} \int_{(\gamma)} W(s - z; f,\overline{g})
      \zeta(z) \frac{\Gamma(z) \Gamma(s - z + k - 1)}{\Gamma(s + k - 1)} \; dz,
    \end{split}
  \end{equation}
  for $1 < \gamma < \Re(s-1)$.
  Here, $W(s; f, \overline{g})$ denotes
  \begin{equation}
    W(s; f,\overline{g}) := \frac{L(s, f\times \overline{g})}{\zeta(2s)} + Z(s, 0, f\times
    \overline{g}),
  \end{equation}
  $L(s, f\times \overline{g})$ denotes the Rankin--Selberg $L$-function % chktex 8
  as in~\ref{ssec:rankinselberg_lfunction},
  and $Z(s, w, f\times \overline{g})$ denotes the symmetrized
  shifted convolution sum\index{Zs@$Z(s, w, f\times f)$}
  \begin{equation}
    Z(s,w,f\times \overline{g}) := \sum_{n,h \geq 1} \frac{a(n)\overline{b(n-h)} +
    a(n-h)\overline{b(n)}}{n^{s + k - 1} h^w}.
  \end{equation}
\end{proposition}


\begin{proof}
  Expand and recollect the partial sums $S_f$ and $S_g$.
  \begin{equation}
    D(s, S_f \times \overline{S_g}) = \sum_{n \geq 1} \frac{S_f(n) \overline{S_g(n)}}{n^{s
    + k - 1}} = \sum_{n \geq 1} \frac{1}{n^{s + k - 1}} \sum_{m = 1}^n a(m) \sum_{h = 1}^n
    \overline{b(h)}.
  \end{equation}
  Separate the sums over $m$ and $h$ into the cases where $m=h, m>h$, and $m<h$.
  We again adopt the convention that $a(n) = 0$ for $n \leq 0$ to simplify notation.
  Reorder the sums, summing down from $n$ instead of up to $n$, giving
  \begin{equation}
    \sum_{n \geq 1} \frac{1}{n^{s + k - 1}} \Bigg( \sum_{h = m > 0} + \sum_{h > m \geq 0}
    + \sum_{m > h \geq 0}\Bigg) a(n-m) \overline{b(n-h)}.
  \end{equation}
  In the first sum, take $h = m$.
  In the second sum, when $h > m$, we let $h = m + \ell$ and then sum over $m$ and $\ell$.
  Similarly in the third sum, when $m > h$, we let $m = h + \ell$.
  Together, this yields
  \begin{equation}
    \begin{split}
      \sum_{n \geq 1} &\frac{1}{n^{s + k - 1}} \bigg( \sum_{m \geq 0} a(n-m)
      \overline{b(n-m)} \\
      &+ \sum_{\substack{\ell \geq 1 \\ m \geq 0}} a(n-m) \overline{b(n-m-\ell)} +
    \sum_{\substack{\ell \geq 1 \\ m \geq 0}} a(n-m-\ell)\overline{b(n-m)}\bigg).
    \end{split}
  \end{equation}


  In each sum, the cases when $m = 0$ are distinguished.
  Altogether, these contribute
  \begin{equation}
    W(s; f,\overline{g}) = \frac{L(s, f\times \overline{g})}{\zeta(2s)} + Z(s, 0, f\times
    \overline{g}).
  \end{equation}
  Within $W(s; f, \overline{g})$, one should think of $L(s, f\times
  \overline{g})\zeta(2s)^{-1}$ as the diagonal part of the double summation, while
  $Z(s, 0, f\times \overline{g})$ contains the off-diagonal, written as the sum of the
  above-diagonal and below-diagonal parts of the double summation.


  Reindexing by changing $n \mapsto n + m$, the remaining sums with $m \geq 1$ can be rewritten as
  \begin{equation}
    \sum_{m,n \geq 1} \frac{1}{(n+m)^{s + k - 1}}\bigg( a(n) \overline{b(n)} + \sum_{\ell
    \geq 1} a(n) \overline{b(n - \ell)} + \sum_{\ell \geq 1} a(n - \ell)
  \overline{b(n)}\bigg).
  \end{equation}
  Decouple $(n+m)^{-(s + k - 1)}$ through the use of the Mellin-Barnes transform detailed
  in~\S\ref{ssec:mellinbarnes_decouple}.
  Restricting to $\gamma > 1$ and $\Re s$ sufficiently large, we can collect the $m$ sum
  into $\zeta(z)$ and the $n$ sum can be colleced into $W(s; f,g)$.
  Simplification completes the proof.
\end{proof}


As in the case of $D(s, S_f)$, the individual pieces $L(s, f\times \overline{g})
\zeta(2s)^{-1}$ and $Z(s, w, f\times \overline{g})$ are known to have meromorphic
continuation to the complex plane.
The Rankin--Selberg $L$-function is classical, % chktex 8
and its meromorphic continuation is explained in~\S\ref{ssec:rankinselberg_lfunction}.
The meromorphic properties of $Z(s,w, f\times \overline{g})$ are extensively studied
in~\cite{HoffsteinHulse13}.
One should expect to be able to perform an analysis similar to the analysis for $D(s,
S_f)$ to study $D(s, S_f \times \overline{S_g})$, perhaps by shifting the line of $z$
integration and analyzing residue terms.


However, the shifted sums $Z(s, 0, f\times \overline{g})$ show miraculous cancellation
with the diagonal $L(s, f\times\overline{g}) \zeta(2s)^{-1}$ that does not occur in $Z(s,
w, f\times \overline{g})$ for general $w$.
We group $W(s; f,\overline{g})$ into a single object and study its analytic properties
in the next section.




\section{Analytic Behavior of $D(s, S_f \times \overline{S_g})$}\label{sec:analyticbehavior}
\index{Zs@$Z(s, w, f\times f)$}


In this section, we will understand the meromorphic continuation of $D(s, S_f \times
\overline{S_g})$ by studying the analytic properties of $W(s; f, \overline{g})$.
Although we work in level $1$, the methodology in this section generalizes to arbitrary
level and half-integral weight forms.
Therefore, we will use $\theta = \sup_j \{ \lvert \Im(t_j) \rvert \}$ to denote progress
towards Selberg's Eigenvalue Conjecture (as described in~\S\ref{ssec:selberg-spectral})
even though it is known that $\theta = 0$ in the level $1$ case.


We first produce a spectral expansion for the off-diagonal component, the symmetrized
shifted double Dirichlet series
\begin{equation}
  Z(s,w, f\times \overline{g}) := \sum_{m \geq 1} \sum_{\ell \geq 1} \frac{a(m)
  \overline{b(m - \ell)} + a(m - \ell)\overline{b(m)}}{m^{s + k - 1} \ell^w}.
\end{equation}
We then use this to understand the analytic behavior of $W(s; f, \overline{g})$ and, from
this, the analytic behavior of $D(s, S_f \times \overline{S_g})$.



\subsubsection{Spectral expansion}


For each integer $h \geq 1$, define the weight zero Poincar\'{e} series on $\Gamma$,
\index{Poincar\'{e} series}
\begin{equation}
  P_h(z,s):=\sum_{\gamma \in\Gamma_\infty \backslash \Gamma} \Im(\gamma z)^s e\left(h
  \gamma z\right),
\end{equation}
defined initially for $\Re(s)$ sufficiently large, but with meromorphic continuation to
all $s\in\mathbb{C}$.


Recall $T_{-1}$ from~\S\ref{ssec:rankinselberg_lfunction}.
Let $\mathcal{V}_{f,\overline{g}}(z) :=y^k (f \overline{g}+T_{-1}(f\overline{g}))$.
Note that $\mathcal{V}_{f, \overline{g}}(z) \in L^2(\Gamma \backslash \mathcal{H})$, so
the Petersson inner product $\langle \mathcal{V}_{f, \overline{g}}, P_h(\cdot,
\overline{s}) \rangle$ converges.
By expanding this inner product, we get
\begin{equation}
  \langle \mathcal{V}_{f,\overline{g}} ,P_h(\cdot, \overline{s}) \rangle
  =\frac{\Gamma\left({s}+k -1\right)}{\left(4\pi \right)^{{s}+ k -1}}
  D_{f,\overline{g}}({s};h),
\end{equation}
where we define $D_{f, \overline{g}}$ to have analogous notation as
in~\cite{HoffsteinHulse13},
\index{D@$D_{f,f}(s;h)$}
\begin{equation}
  D_{f,g}(s;h) := \sum_{n \geq
  1}\frac{a(n)\overline{b(n-h)}+a(n-h)\overline{b(n)}}{n^{s+k-1}},
\end{equation}
which converges absolutely for $\Re s$ sufficiently positive.
Dividing by $h^w$ and summing over $h \geq 1$ recovers $Z(s, w, f\times \overline{g} )$,
\begin{equation}\label{eq:Zsw_poincare}
  Z(s, w, f\times \overline{g} ) := \sum_{n, h \geq 1} \frac{D_{f, \overline{g}
  }(s;h)}{h^w} = \frac{(4\pi)^{s+k-1}}{\Gamma(s+k-1)} \sum_{h \geq 1} \frac{\langle
  \mathcal{V}_{f,\overline{g} }, P_h \rangle}{h^w},
\end{equation}
for $\Re s$ and $\Re w$ sufficiently positive.


We will obtain a meromorphic continuation of $Z(s, w, f\times \overline{g} )$ by using the
spectral expansion of the Poincar\'e series and substituting it
into~\eqref{eq:Zsw_poincare}.
Let $\{\mu_j\}$ be an orthonormal basis of Maass eigenforms with associated types
$\frac{1}{2} + it_j$ for $L^2(\Gamma \backslash \mathcal{H})$ as in
\S\ref{ssec:selberg-spectral}, each with Fourier expansion
\begin{equation}
  \mu_j(z)=\sum_{n \neq 0} \rho_j(n)y^{\frac{1}{2}}K_{it_j}(2\pi \vert n \vert y)
  e^{2\pi i n x}.
\end{equation}


The spectral expansion of the Poincar\'{e} series is given by
\begin{align}\label{eq:Pspectral}
  \begin{split}
    P_h(z,s)&=\sum_j \langle P_h(\cdot,s),\mu_j \rangle \mu_j(z) \\
            &\quad + \frac{1}{4\pi}\int_{-\infty}^\infty\langle
    P_h(\cdot,s),E(\cdot,\tfrac{1}{2}+it)\rangle E(z,\tfrac{1}{2}+it)\,dt.
  \end{split}
\end{align}
We shall refer to the above sum and integral as the discrete and continuous spectrum,
respectively, similar to the convention in~\S\ref{ssec:selberg-spectral}.


The inner product of $\mu_j$ against the Poincar\'{e} series gives
\begin{equation}\label{eq:mujP}
  \langle P_h(\cdot,s),\mu_j \rangle = \frac{\overline{\rho_j(h)}\sqrt{\pi}}{(4\pi
  h)^{s-\frac{1}{2}}}
  \frac{\Gamma(s-\frac{1}{2}+it_j)\Gamma(s-\frac{1}{2}-it_j)}{\Gamma(s)}.
\end{equation}
\begin{remark}
  In the computation of this inner product and the inner product of the Eisenstein series
  against the Poincar\'e series, we use formula~\cite[$\S$6.621(3)]{GradshteynRyzhik07} to
  evaluate the final integrals.
\end{remark}

%%%%%%%%%% NOTE GR 6.621(3) for reference %%%%%%%%%%
%
% For completeness, and since I made an error before, let me copy down $\S$6.621(3).
% \[
%   \int_0^\infty x^{\mu-1}e^{-\alpha x}K_{\nu}(\beta x)dx = \frac{\sqrt \pi
%   (2\beta)^\nu}{(\alpha + \beta)^{\nu + \mu}}\frac{\Gamma(\mu + \nu)\Gamma(\mu -
%   \nu)}{\Gamma(\mu + \tfrac{1}{2})}F\left( \mu+\nu, \nu + \tfrac{1}{2}; \mu +
%   \tfrac{1}{2}; \frac{\alpha- \beta}{\alpha + \beta} \right),
% \]
% and for us, we have
% \[
%   \begin{cases}
%     \mu = s - \tfrac{1}{2} \\
%     \nu = it_j \\
%     \alpha = 1 \\
%     \beta = 1
%   \end{cases}
% \]
% and the hypergeometric function is (very conveniently) exactly $1$ as $\alpha - \beta =
% 0$.
%%%%%%%%%% END NOTE GR %%%%%%%%%%


Let $E(z,w)$ be the Eisenstein series on $\SL(2, \mathbb{Z})$, given by
\begin{equation}
  E(z,w) = \sum_{\gamma \in \Gamma_\infty \backslash \SL(2, \mathbb{Z})} (\Im \gamma z)^s.
\end{equation}
Then $E(z,w)$ has Fourier expansion (as in~\cite[Chapter 3]{Goldfeld2006automorphic})
\begin{align}\label{eq:sums:E_Fourier}
  E(z,w)& =y^w + \phi(w)y^{1-w}  \\
  &\quad +  \frac{2\pi^w \sqrt{y} }{\Gamma(w)\zeta(2w)}\sum_{m \neq 0} \vert m
  \vert^{w-\frac{1}{2}}\sigma_{1-2w}(\vert m \vert) K_{w-\frac{1}{2}}(2\pi \vert m \vert
  y)e^{2\pi i m x}, \notag
\end{align}
where
\begin{equation}
  \phi(w) = \sqrt{\pi} \frac{\Gamma(w - \tfrac{1}{2})\zeta(2w - 1)}{\Gamma(w)\zeta(2w)}.
\end{equation}
The Petersson inner product of the Poincar\'{e} series ($h \geq 1$) against the Eisenstein
series $E(z,w)$ is given by
\begin{align}\label{eq:PhE}
  \left\langle P_h(\cdot,s),E(\cdot,w)\right\rangle =\frac{2\pi^{\overline{w}+\frac{1}{2}}
  h^{\overline{w}-\frac{1}{2}}\sigma_{1-2\overline{w}}(h)}{\zeta(2\overline{w})(4\pi
  h)^{s-\frac{1}{2}}}\frac{\Gamma(s+\overline{w}-1)\Gamma(s-\overline{w})}{\Gamma(\overline{w})\Gamma(s)},
\end{align}
provided that $\Re s >\frac{1}{2}+\vert \Re w-\frac{1}{2}\vert$.
For $t$ real, $w=\frac{1}{2}+it$, and $\Re s>\frac{1}{2}$, we can
specialize~\eqref{eq:PhE} to
\begin{equation}\label{eq:PhEspecialized}
  \langle P_h(\cdot,s),E(\cdot,\tfrac{1}{2}+it)\rangle=\frac{2\sqrt{\pi}
  \sigma_{2it}(h)}{\Gamma(s)(4\pi
  h)^{s-\frac{1}{2}}}\frac{\Gamma(s-\frac{1}{2}+it)\Gamma(s-\frac{1}{2}-it)}{h^{it}\zeta^*(1-2it)},
\end{equation}
in which $\zeta^*(2s):=\pi^{-s}\Gamma(s)\zeta(2s)$ denotes the completed zeta function.


Now that we have computed the inner products of the Eisenstein series and Maass forms with
the Poincar\'e series, we are ready to analyze the spectral expansion.
After substituting~\eqref{eq:mujP} into the discrete part of~\eqref{eq:Pspectral}, the
discrete spectrum takes the form
\begin{equation}
  \frac{\sqrt{\pi}}{(4\pi h)^{s-\frac{1}{2}}\Gamma(s)}\sum_j \overline{\rho_j(h)}
  \Gamma(s-\tfrac{1}{2}+it_j)\Gamma(s-\tfrac{1}{2}-it_j) \mu_j(z)
\end{equation}
and is analytic in $s$ in the right half-plane $\Re s> \frac{1}{2}+\theta$.
After inserting~\eqref{eq:PhEspecialized}, the continuous spectrum takes the form
\begin{equation}
  \frac{\sqrt{\pi}}{2\pi(4\pi h)^{s-\frac{1}{2}}}\int_{-\infty}^\infty
  \frac{\sigma_{2it}(h)}{h^{it}}\frac{\Gamma(s-\frac{1}{2}+it)
  \Gamma(s-\frac{1}{2}-it)}{\zeta^*(1-2it)\Gamma(s)}E(z,\tfrac{1}{2}+it)\,dt,
\end{equation}
which is analytic in $s$ for $\Re s > \frac{1}{2}$ and has apparent poles when $\Re
s=\frac{1}{2}$.


Substiting this spectral expansion into~\eqref{eq:Zsw_poincare} and summing over $h \geq
1$ recovers an expression for all of $Z(s, w, f\times \overline{g})$.
Recognizing the Dirichlet series (as described in~\cite{Goldfeld2006automorphic}, for
instance)
\begin{align}
  \sum_{h \geq 1} \frac{\rho_j(h)}{h^{s + w - \frac{1}{2}}} &= L(s + w - \tfrac{1}{2},
  \mu_j) \\
  \sum \frac{\sigma_{1-2w}(h)}{h^{s + \frac{1}{2} - w}} &= L(s, E(\cdot, w)) = \zeta(s + w
  - \tfrac{1}{2})\zeta(s - w + \tfrac{1}{2}),
\end{align}
we are able to execute the $h$ sum completely.
Assembling it all together, we have proved the following proposition.


\begin{proposition}\label{prop:spectralexpansionfull}
  For $f,g$ weight $k$ forms on $\SL_2(\mathbb{Z})$, the shifted convolution sum $Z(s, w,
  f\times \overline{g} )$ can be expressed as
\begin{align}
  Z(s&, w, f\times \overline{g} ) := \sum_{m=1}^\infty \frac{a(m)
  \overline{b(m-h)}+a(m-h)\overline{b(m)} }{m^{s+k -1}h^w} \nonumber \\
  &= \frac{(4\pi )^k}{2} \sum_j\rho_j(1) G(s, i t_j) L(s + w -\tfrac{1}{2},\mu_j)\langle
\mathcal{V}_{f,\overline{g} },\mu_j \rangle \label{line:1spectralexp} \\
  &\quad+\frac{(4\pi)^{k}}{4\pi i}\int_{(0)} G(s, z) \mathcal{Z}(s,w,z) \langle
  \mathcal{V}_{f,\overline{g} },E(\cdot,\tfrac{1}{2}-\overline{z})\rangle \,dz,
  \label{line:2spectralexp}
\end{align}
when $\Re (s+w)>\frac{3}{2}$, where $G(s, z)$ and $\mathcal{Z} (s,w,z)$ are the collected
$\Gamma$ and $\zeta$ factors of the discrete and continuous spectra,
\begin{align*}
  G(s, z) &= \frac{\Gamma(s - \tfrac{1}{2} + z)\Gamma(s - \tfrac{1}{2} -
  z)}{\Gamma(s)\Gamma(s+k-1)} \\
  \mathcal{Z}(s,w,z) &= \frac{\zeta(s + w -\frac{1}{2} + z)\zeta(s + w -\frac{1}{2} -
  z)}{\zeta^*(1+2z)}.
\end{align*}
\end{proposition}


\begin{remark}\label{rem:extraremark}
Let's verify that this spectral expansion converges.
Recall Stirling's approximation: for $x,y \in \mathbb{R}$,
\index{Stirling's approximation}
\begin{equation}
  \gamma(x+iy) \sim (1+|y|)^{x-\frac{1}{2}}e^{-\frac{\pi}{2}|y|}
\end{equation}
as $y \to \pm\infty$ with $x$ bounded.
For vertical strips in $s$ and $z$,
\begin{equation}
  G(s,z) \sim P(s,z) e^{-\frac{\pi}{2}(2\max(|s|,|z|)-2|s|)},
\end{equation}
where $P(s,z)$ has at most polynomial growth in $s$ and $z$.
When $k$ is a full-integer, Watson's triple product formula (given in Theorem 3
of~\cite{watson2008rankin}) shows that
\begin{equation}
  \rho_j(1)\langle f\overline{g} \Im(\cdot)^k,\overline{\mu_j}\rangle, \quad \text{and}
  \quad  \rho_j(1)\langle T_{-1}(f\overline{g}) \Im(\cdot)^k,\overline{\mu_j}\rangle
\end{equation}
has at most polynomial growth in $|t_j|$.
When $k$ is a half-integer, K\i{}ral's bound (given in Proposition~13
of~\cite{mehmet2015}) also proves polynomial growth, albeit of a higher degree.
Through direct computation with the associated Rankin--Selberg $L$-function, % chktex 8
the same can be said about
\begin{equation}
  \langle \mathcal{V}_{f,\overline{g} }, E(\cdot,\tfrac{1}{2}+z)\rangle / \zeta^*(1+2z).
\end{equation}
Both~\eqref{line:1spectralexp} and~\eqref{line:2spectralexp} converge uniformly on
vertical strips in $t_j$ and have at most polynomial growth in $s$.
\end{remark}


We will now specialize to $w = 0$ and analyze the meromorphic properties of $Z(s, 0,
f\times \overline{g})$.
This very naturally breaks into two parts: the contribution from the discrete spectrum (in
line~\eqref{line:1spectralexp}) and the contribution from the continuous spectrum (in
line~\eqref{line:2spectralexp}).


\subsubsection{Meromorphic continuation of $Z(s, 0, f\times \overline{g})$: discrete spectrum}


Examination of line~\eqref{line:1spectralexp}, the contribution from the discrete
spectrum, reveals that the poles come only from $G(s, it_j)$.
There are apparent poles when $s = \tfrac{1}{2} \pm it_j - n$ for $n \in \mathbb{Z}_{\geq
0}$.
Interestingly, the first set of apparent poles at $s = \frac{1}{2} \pm it_j$ do not
actually occur.


\begin{lemma}\label{lem:Litj_equals_zero}
  For even Maass forms $\mu_j$, we have $L(-2n \pm it_j, \mu_j) = 0$ for $n \in
  \mathbb{Z}_{\geq 0}$.
\end{lemma}


\begin{proof}
  The completed $L$-function associated to a Maass form $\mu_j$ is given by
  \begin{equation} \label{eq:feq}
    \Lambda_j(s) = \pi^{-s} \Gamma\left( \tfrac{s + \epsilon + it_j}{2}
    \right)\Gamma\left( \tfrac{s + \epsilon - it_j}{2} \right) L(s, \mu_j) = (-1)^\epsilon
    \Lambda_j(1-s),
  \end{equation}
  as in~\cite[Sec 3.13]{Goldfeld2006automorphic}, where $\epsilon = 0$ if the Maass form
  $\mu_j$ is even and $1$ if it is odd.
  In the case of an even Maass form, the functional equation is of shape
  \begin{equation}
    \Lambda_j(s) = \pi^{-s} \Gamma\left( \tfrac{s + it_j}{2} \right)\Gamma\left( \tfrac{s
    - it_j}{2} \right) L(s, \mu_j) = \Lambda_j(1-s).
  \end{equation}
  The completed $L$-function is entire.
  Thus $L(-2n \pm it_j, \mu_j)$ must be trivial zeroes to cancel the apparent poles at $s
  =-2n \pm it_j$ from the Gamma functions.
\end{proof}


It turns out that there are no poles appearing from odd Maass forms due to the symmetry of
the above-diagonal and below-diagonal terms in $\mathcal{V}_{f, \overline{g}}$.


\begin{lemma}\label{lem:oddorthogonaltoeven}
  Suppose $f$ and $g$ are weight $k$ cusp forms, as above. For odd Maass forms $\mu_j$, we
  have $\langle \mathcal{V}_{f,\overline{g}}, \mu_j \rangle = 0$.
\end{lemma}


\begin{proof}
  Recall that $\mathcal{V}_{f,g} :=y^k(f\overline{g}+T_{-1}(f\overline{g}))$, so clearly
  $T_{-1}\mathcal{V}_{f,\overline{g} }=\mathcal{V}_{f,\overline{g} }$.
  Recall also that $T_{-1} \mu_j = -\mu_j$ for odd Maass forms $\mu_j$ (in fact, this is
  the defining characteristic of an odd Maass form).
  Since $T_{-1}$ is a self-adjoint operator with respect to the Petersson inner product we
  have that
  \begin{equation}
    \langle \mathcal{V}_{f,\overline{g} },\mu_j \rangle
    = \langle T_{-1}\mathcal{V}_{f,\overline{g} },\mu_j \rangle
    = \langle \mathcal{V}_{f,\overline{g} },T_{-1}\mu_j \rangle
    = - \langle \mathcal{V}_{f,\overline{g} },\mu_j \rangle.
  \end{equation}
  Thus $\langle \mathcal{V}_{f,\overline{g} },\mu_j \rangle  =0$.
\end{proof}


In the special case when $f = g$, it is possible to show that odd Maass forms $\mu_j$ do
not contribute poles in \emph{either} the above-diagonal or below-diagonal terms in
$\mathcal{V}_{f, \overline{f}}$.
We record this observation as a corollary to the above lines of thought, even though it is
not necessary for the applications in this thesis.


\begin{corollary*}
  Suppose $f$ is a full-integral weight cuspidal Hecke eigenform, not necessarily with
  real coefficients.
  Then for odd Maass forms $\mu_j$, we have $\langle \lvert f \rvert^2 \Im(\cdot)^k, \mu_j
  \rangle = 0$.
  Similarly, we have $\langle f^2 \Im(\cdot)^k, \mu_j \rangle = 0$.
\end{corollary*}


\begin{proof}
  We sketch the proof. From Watson's well-known triple product
  formula~\cite{watson2008rankin}, we have
  \begin{equation}
    \langle \lvert f \rvert^2 \Im(\cdot)^k, \mu_j \rangle^2 \sim \frac{L(\frac{1}{2},
    f\times \overline{f} \times \mu_j)}{L(1, f, \text{Ad}) L(1, \overline{f}, \text{Ad})
    L(1, \mu_j, \text{Ad})}
  \end{equation}
  up to multiplication by a nonzero constant coming from the missing Gamma factors.
  The $L$-functions in the denominator are all nonzero, and the numerator factors as
  \begin{equation}
    L(\tfrac{1}{2}, f \times \overline{f} \times \mu_j) = L(\tfrac{1}{2}, \text{Ad}^2 f
    \times \mu_j) L(\tfrac{1}{2}, \mu_j).
  \end{equation}
  Since $\mu_j$ is odd, $L(\tfrac{1}{2}, \mu_j) = 0$ by the functional equation for odd
  Maass forms, given in~\eqref{eq:feq}.


  Applying Watson's triple product to $\langle f^2 \Im(\cdot)^k, \mu_j \rangle$ yields the
  numerator
  \begin{equation}
    L(\tfrac{1}{2}, \text{Sym}^2 f \times \mu_j) L(\tfrac{1}{2}, \mu_j),
  \end{equation}
  which is zero for the same reason.
\end{proof}



Lemma~\ref{lem:oddorthogonaltoeven} guarantees that the only Maass forms appearing in
line~\eqref{line:1spectralexp} are even. The first set of apparent poles from even Maass
forms appear at $s = \frac{1}{2} \pm it_j$ and occur as simple poles of the Gamma
functions in the numerator of $G(s, t_j)$. They come multiplied by the value of $L(it_j,
\mu_j)$, which by Lemma~\ref{lem:Litj_equals_zero} is zero.


In summary, $D(s, S_f \times S_g)$ has no poles at $s = \frac{1}{2} \pm it_j$.
The next set of apparent poles are at $s = -\frac{1}{2} \pm it_j$, appearing at the next
set of simple poles of the Gamma functions in the numerator.
Unlike the previous poles, these do not coincide with trivial zeroes of the $L$-function.
We have poles of the discrete spectrum at $s = -\frac{1}{2} \pm it_j$.



\subsubsection{Meromorphic continuation of $Z(s, 0, f\times \overline{g})$: continuous spectrum}


Examination of the line~\eqref{line:2spectralexp}, the contribution from the continuous
spectrum, is substantially more involved than the discrete spectrum.
It is here where the most remarkable cancellation occurs.
For ease of reference, we repeat this line, the part we call the continuous spectrum:
\begin{equation}\label{eq:line2-repeat}
  \frac{(4\pi)^{k}}{4\pi i}\int_{(0)} G(s, z) \mathcal{Z}(s,w,z) \langle
  \mathcal{V}_{f,\overline{g} },E(\cdot,\tfrac{1}{2}-\overline{z})\rangle \,dz
  \tag{\ref{line:2spectralexp}}
\end{equation}\noeqref{eq:line2-repeat} % <-- Necessary for automatic tagging to continue working
where $G(s, z)$ and $\mathcal{Z} (s,w,z)$ are the collected $\Gamma$ and $\zeta$ factors
\begin{align}
  G(s, z) = \frac{\Gamma(s - \tfrac{1}{2} + z)\Gamma(s - \tfrac{1}{2} -
  z)}{\Gamma(s)\Gamma(s+k-1)},
  \; \mathcal{Z}(s,w,z) = \frac{\zeta(s + w -\frac{1}{2} + z)\zeta(s + w -\frac{1}{2} -
  z)}{\zeta^*(1+2z)}.
\end{align}
The rightmost pole seems to occur from the pair of zeta functions in the numerator,
occurring when $s + w - \frac{1}{2} \pm z = 1$. We must disentangle $s$ and $w$ from $z$
in order to understand these poles.


Line~\eqref{line:2spectralexp} is analytic for $\Re (s+w) > \frac{3}{2}$, $\Re s >
\tfrac{1}{2}$.
As we will shortly set $w = 0$, we treat the boundary $\Re (s+w) > \frac{3}{2}$.
For $s$ with $\Re s \in (\frac{3}{2} - \Re w, \frac{3}{2} - \Re w + \epsilon)$ for some
very small $\epsilon$, we want to shift the contour of integration, avoiding poles coming
from the $\zeta^*(1 -2z)$ appearing in the denominator of the Fourier expansion of
$E(\cdot,\frac{1}{2}+\overline{z})$ (described in~\eqref{eq:sums:E_Fourier}).


We shift the $z$-contour to the right while staying within the zero-free region of
$\zeta(1 - 2z)$.
By an abuse of notation, we denote this shift here by $\Re z = \epsilon$ and let
$\epsilon$ in this context actually refer to the real value of the $z$-contour at the
relevant imaginary value.
This argument can be made completely rigorous, cf.~\cite[p. 481-483]{HoffsteinHulse13}.


We perform this shift in order to guarantee that the two poles in $z$ coming from $\zeta(s
+ w - \tfrac{1}{2} \pm z)$, occurring at $\pm z = \tfrac{3}{2} - s - w$, have different
real parts; simultaneously, we pass the pole with more positive real part, occurring at
$z = s + w - \frac{3}{2}$ from $\zeta(s + w - \tfrac{1}{2} - z)$.
By the residue theorem,
\begin{align}\label{eq:FEfirstpart}
  &\frac{(4\pi)^k}{4\pi i} \int_{(0)}G(s,z) \mathcal{Z}(s,w,z) \langle
  \mathcal{V}_{f,\overline{g} }, E\rangle  \ dz \\
  &= \frac{(4\pi)^k}{4\pi i}\int_{(\epsilon)} G \mathcal{Z} \langle
  \mathcal{V}_{f,\overline{g} }, E \rangle \  dz - \frac{(4\pi)^k}{2}\Res_{z = s + w -
  \frac{3}{2}} G \mathcal{Z} \langle \mathcal{V}_{f,\overline{g} }, E\rangle \nonumber,
\end{align}
where the above residue is found to be
\begin{equation}\label{eq:FEfirstresidue}
  -\frac{\zeta(2s + 2w - 2)\Gamma(2s + w -
  2)\Gamma(1-w)}{\zeta^*(2s+2w-2)\Gamma(s)\Gamma(s + k - 1)}\langle
\mathcal{V}_{f,\overline{g} }, E(\cdot, 2-\overline{s}-\overline{w})\rangle.
\end{equation}
The residue is analytic in $s$ for $\Re s \in (1 - \Re w, \tfrac{3}{2} - \Re w +
\epsilon)$, and has an easily understood meromorphic continuation to the whole plane.
Notice also that the shifted contour integral has no poles in $s$ for $\Re s \in
(\tfrac{3}{2} - \Re w - \epsilon, \tfrac{3}{2} - \Re w + \epsilon)$, so we have found an
analytic (not meromorphic!) continuation in $s$ of Line~\eqref{line:2spectralexp} past the
first apparent pole along $\Re s = \frac{3}{2} - \Re w$.



For $s$ with $\Re s \in (\frac{3}{2} - \Re w - \epsilon, \frac{3}{2} - \Re w)$, we shift
the contour of integration back to $\Re z = 0$.
Since this passes a pole, we pick up a residue.
But notice that this is the residue at the \emph{other} pole, $\tfrac{3}{2} - s - w$,
coming from $\zeta(s + w - \tfrac{1}{2} + z)$,
\begin{align}\label{eq:FEsecondpart}
  &\frac{(4\pi)^k}{4\pi i} \int_{(\epsilon)}G(s, w, z) \mathcal{Z}(s,w,z) \langle
  \mathcal{V}_{f,\overline{g} }, E\rangle dz  \\
  =&\frac{(4\pi)^k}{4\pi i}\int_{(0)} G \mathcal{Z} \langle \mathcal{V}_{f,\overline{g} },
  E \rangle dz + \frac{(4\pi)^k}{2}\Res_{z = \frac{3}{2} - s - w} G \mathcal{Z}  \langle
  \mathcal{V}_{f,\overline{g} }, E\rangle \nonumber.
\end{align}
By using the functional equations of the Eisenstein series and zeta functions, one can
check that
\begin{equation}
  \Res_{z = \frac{3}{2} - s - w} G \mathcal{Z} \langle \mathcal{V}_{f,\overline{g} },
  E\rangle = - \Res_{z = s + w -  \frac{3}{2}} G \mathcal{Z}
  \langle\mathcal{V}_{f,\overline{g} }, E\rangle,
\end{equation}
so the two residues combine together, and have well-understood meromorphic continuations
to the whole plane.
The shifted integral has clear meromorphic continuation until the next apparent poles at
$\Re s = \tfrac{1}{2}$ coming from the Gamma functions $\Gamma(s - \tfrac{1}{2} \pm z)$ in
the integrand.
Thus~\eqref{line:2spectralexp}, originally defined for $\Re s > \frac{3}{2} - \Re w$, has
meromorphic continuation for $\frac{1}{2} < \Re s < \frac{3}{2} - \Re w$ given by
\begin{equation}
  \frac{(4\pi)^k}{4\pi i}\int_{(0)}G \mathcal{Z} \langle \mathcal{V}_{f,\overline{g} },
  E\rangle dz + (4\pi)^k \Res_{z = \frac{3}{2} - s - w} G \mathcal{Z} \langle
\mathcal{V}_{f,\overline{g} }, E\rangle.
\end{equation}


A very similar argument works to extend the meromorphic continuation in $s$ of the contour
integral past the next apparent poles at $\Re s  = \frac{1}{2}$ from the Gamma functions,
leading to a meromorphic continuation in the region $-\frac{1}{2}< \Re s < \frac{1}{2}$
given by
\begin{align}\label{eq:fullcontinuation}
  &\frac{(4\pi)^k}{4\pi i}\int_{(0)}G(s, w, z)\mathcal{Z}(s,w,z)\langle
\mathcal{V}_{f,\overline{g} }, E(\cdot, \tfrac{1}{2} - \overline{z})\rangle dz \\
  &\quad+ (4\pi)^k \Res_{z = \frac{3}{2} - s - w}  G(s, w, z) \mathcal{Z}(s,w,z) \langle
  \mathcal{V}_{f,\overline{g} }, E(\cdot, \tfrac{1}{2} - \overline{z})\rangle
  \label{line:firstresidual} \\
  &\quad+ (4\pi)^k \Res_{z = \frac{1}{2} - s} G(s, w, z) \mathcal{Z}(s,w,z) \langle
\mathcal{V}_{f,\overline{g} }, E(\cdot, \tfrac{1}{2} -\overline{z})\rangle.
\label{line:secondresidual}
\end{align}


We iterate this argument, as in Section~4 of~\cite[p. 481-483]{HoffsteinHulse13}.
Somewhat more specifically, when $\Re(s)$ approaches a negative half-integer,
$\frac{1}{2}-n$, we can shift the line of integration for $z$ right past the pole due to
$G(s,z)$ at $z=s-\frac{1}{2}+n$, move $s$ left past the line $\Re(s)=\frac{1}{2}-n$ and
then shift the line of integration for $z$ left, back to zero and over the pole at
$z=\frac{1}{2}-s-n$.
This gives meromorphic continuation of~\eqref{line:2spectralexp} arbitrarily far to the
left, accumulating an additional pair of residual terms each time $\Re s$ passes a
half-integer, and of a similar form to the first residual term coming from $G(s,z)$,
appearing in~\eqref{line:secondresidual}.


\index{residual terms}
We now specialize to $w = 0$.
It is advantageous to codify some terminology for these additional terms appearing in the
meromorphic continuation of the integral, i.e.\ terms like~\eqref{line:firstresidual}
and~\eqref{line:secondresidual}.
We call these terms \emph{residual} terms, as they come from residues in the $z$ variable;
these are distinct from residues in $s$, as these residual terms are functions in $s$.
We also introduce a notation for these residual terms.
Substituting $w = 0$ into~\eqref{line:firstresidual}, we get the residual term
\begin{equation}\label{eq:residual_0}
  \rho_{\frac{3}{2}}(s) = \frac{(4\pi)^k \zeta(2s - 2)\Gamma(2s - 2)}{\Gamma(s) \Gamma(s +
  k - 1) \zeta^*(2s - 2)} \langle \mathcal{V}_{f, \overline{g}}, \overline{E(\cdot, 2 -
s)} \rangle.
\end{equation}
This residual term is distinguished as the only residual term appearing as a residue of
the zeta functions.


The remaining residual terms all come from residues of Gamma functions, and all have a
similar form.
For each $m \geq 1$, there is a residual term $\rho_{\frac{3}{2} - m}(s)$ appearing for
$\Re s < \tfrac{3}{2} - m$, appearing from an apparent pole of Gamma functions in $G(s,z)$
at $z = \tfrac{3}{2} - m - s$, given by
\begin{equation}\label{eq:residual_m}
  \begin{split}
    \rho_{\frac{3}{2} - m}(s) = &\frac{(-1)^{m-1}(4\pi)^k \zeta(1 - m) \zeta(2s + m - 2)
    \Gamma(2s + m - 2)}{\Gamma(m) \Gamma(s) \Gamma(s + k - 1) \zeta^*(4 - 2s - 2m)} \times
    \\
    &\quad\langle \mathcal{V}_{f, \overline{g}}, \overline{E(\cdot, s + m - 1)}\rangle.
  \end{split}
\end{equation}


We summarize these computations with the following proposition.
\begin{proposition}
  The continuous spectrum component of $Z(s, 0, f\times \overline{g})$, as given
  by~\eqref{line:2spectralexp} in Proposition~\ref{prop:spectralexpansionfull}, has
  meromorphic continuation to the complex plane.
  Further, the meromorphic continuation can be written explicitly as
  \begin{equation}
    \frac{(4\pi)^k}{4\pi i} \int_{(0)} G(s,z) \mathcal{Z}(s,0,z) \langle \mathcal{V}_{f,
    \overline{g}}, \overline{E(\cdot, \tfrac{1}{2} - z)}\rangle dz + \sum_{\mathclap{0
    \leq m < \frac{3}{2} - \Re s}} \rho_{\frac{3}{2} - m}(s),
  \end{equation}
  where each residual term $\rho_{\frac{3}{2} - m}(s)$ is given by~\eqref{eq:residual_0}
  (in the case that $m = 0$) or~\eqref{eq:residual_m} (when $m \geq 1$), and
  $\rho_{\frac{3}{2} - m}$ appears only when $\Re s < \tfrac{3}{2} - m$.
\end{proposition}



\subsection{Polar Analysis of $Z(s, 0, f\times \overline{g})$}


Comparing the meromorphic continuations of the discrete spectrum component and continuous
spectrum component of $Z(s, 0, f\times \overline{g})$ reveals that the rightmost pole of
$Z(s, 0, f\times \overline{g})$ occurs in the first residual term,
$\rho_{\frac{3}{2}}(s)$, appearing in~\eqref{eq:residual_0}.
The pole occurs at $s = 1$, caused by the Eisenstein series.
By comparison, the discrete spectrum component is analytic for $\Re s > -\tfrac{1}{2} \pm
it_j$, and the rest of the continuous spectrum is analytic for $\Re s > \tfrac{1}{2}$.


The residue at this pole is given by
\begin{equation}\label{eq:Zresidue}
  \Res_{s = 1} \rho_{\frac{3}{2}}(s) = \Res_{s = 1} \frac{(4\pi)^k \zeta(2s - 2)\Gamma(2s
  - 2)}{\zeta^*(2s-2)\Gamma(s)\Gamma(s + k - 1)}\langle \mathcal{V}_{f,\overline{g} },
E(\cdot, 2-\overline{s})\rangle.
\end{equation}
Expand $\zeta^*(2s-2) = \zeta(2s-2) \Gamma(s-1) \pi^{1-s}$ in the denominator, cancel the
two zeta functions, and use the Gamma duplication identity
\begin{equation}
  \frac{\Gamma(2z)}{\Gamma(z)} = \frac{\Gamma(z + \tfrac{1}{2}) 2^{2z - 1}}{\sqrt \pi}
\end{equation}
with $z = s-1$ to rewrite the residue as
\begin{align}
  \Res_{s = 1} \rho_{\frac{3}{2}}(s) &= \Res_{s = 1} \frac{\Gamma(s - \tfrac{1}{2})
  2^{2s-3}}{\sqrt \pi} \frac{\pi^{s-1}}{\Gamma(s)} \frac{(4\pi)^k}{\Gamma(s + k - 1)}
  \langle \mathcal{V}_{f, \overline{g}}, E(\cdot, 2 - \overline{s}) \rangle \\
  &= -\frac{(4\pi)^k}{\Gamma(k)} \Res_{s = 1} \langle f \overline{g} \Im(\cdot)^k,
E(\cdot, \overline{s})\rangle.
\end{align}
Through the relationship between the Eisenstein series and the
Rankin--Selberg $L$-function (cf.~\S\ref{ssec:rankinselberg_lfunction}), % chktex 8
we can rewrite this as
\begin{equation}\label{eq:first_residual_pole_1}
  \Res_{s = 1} \rho_{\frac{3}{2}}(s) = -\Res_{s = 1} \frac{L(s, f\times
  \overline{g})}{\zeta(2)}.
\end{equation}


The next pole of $Z(s, 0, f\times \overline{g})$ also comes from the first residual term
$\rho_{\frac{3}{2}}(s)$, occurring at $s = \tfrac{1}{2}$ from the Gamma function in the
numerator.
Similar to the computation of the residue at $s = 1$, we expand $\zeta^*$, cancel the two
zeta functions, and apply the Gamma duplication identity to recognize the residue as
\begin{align}
  \Res_{s = \frac{1}{2}} \rho_{\frac{3}{2}}(s) &= \Res_{s = \frac{1}{2}} \frac{(4\pi)^{s +
  k - 1}}{2\sqrt \pi} \frac{\Gamma(s - \tfrac{1}{2})}{\Gamma(s) \Gamma(s + k - 1)} \langle
  \mathcal{V}_{f, \overline{g}}, E(\cdot, 2 - \overline{s}) \rangle \\
  &= \frac{1}{2 \pi }\frac{(4\pi)^{k - \frac{1}{2}}}{\Gamma(k - \frac{1}{2})}\langle
\mathcal{V}_{f,\overline{g} }, E(\cdot, \tfrac{3}{2}) \rangle.
\end{align}
We rewrite this as a special value of the Rankin--Selberg $L$-function, % chktex 8
\begin{equation}\label{eq:first_residual_pole_half}
  \Res_{s = \frac{1}{2}} \rho_{\frac{3}{2}}(s) = \frac{1}{2 \pi}\frac{(k -
  \frac{1}{2})}{4\pi} \frac{(4\pi)^{k + \frac{1}{2}}}{\Gamma(k + \frac{1}{2})}\langle
\mathcal{V}_{f,\overline{g} }, E(\cdot, \tfrac{3}{2}) \rangle = \frac{(k -
\frac{1}{2})}{4\pi^2} \frac{L(\frac{3}{2}, f\times \overline{g} )}{\zeta(3)}.
\end{equation}


Returning to the rest of the meromorphic continuation of the continuous spectrum, let us
examine the second residual term $\rho_{\frac{3}{2} - 1}(s) = \rho_{\frac{1}{2}}(s)$,
which appears as part of the meromorphic continuation only for $\Re s < \tfrac{1}{2}$.
We simplify the expression in~\eqref{eq:residual_m}, with $m = 1$,
\begin{equation}
  \rho_{\frac{1}{2}}(s) = \frac{(4\pi)^k \zeta(0)}{\Gamma(s + k - 1)} \frac{\zeta(2s
  -1)}{\zeta^*(2 - 2s)} \frac{\Gamma(2s - 1)}{\Gamma(s)} \langle
\mathcal{V}_{f,\overline{g}}, E(\cdot, \overline{s})\rangle.
\end{equation}
By using the functional equation for $\zeta^*(2 - 2s)$, the Gamma duplication formula, and
recognizing the Eisenstein series inner product as a sum of
two Rankin--Selberg $L$-functions, this simplifies to % chktex 8
\begin{equation}\label{eq:secondresidualsimple}
  \rho_{\frac{1}{2}}(s) = -\frac{L(s, f\times \overline{g})}{\zeta(2s)}.
\end{equation}
We now recognize that $\rho_{\frac{1}{2}}(s)$ has poles at zeroes of $\zeta(2s)$.



\begin{remark}
  At first glance, it would appear that $\rho_{\frac{1}{2}}(s)$ also has a pole at $s=1$,
  coming from the pole of the Rankin--Selberg convolution, % chktex 8
  and that this term therefore contributes a pole to $Z(s, 0, f\times \overline{g})$ at
  $s=1$.
  However, the term $\rho_{\frac{1}{2}}(s)$ does not appear as part of the meromorphic
  continuation of $Z(s, 0, f\times \overline{g})$ except when $\Re s < \frac{1}{2}$, so
  this term does not contribute a pole at $s = 1$.
\end{remark}


More generally, for each $m \geq 1$, the residual term $\rho_{\frac{3}{2} - m}(s)$, which
appears for $\Re s < \frac{3}{2} - m$, also contributes poles.
As in~\eqref{eq:secondresidualsimple}, the Eisenstein series in $\rho_{\frac{3}{2} -
m}(s)$ introduces poles at $s = \frac{\gamma}{2} - m + 1$ for each nontrivial zero
$\gamma$ of $\zeta(s)$, in addition to potential poles at negative integers and
half-integers from the Gamma function in the numerator.


Poles appearing in the discrete spectrum component do not exhibit the same properties of
cancellation, aside from those noted in Lemmas~\ref{lem:Litj_equals_zero}
and~\ref{lem:oddorthogonaltoeven}.
Analysis of the Gamma functions in the discrete component,
in~\eqref{line:1spectralexp}, shows that there are potential simple poles at $s =
\frac{1}{2} \pm it_j - n$ for $n \in \mathbb{Z}_{\geq 0}$.
The Lemmas~\ref{lem:Litj_equals_zero} and~\ref{lem:oddorthogonaltoeven} show that those
poles occurring at $s = \frac{1}{2} \pm it_j - n$ with $n$ even are cancelled by trivial
zeroes.
Together, these indicate that there are potential poles at $s = \frac{1}{2} \pm it_j - n$
for each odd, positive integer $n$.



\subsection{Analytic Behavior of $W(s; f, \overline{g})$}
\index{Ws@$W(s; f, f)$}


Recall that
\begin{equation}
  W(s; f, \overline{g}) = \frac{L(s, f\times \overline{g})}{\zeta(2s)} + Z(s, 0, f\times \overline{g}).
\end{equation}
As noted in~\eqref{eq:first_residual_pole_1}, the leading pole of $L(s, f\times
\overline{g}) \zeta(2s)^{-1}$ perfectly cancels with the leading pole of $Z(s, 0, f\times
\overline{g})$.
Therefore $W(s, f, \overline{g})$ is analytic for $\Re s > \tfrac{1}{2}$ and has a pole at
$s = \tfrac{1}{2}$, identified in~\eqref{eq:first_residual_pole_half}.


Further, the second residual term, $\rho_{\frac{1}{2}}(s)$, was shown to be exactly $-L(s,
f\times \overline{g}) \zeta(2s)^{-1}$ in~\eqref{eq:secondresidualsimple}, and appears for
$\Re s < \frac{1}{2}$.
Therefore the Rankin--Selberg $L$-function % chktex 8
$L(s, f\times \overline{g})\zeta(2s)^{-1}$ perfectly cancels with $\rho_{\frac{1}{2}}(s)$
for $\Re s < \frac{1}{2}$.
For $\Re s < \frac{1}{2}$, the analytic behavior of $W(s; f, \overline{g})$ is determined
entirely by the analytic behavior of $Z(s, 0, f\times \overline{g})$ (and omitting
$\rho_{\frac{1}{2}}(s)$).


Therefore $W(s; f, \overline{g})$ has meromorphic continuation to $\mathbb{C}$.
For $\Re s > - \tfrac{1}{2}$, the only possible poles of $W(s; f, \overline{g})$ are at $s
= \tfrac{1}{2}$, coming from~\eqref{eq:first_residual_pole_half}, and those at $s = -
\tfrac{1}{2} \pm it_j$ coming from exceptional eigenvalues of the discrete spectrum.
(There are no exceptional eigenvalues on $\SL(2, \mathbb{Z})$).
Collecting the analytic data, we have proved the following.



\begin{theorem}\label{thm:Wsfgmero}
  Let $f,g$ be two holomorphic cusp forms on $\SL(2, \mathbb{Z})$.
  Maintaining the same notation as above, the function $W(s; f,\overline{g})$ has a
  meromorphic continuation to $\mathbb{C}$ given by the Rankin--Selberg % chktex 8
  $L$-function~\eqref{eq:Lsfgbar_equals_eisenstein}
  and spectral decomposition in Proposition~\ref{prop:spectralexpansionfull},
  with potential poles at $s$ with $\Re s \leq \tfrac{1}{2}$ and $s \in
  \mathbb{Z}\cup(\mathbb{Z} + \tfrac{1}{2})\cup\mathfrak{S}\cup\mathfrak{Z}$, where
  $\mathfrak{Z}$ denotes the set of shifted zeta-zeroes $\{-1 + \frac{\gamma}{2} - n: n
  \in \mathbb{Z}_{\geq 0}\}$, and $\mathfrak{S}$ denotes the set of shifted discrete types
  $\{-\tfrac{1}{2} \pm it_j - n: n \in \mathbb{Z}_{\geq 0}, n \; \text{odd}\; \}$.

  The leading pole is at $s = \frac{1}{2}$ and
    \begin{equation}
      \Res_{s = \frac{1}{2}} W(s; f, \overline{g}) = \frac{(k - \frac{1}{2})}{4\pi^2}
      \frac{L(\tfrac{3}{2}, f\times \overline{g})}{\zeta(3)}.
    \end{equation}
\end{theorem}



\subsection{Complete Meromorphic Continuation of $D(s, S_f\times \overline{S_{g}})$}


With Theorem~\ref{thm:Wsfgmero} and the decomposition from
Proposition~\ref{prop:SfSg_decomposition}, we can quickly give the meromorphic
continuation of the Dirichlet series $D(s, S_f \times \overline{S_g})$.
In particular, by Proposition~\ref{prop:SfSg_decomposition}, we know that
\begin{equation}
  D(s, S_f \times \overline{S_g}) = W(s; f, \overline{g}) + \frac{1}{2\pi i}
  \int_{(\gamma)} W(s-z; f, \overline{g}) \zeta(z)\frac{\Gamma(z) \Gamma(s - z + k -
  1)}{\Gamma(s + k - 1)} \; dz,
\end{equation}
where initially $\Re s$ is large and $\gamma \in (1, \Re(s) - 1)$.
Notice that $W(s; f, \overline{g})$ can also be written as a single Dirichlet series as
\begin{align}
  W(s; f, \overline{g}) &= \frac{L(s, f\times \overline{g})}{\zeta(2s)} + Z(s, 0, f\times
  \overline{g}) \\
  &= \sum_{n,h \geq 1} \frac{a(n) \overline{b(n)} + a(n)\overline{b(n-h)} +
  a(n-h)\overline{b(n)}}{n^{s + k - 1}} \\
  &= \sum_{\substack{n \geq 1 \\ h \geq 0}} \frac{a(n)\overline{b(n-h)} +
  a(n-h)\overline{b(n)} - a(n)\overline{b(n)}}{n^{s + k - 1}} \\
  &= \sum_{n \geq 1} \frac{a(n) \overline{S_g(n)} + S_f(n) \overline{b(n)}}{n^{s + k - 1}}
  =: \sum_{n \geq 1} \frac{w(n)}{n^{s + k - 1}}.
\end{align}
We denote the $n$th coefficient of this Dirichlet series of $w(n)$.
As $a(n) \ll n^{\frac{k-1}{2} + \epsilon}$ and $S_f(n) \ll n^{\frac{k-1}{2} +
\frac{1}{3}}$, we know that $w(n) \ll n^{k-1 + \frac{1}{3} + \epsilon}$.
Thus $W(s; f, \overline{g})$ converges absolutely for $\Re s > \frac{4}{3}$.

Consider $D(s, S_f \times \overline{S_g})$ for $\Re s > 4$ and $\gamma = 2$ initially, so
that both $W(s; f, \overline{g})$ and $W(s-z; f, \overline{g})$ are absolutely convergent.
Shifting the line of $z$ integration to $-M - \frac{1}{2}$ for some positive integer $M$
passes several poles occurring when $z = 1$ (from $\zeta(z)$) or $z = -j$ with $j \in
\mathbb{Z}_{\geq 0}$ (from $\Gamma(z)$).
Notice that in this region, $W(s-z; f, \overline{g})$ converges absolutely, $\zeta(z)$ has
at most polynomial growth in vertical strips, and the Gamma functions have exponential
decay for any fixed $s$.
By Cauchy's Theorem, we have
\begin{align}
  &D(s, S_f \times \overline{S_g}) =\\
  &= W(s; f, \overline{g}) + \sum_{-M \leq j \leq 1} \Res_{z = j} W(s-z; f, \overline{g})
  \zeta(z)\frac{\Gamma(z) \Gamma(s - z + k - 1)}{\Gamma(s + k - 1)} \\
  &\quad + \frac{1}{2\pi i} \int_{(-M - \frac{1}{2})} W(s-z; f, \overline{g})
  \zeta(z)\frac{\Gamma(z) \Gamma(s - z + k - 1)}{\Gamma(s + k - 1)} \; dz \\
  &= W(s; f, \overline{g}) + \frac{W(s-1; f, \overline{g})}{s+k-2} + \sum_{j = 0}^{M}
  \frac{(-1)^j}{j!} W(s + j; f, \overline{g}) \zeta(-j) \frac{\Gamma(s + j + k -
  1)}{\Gamma(s + k - 1)} \\
  &\quad + \frac{1}{2\pi i} \int_{(-M - \frac{1}{2})} W(s-z; f, \overline{g})
  \zeta(z)\frac{\Gamma(z) \Gamma(s - z + k - 1)}{\Gamma(s + k - 1)} \; dz.
\end{align}
Each of the residues gives an expression containing $W(s; f, \overline{g})$ with clear
meromorphic continuation to the plane.
The remaining shifted integral contains $W(s-z; f, \overline{g})$ in its integrand, with
$\Re z = -M-\tfrac{1}{2}$.
Therefore $\Re s-z = \Re s + M + \tfrac{1}{2}$, and so $W(s-z; f, \overline{g})$ is
absolutely convergent for $\Re s > -M + \tfrac{5}{2}$.
As $\zeta(z)$ has only polynomial growth and $\Gamma(z)\Gamma(s-z+k-1)$ has exponential
decay in vertical strips, we see that the integral represents an analytic function of $s$
for $\Re s > -M + \tfrac{5}{2}$.
Therefore the entire right hand side has meromorphic continuation to the region $\Re s >
-M + \tfrac{5}{2}$.
As $M$ is arbitrary, we have proved the following, which we record as a corollary to
Theorem~\ref{thm:Wsfgmero}.


\begin{corollary}\label{cor:DsSfSg_has_meromorphic}
  The Dirichlet series $D(s, S_f \times \overline{S_g})$ has meromorphic continuation to
  the entire complex plane.
\end{corollary}


\begin{remark}
  Very similar work gives the meromorphic continuation for $D(s, S_f \times
  \overline{S_g})$, mainly replacing $\overline{g}$ with $T_{-1}g$ in the above
  formulation.
  This distinction only matters at higher levels when $f$ and $g$ have nontrivial
  nebentypus, and the spectral expansion is modified accordingly.
\end{remark}


Analysis of the exact nature of the poles of $D(s, S_f \times \overline{S_g})$ can be
performed directly on this presentation of the meromorphic continuation.
In many cases, the leading behavior of integral transforms on $D(s, S_f \times
\overline{S_g})$ will come from $W(s-1; f, \overline{g})/(s+k-2)$, as this term contains
the largest negative shift in $s$.
For later reference, it will be useful to view $D(s, S_f \times \overline{S_g})$ in this
form for clear arithmetic application.
We codify this in the following lemma.
\begin{lemma}\label{lem:D_is_lots_of_W}
  \begin{equation}
    \begin{split}
    D(s, S_f \times \overline{S_g}) &= W(s; f, \overline{g})
      + \frac{W(s-1; f, \overline{g})}{s+k-2} \\
    &\quad + \sum_{j = 0}^M \frac{(-1)^j}{j!} W(s+j; f, \overline{g}) \zeta(-j)
      \frac{\Gamma(s + j + k - 1)}{\Gamma(s + k - 1)} \\
    &\quad + \frac{1}{2\pi i} \int_{(-M - \frac{1}{2})} W(s-z; f, \overline{g})
      \zeta(z)\frac{\Gamma(z) \Gamma(s - z + k - 1)}{\Gamma(s + k - 1)} \; dz.
    \end{split}
  \end{equation}
\end{lemma}



\section{Second-Moment Analysis}\label{sec:secondmoment}


It is now necessary to estimate the growth of $D(s, S_f \times \overline{S_g})$ and to use
the analytic properties described above to study the sizes of sums of coefficients of cusp
forms.
It will be necessary to understand the size of growth of $D(s, S_f \times
\overline{S_g})$, but it is relatively straightforward to see that $D(s, S_f \times
\overline{S_g})$ has polynomial growth in vertical strips.


\begin{lemma}\label{lem:DsSfSg_poly_growth}
  For $\sigma < \Re s < \sigma'$ and $s$ uniformly away from poles, there exists some $A$
  such that
  \begin{equation}
    D(s, S_f \times \overline{S_g}) \ll \lvert \Im s \rvert^A.
  \end{equation}
  Therefore $D(s, S_f \times \overline{S_g})$ is of polynomial growth in vertical strips.
\end{lemma}


\begin{proof}
  From Lemma~\ref{lem:D_is_lots_of_W} it is only necessary to study the growth properties
  of $W(s; f, \overline{g})$ and the Mellin-Barnes integral transform of $W(s; f,
  \overline{g})$.


  We first handle $W(s; f, \overline{g})$.
  The diagonal component of $W(s; f, \overline{g})$ is just the Rankin--Selberg % chktex 8
  $L$-function $L(s, f\times \overline{g}) \zeta(2s)^{-1}$, which has polynomial growth in
  vertical strips as a consequence of the Phragm\'{e}n-Lindel\"{o}f convexity principle
  and the functional equation.


  As noted in Remark~\ref{rem:extraremark}, the discrete spectrum and integral term in the
  continuous spectrum each have polynomial growth in vertical strips.
  It remains to consider the possible contribution from the residual terms
  $\rho_{\frac{3}{2}}(s)$ and $\rho_{\frac{3}{2} - m}(s)$.
  These each consist of a product of zeta functions, Gamma functions,
  and Rankin--Selberg % chktex 8
  $L$-functions, and a quick analysis through Stirling's approximation shows that the
  exponential contributions from the Gamma functions all perfectly cancel.
  Therefore these are also of polynomial growth.


  We now handle the Mellin-Barnes transform of $W(s; f, \overline{g})$.
  We actually prove a slightly more general result.


  Let $F(s)$ be a function of polynomial growth in $\lvert \Im s \rvert$ in vertical
  strips containing $\sigma$.
  Then the function
  \begin{equation}
    \frac{1}{2\pi i} \int_{(\sigma)}F(s-z)\zeta(z) \frac{\Gamma(z) \Gamma(s -
    z)}{\Gamma(s)} dz
  \end{equation}
  has at most polynomial growth in $\lvert \Im s \rvert$.
  Indeed, through Stirling's Approximation, the integrand is bounded by
  \begin{equation}
    \lvert \Im s \rvert^A \lvert \Im (s-z) \rvert^B \lvert \Im z \rvert^C
    \exp\bigg(-\frac{\pi}{2}\Big(\lvert \Im z \rvert + \lvert \Im(s-z) \rvert - \lvert \Im
    s \rvert \Big)\bigg).
  \end{equation}
  Therefore, for $\lvert \Im z \rvert > \lvert \Im s \rvert$, the integrand has
  exponential decay and converges rapidly.
  Thus the integral is essentially of an integrand of polynomial growth along an interval
  of length $2 \lvert \Im s \rvert$, leading to an overall polynomial bound in $\lvert \Im
  s \rvert$.
\end{proof}


This is already sufficient for many applications.
Consider the integral transform
\begin{equation}\label{eq:smooth_cutoff_ref}
  \frac{1}{2\pi i} \int_{(\sigma)} D(s, S_f \times \overline{S_g}) X^s \Gamma(s) ds =
  \sum_{n \geq 1} \frac{S_f(n) \overline{S_g(n)}}{n^{k-1}} e^{-n/X},
\end{equation}
as described in \S\ref{sec:cutoff_integrals}.
Initially take $\sigma$ large enough to be in the domain of absolute convergence of $D(s,
S_f \times \overline{S_g})$, say $\sigma \geq 4$.


Through Lemma~\ref{lem:D_is_lots_of_W}, we rewrite~\eqref{eq:smooth_cutoff_ref} as
\begin{equation}\label{eq:smooth_cutoff_in_W}
  \begin{split}
    &\frac{1}{2\pi i} \int_{(4)} \bigg(\frac{1}{2} W(s; f, \overline{g}) + \frac{W(s-1; f,
\overline{g})}{s + k - 2} \bigg) X^s \Gamma(s) ds \\
    &\quad+ \frac{1}{(2\pi i)^2} \int_{(4)} \int_{(-1 + \epsilon)} W(s-z; f, \overline{g})
\zeta(z) \frac{\Gamma(z) \Gamma(s - z + k - 1)}{\Gamma(s + k - 1)}dz X^s \Gamma(s) ds.
  \end{split}
\end{equation}
From the proof and statement of Theorem~\ref{thm:Wsfgmero}, we see that $W(s; f,
\overline{g})$ is analytic in $\Re s > -\frac{1}{2}$ except for poles at $s = \frac{1}{2}$
and at $s = -\frac{1}{2} \pm it_j$.
Therefore when we shift lines of $s$-integration in~\eqref{eq:smooth_cutoff_in_W} to
$\frac{1}{2} + 2\epsilon$ passes a pole at $s = \frac{3}{2}$ from $W(s-1; f,
\overline{g})$, and otherwise no poles.


\begin{remark}
  For general level, we shift lines of $s$-integration to $\frac{1}{2} + \theta +
  2\epsilon$, where $\theta < \frac{7}{64}$ is the best-known progress towards Selberg's
  Eigenvalue Conjecture, as noted above.
\end{remark}


By Lemma~\ref{lem:DsSfSg_poly_growth}, this shift is justified and the resulting integral
converges absolutely.
Therefore
\begin{equation}
  \sum_{n \geq 1} \frac{S_f(n)\overline{S_g(n)}}{n^{k-1}} e^{-n/X} = C X^{\frac{3}{2}} +
  O_{f,g,\epsilon}(X^{\frac{-1}{2}+\theta+\epsilon})
\end{equation}
We can evaluate the residue $C$ using Theorem~\ref{thm:Wsfgmero}.
Note that the same analysis holds on $D(s, S_f \times S_g)$ as well.
In total, we have proved the following theorem.



\begin{theorem}\label{thm:second_moment_Sf}
  Suppose $f$ and $g$ are weight $k$ holomorphic cusp forms on $\SL(2, \mathbb{Z})$.
  For any $\epsilon > 0$,
  \begin{align}
    \sum_{n \geq 1} \frac{S_f(n) \overline{S_g(n)}}{n^{k-1}} e^{-n/X} &= C X^{\frac{3}{2}}
    + O_{f,g,\epsilon} (X^{\frac{1}{2} + \epsilon}) \\
    \sum_{n \geq 1} \frac{S_f(n) S_g(n)}{n^{k-1}} e^{-n/X} &= C' X^{\frac{3}{2}} +
    O_{f,g,\epsilon} (X^{\frac{1}{2} + \epsilon})
  \end{align}
  where
  \begin{equation}
    C = \frac{\Gamma(\frac{3}{2})}{4\pi^2} \frac{L(\frac{3}{2}, f \times
    \overline{g})}{\zeta(3)},
    \qquad C' = \frac{\Gamma(\frac{3}{2})}{4\pi^2} \frac{L(\frac{3}{2}, f \times
    g)}{\zeta(3)}.
  \end{equation}
\end{theorem}



As an immediate corollary, we have the following smoothed analogue of the Classical
Conjecture.
\begin{corollary}
  \begin{equation}
    \sum_{n \geq 1} \frac{\lvert S_f(n) \rvert^2}{n^{k-1}} e^{-n/X} = C X^{\frac{3}{2}} +
    O_{f,\epsilon} (X^{\frac{1}{2} + \epsilon}),
  \end{equation}
  where $C$ is the special value of $L(\frac{3}{2}, f \times \overline{f})
  \Gamma(\frac{3}{2}) (\zeta(3) 4\pi^2)^{-1}$ as above.
\end{corollary}



\section{A General Cancellation Principle}\label{sec:higherlevel}



While the techniques and methodology employed so far should work for general weight and
level, it is not immediately obvious that that the miraculous cancellation that occurs in
the level $1$ case should always occur.
In particular, it is not clear that the continuous spectrum of $Z(s, 0,f \times
\overline{g})$ will always perfectly cancel both the leading pole and potentially
infinitely many poles from the zeta zeroes of $L(s, f\times g)\zeta(2s)^{-1}$.


In the case when $f=g$ are the same cusp form, we can compare our methodology with the
results of Chandrasekharan and Narasimhan to show that the leading polar cancellation does
always occur.
A more detailed analysis using the same methodology as the rest of this chapter would
likely be able to show general cancellation.


\begin{remark}
  In Section~6 of the soon-to-be-published paper~\cite{hkldw}, my collaborators and I
  explicitly show that this cancellation continues to hold for cusp forms $f$ and $g$ on
  $\Gamma_0(N)$ when $N$ is square-free.
  This is much stronger than what is showed in the rest of this section concerning general
  cancellation between the diagonal and off-diagonal sums corresponding to $f \times
  \overline{f}$.
\end{remark}


Suppose $f(z) = \sum a(n)e(nz)$ is a cusp form on $\Gamma_0(N)$ and of weight $k \in
\mathbb{Z}\cup(\mathbb{Z} + \frac{1}{2})$ with $k > 2$.
Theorem~1 of~\cite{chandrasekharan1964mean} gives that
\begin{equation}\label{eq:CN_compare}
  \frac{1}{X} \sum_{n \leq X} \frac{\lvert S_f(n) \rvert^2}{n^{k-1}}  = C X^{\frac{1}{2}}
  + O(\log^2 X).
\end{equation}


Performing the decomposition from Proposition~\ref{prop:SfSg_decomposition} leads us to
again study $Z(s, 0, f\times \overline{f})$ and $W(s; f,\overline{f})$.
The Rankin--Selberg convolution % chktex 8
$L(s, f\times f)/\zeta(2s)$ has a pole at $s = 1$.
This pole must cancel with poles from $Z(s, 0, f\times \overline{f})$, as otherwise the
methodology of this chapter contradicts~\eqref{eq:CN_compare}.
Stated differently, we must have that the leading contribution of the diagonal term
cancels perfectly with a leading contribution from the off-diagonal,
\begin{equation*}
  \Res_{s = 1}\sum_{n \geq 1}\frac{\lvert a(n) \rvert^2}{n^{s + k - 1}} = - \Res_{s =
  1}\sum_{n,h \geq 1} \frac{a(n) \overline{a(n-h)} + \overline{a(n)}a(n-h)}{n^{s + k -
  1}}.
\end{equation*}
We investigate this cancellation further by sketching the arguments of
\S\ref{sec:analyticbehavior} and \S\ref{sec:secondmoment} in greater generality.


The spectral decomposition corresponding to Proposition~\ref{prop:spectralexpansionfull}
is more complicated since we must now use the Selberg Poincar\'{e} series on $\Gamma_0(N)$
\begin{equation}
  P_h(z,s) := \sum_{\gamma \in \Gamma_\infty \backslash \Gamma_0(N)} \Im(\gamma z)^s
  e(h\gamma \cdot z).
\end{equation}
The spectral decomposition of $P_h$ will involve Eisenstein series associated to each cusp
$\mathfrak{a}$ of $\Gamma_0(N)$.
These Eisenstein series have expansions
\begin{equation}
  E_\mathfrak{a} (z,w) = \delta_\mathfrak{a} y^w + \varphi_\mathfrak{a}(0,w) y^{1-w} +
  \sum_{m \neq 0} \varphi_\mathfrak{a} (m,w) W_w(\lvert m\rvert z),
\end{equation}
where $\delta_\mathfrak{a} = 1$ if $\mathfrak{a} = \infty$ and is $0$ otherwise,
\begin{align}
  \varphi(0, w) &= \sqrt \pi \frac{\Gamma(w - \frac{1}{2})}{\Gamma(w)} \sum_c
  c^{-2w}S_\mathfrak{a}(0,0;c) \\
  \varphi(m, w) &= \frac{\pi^w}{\Gamma(w)} \lvert m \rvert^{w-1} \sum_c c^{-2w}
  S_\mathfrak{a}(0, m; c)
\end{align}
are generalized Whittaker-Fourier coefficients,
\begin{equation}
  W_w(z) = 2\sqrt y K_{w - \frac{1}{2}}(2\pi y) e(x)
\end{equation}
is a Whittaker function, $K_\nu(z)$ is a $K$-Bessel function, and
\begin{equation}
  S_\mathfrak{a}(m,n; c) = \sum_{\left(\begin{smallmatrix} a&\cdot \\ c&d
  \end{smallmatrix}\right) \in \Gamma_\infty \backslash \sigma_\alpha^{-1} \Gamma_0(N) /
  \Gamma_\infty} e\left( m \frac{d}{c} + n \frac{a}{c}\right)
\end{equation}
is a Kloosterman sum associated to double cosets of $\Gamma_0(N)$ with
\begin{equation}
  \Gamma_\infty = \left \langle \begin{pmatrix} 1&n \\ &1 \end{pmatrix} : n \in \mathbb{Z}
\right\rangle \subset \SL_2(\mathbb{Z}).
\end{equation}
This expansion is given in Theorem 3.4 of~\cite{iwaniec2002spectral}.



Letting $\mu_j$ be an orthonormal basis of the residual and cuspidal spaces, we may expand
$P_h(z,s)$ by the Spectral Theorem (as presented in Theorem~15.5
of~\cite{IwaniecKowalski04}) to get
\begin{align}
  P_h(z,s) &= \sum_j \langle P_h(\cdot, s), \mu_j \rangle \mu_j(z) \\
           &\quad + \sum_\mathfrak{a} \frac{1}{4\pi} \int_\mathbb{R} \langle P_h(\cdot,
           s), E_\mathfrak{a}(\cdot, \tfrac{1}{2} + it)\rangle E_\mathfrak{a}(z,
           \tfrac{1}{2} + it) \ dt. \label{line:PoincareLevelNContinuous}
\end{align}
This is more complicated than the $\SL_2(\mathbb{Z})$ spectral expansion
in~\eqref{eq:Pspectral} for two major reasons: we are summing over cusps and the
Kloosterman sums within the Eisenstein series are trickier to handle.
Continuing as before, we try to understand the shifted convolution sum
\begin{equation}
  Z(s,w,f\times f) = \frac{(4\pi)^{s + k - 1}}{\Gamma(s + k - 1)}\sum_{h \geq
  1}\frac{\langle \lvert f \rvert^2 \Im(\cdot)^k, P_h\rangle}{h^w}
\end{equation}
by substituting the spectral expansion for $P_h(z,s)$ and producing a meromorphic
continuation.



The analysis of the discrete spectrum is almost exactly the same: it is analytic for $\Re
s > -\frac{1}{2} + \theta$.
The only new facet is understanding the continuous spectrum component corresponding
to~\eqref{line:PoincareLevelNContinuous}.
We expect that the continuous spectrum of $Z(s, 0, f\times \overline{f})$ has leading
poles that perfectly cancel the leading pole of $L(s, f\times \overline{f})
\zeta(2s)^{-1}$.

Using analogous methods to those in Section~\ref{sec:analyticbehavior}, we compute the
continuous spectrum of $Z(s, 0, f\times \overline{f})$ to get
\begin{align}
  &\sum_{h \geq 1}\frac{(4\pi)^{s + k - 1}}{\Gamma(s + k - 1)}\sum_{\mathfrak{a}}
  \frac{1}{4\pi i} \int_{(\frac{1}{2})} \langle P_h(\cdot, s), E_\mathfrak{a}(\cdot,
  t)\rangle \langle \lvert f \rvert^2 \Im(\cdot)^k, \overline{E_\mathfrak{a}(\cdot,
  t)}\rangle \, dt \nonumber \\
  \begin{split}
  & = \frac{(4\pi)^k}{\Gamma(s + k - 1)\Gamma(s)}\sum_{\mathfrak{a}} \int_{-\infty}^\infty
    \left( \sum_{h,c \geq 1} \frac{S_\mathfrak{a} (0, h; c)}{h^{s + it} c^{1 -
    2it}}\frac{\pi^{\frac{1}{2} - it}}{\Gamma(\frac{1}{2} - it)}\right) \times \\
  &\quad \times \Gamma(s - \tfrac{1}{2} + it)\Gamma(s - \tfrac{1}{2} - it) \langle \lvert
    f \rvert^2 \Im(\cdot)^k, \overline{E_\mathfrak{a}(\cdot, \tfrac{1}{2} + it)} \rangle \,
    dt.
  \end{split}
\end{align}
We've placed parentheses around the arithmetic part, including the Kloosterman sums and
factors for completing a zeta function that appears within the Kloosterman sums.


The arithmetic part of the Eisenstein series are classically-studied $L$-functions, and
each satisfies analogous analytic properties to the function denoted $\mathcal{Z}(s,0,z)$
in \S\ref{sec:analyticbehavior}.
We summarize the results of this section with the following theorem.


\begin{theorem}\label{thm:general_weight_level_comparison}
  Let $f$ be a weight $k > 2$ cusp form on $\Gamma_0(N)$. Then
  \begin{align}\label{eq:general_level_general_comparison}
    \Res_{s = 1} \sum_{n \geq 1} \frac{\lvert a(n) \rvert^2}{n^{s + k - 1}} = - \Res_{s =
    1} \sum_{n,h \geq 1} \frac{a(n)\overline{a(n-h)}}{n^{s + k - 1}}- \Res_{s = 1}
    \sum_{n,h \geq 1} \frac{\overline{a(n)} a(n-h)}{n^{s + k - 1}},
  \end{align}
  or equivalently
  \begin{align}\label{eq:general_level_kloosterman}
    -\frac{1}{2}\Res_{s = 1} &\frac{L(s, f\times f)}{\zeta(2s)} =\\
    &= \Res_{s = 1}\sum_{\mathfrak{a}} \frac{1}{4\pi} \int_\mathbb{R} \langle P_h(\cdot,
s), E_\mathfrak{a}(\cdot, \tfrac{1}{2} + it)\rangle \langle \lvert f \rvert^2
\Im(\cdot)^k, \overline{E_\mathfrak{a}(\cdot, \tfrac{1}{2} + it)}\rangle \, dt \\
    \begin{split}
      &= \Res_{s = 1} \frac{(4\pi)^k}{\Gamma(s + k - 1)\Gamma(s)}\sum_{\mathfrak{a}}
      \int_{-\infty}^\infty  \left( \sum_{h,c \geq 1} \frac{S_\mathfrak{a} (0, h; c)}{h^{s
      + it} c^{1 - 2it}}\frac{\pi^{\frac{1}{2} - it}}{\Gamma(\frac{1}{2} - it)}\right) \\
      &\quad \times \Gamma(s - \tfrac{1}{2} + it)\Gamma(s - \tfrac{1}{2} - it) \langle
      \lvert f \rvert^2 \Im(\cdot)^k, \overline{E_\mathfrak{a}(\cdot, \tfrac{1}{2} + it)}
      \rangle \, dt.
    \end{split}
  \end{align}
\end{theorem}

% vim: tw=90
