
In this chapter, we highlight some applications of the Dirichlet series $D(s, S_f \times
\overline{S_g})$ and $D(s, S_f \times S_g)$.
We begin with completed applications, summarizing the results and methodology.
Towards the end of the chapter, we highlight ongoing and future work.


\vspace{-1in}
\subsection*{\center{Reacknowledgements to my Collaborators}}

\begin{quote}
  Each of the applications in this chapter were carried out (or are being carried out)
  with Alex Walker, Chan Ieong Kuan, and Tom Hulse, each of whom are my academic brothers,
  collaborators, and friends.
\end{quote}




\section{Applications}


The Dirichlet series $D(s, S_f \times \overline{S_g})$ and $D(s, S_f \times S_g)$ present
a new avenue for investigating the behavior of $S_f$, $S_f(n)S_g(n)$, and related objects.
As $S_f$ is analogous to the error term in the Gauss Circle Problem (cf.
Chapter~\ref{c:introduction}), it is perhaps most natural to ask about the sizes of
$S_f(n)$.



\subsection*{Long Sums}


One result of this form was presented in \S\ref{c:sums}, in
Theorem~\ref{thm:second_moment_Sf} giving smoothed averages for $S_f(n)\overline{S_g(n)}$
and, if $f = g$, smoothed averages for $\lvert S_f(n) \rvert^2$.
This theorem was proved completely for $f$ and $g$ on level $1$.


In~\cite{hkldw}, my collaborators and I analyze further smoothed asymptotics for
$S_f(n)\overline{S_g(n)}$ for forms $f$ and $g$ on $\Gamma_0(N)$ for $N$ squarefree,
proving analogous results to Theorem~\ref{thm:second_moment_Sf} for forms even of
half-integral weight.
When $f = g$, this is a generalized smoothed analogue to the result of
Cram\'er~\cite{cramer1922} giving that
\begin{equation}
  \frac{1}{X} \int_0^X \Big \lvert \sum_{n \leq t} r_2(n) - \pi t \Big \rvert^2 dt = c
  X^{1/2} + O(X^{\frac{1}{4} + \epsilon}),
\end{equation}
giving evidence towards a generalized Circle problem.


It is very interesting (and very new) that if $f \neq g$, then $S_f S_g$ still appears to
satisfy a generalized Circle problem.
Recalling Chandrasekharan and Narasimhan's result that $S_f(X) =
\Omega_{\pm}(X^{\frac{k-1}{2} + \frac{1}{4}})$, it's apparent that both $S_f(n)$ and
$S_g(n)$ oscillate in size between $\pm n^{\frac{k-1}{2} + \frac{1}{4}}$.
From first principles alone, it seems possible that $S_f(n)$ might be small or negative
with $S_g(n)$ is large and positive, leading to large cancellation in average sums $\sum
S_f(n) S_g(n)$.
But Theorem~\ref{thm:second_moment_Sf} indicates that there is not large cancellation of
this sort.
Indeed, Theorem~\ref{thm:second_moment_Sf} roughly indicates that the partial sums of
Fourier coefficients of $f(z)$ correlate about as well with the partial sums of Fourier
coefficients of $g(z)$ as with itself, up to the constant $L(\frac{3}{2}, f \times g)$ of
proportionality.


By mimicking the techniques of Chapter~\ref{c:sums}, it is possible to apply different
integral transforms to $D(s, S_f \times \overline{S_g})$ or $D(s, S_f \times S_g)$, either
to get different long-average estimates, or estimates of a different variety.



\subsection*{Short-Interval Averages}


Now restrict attention to $f$ a full-integer weight cusp form on $\SL(2, \mathbb{Z})$, and
suppose that $f = g$.
In~\cite{hkldwShort}, my collaborators and I analyze short-interval estimates of the type
\begin{equation}\label{eq:short-interval-estimate}
  \frac{1}{X^{2/3}(\log X)^{1/6}} \sum_{\lvert n-X \rvert < X^{2/3} (\log X)^{1/6}} \lvert
  S_f(n) \rvert^2 \ll X^{k-1 + \frac{1}{2}},
\end{equation}
which says essentially that the Classical Conjecture holds \emph{on average} over short
intervals of width $X^{\frac{2}{3}} (\log X)^{\frac{1}{6}}$ around $X$.
This is qualitatively a  much stronger result than the long-interval estimate, and is a
vast improvement over previously known results.


It is interesting to note that it is possible to get bounds for individual sums $S_f(n)$
from short-interval estimates.


\begin{proposition}
  Suppose that
  \begin{equation}
    \frac{1}{X^w} \sum_{\lvert n - X \rvert < X^w} \lvert S_f(n) \rvert^2 \ll X^{k - 1 +
    \frac{1}{2}}
  \end{equation}
  for $w \geq \frac{1}{4}$.
  Then also
  \begin{equation}
    S_f(X) \ll X^{\frac{k-1}{2} + \frac{1}{4} + (\frac{w}{3} - \frac{1}{12})}.
  \end{equation}
\end{proposition}


\begin{proof}
  We only sketch the proof.
  For each individual Fourier coefficient $a(n)$, we have Deligne's bound $a(n) \ll
  n^{\frac{k-1}{2}}$.
  Suppose there is an $X$ such that $S_f(X) \gg X^{\frac{k-1}{2} + \frac{1}{4} + \alpha}$.
  Then $S_f(X + \ell) \gg X^{\frac{k-1}{2} + \frac{1}{4} + \alpha}$ for $\ell <
  X^{\frac{1}{4} + \alpha - \epsilon}$ for any $\epsilon > 0$, as it takes approximately
  $X^{\frac{1}{4} + \alpha}$ coefficients $a(n)$ to combine together to cancel $S_f(X)$.
  But then
  \begin{equation}
    \frac{1}{X^w} \sum_{\lvert n - X \rvert < X^w} \lvert S_f(X) \rvert^2 \gg
    \frac{1}{X^w} X^{k-1 + \frac{1}{2} + 2\alpha} X^{\min(w, \frac{1}{4} + \alpha)}.
  \end{equation}
  The $X^{\min(w, \frac{1}{4} + \alpha)}$ term comes from the width of the interval where
  each $S_f(X + \ell)$ is approximately $X^{\frac{k-1}{2} + \frac{1}{4} + \alpha}$.
  Comparing exponents of $X$ with
  \begin{equation}
    \frac{1}{X^w} \sum_{\lvert n - X \rvert < X^w} \lvert S_f(n) \rvert^2 \ll X^{k - 1 +
    \frac{1}{2}}
  \end{equation}
  shows that
  \begin{equation}
    2\alpha + \min(w, \tfrac{1}{4} + \alpha) \leq w,
  \end{equation}
  from which either $\alpha = 0$ or $\alpha < \tfrac{1}{3}( w - \tfrac{1}{4})$.
\end{proof}


\begin{corollary}
  A short-interval estimate of the type~\eqref{eq:short-interval-estimate}, with interval
  $\lvert n-X \rvert \leq X^{\frac{1}{4} + \epsilon}$, would prove the Classical
  Conjecture.
\end{corollary}

The short-interval estimate~\eqref{eq:short-interval-estimate} only produces the
individual estimate $S_f(X) \ll X^{\frac{k-1}{2} + \frac{1}{4} + \frac{5}{36}}$, which is
$1/18$ worse than the current best-known individual bound $S_f(X) \ll X^{\frac{k-1}{2} +
\frac{1}{3}}$.
On the other hand, it is by far the strongest short-interval average estimate.
(Note that the individual bound $S_f(X) \ll X^{\frac{k-1}{2} + \frac{1}{3}}$ doesn't give
a Classical Conjecture on average type result for any interval length).


\begin{remark}
  It is possible to use the short-interval estimate~\eqref{eq:short-interval-estimate},
  along with an argument used in Chapter~\ref{c:hyperboloid} which makes use of multiple
  integral transforms in combination, to recover the Hafner-Ivi\'c tyle bound $S_f(n) \ll
  n^{\frac{k-1}{2} + \frac{1}{3}}$.
\end{remark}


To prove the estimate~\eqref{eq:short-interval-estimate}, one builds upon the meromorphic
information of $D(s, S_f \times \overline{S_f})$ and applies an integral transform
\begin{equation}
  \frac{1}{2\pi i} \int_{(4)} D(s, S_f \times \overline{S_f}) \exp{\Big(\frac{\pi
  s^2}{y^2} \Big)} \frac{X^s}{y} ds
\end{equation}
to understand (essentially) a sum over the interval $\lvert n - X \rvert < X/y$.
For more details and analysis, refer to~\cite{hkldwShort}.


\subsection*{Sign-Changes of Sums of Coefficients of Cusp Forms}

As noted in Chapter~\ref{c:motivations}, the original guiding question that led to the
investigation of $D(s, S_f \times \overline{S_g})$ was concerning the sign changes within
the sequence $\{S_f(n)\}_{n \in \mathbb{N}}$.
In~\cite{hkldwSigns}, my collaborators and I succeeded in answering our original guiding
question.\footnote{The story behind the scope of this paper is a bit interesting, as it
  was written at the same time as~\cite{hkldwShort}.
  Tom Hulse had learned of a set of criteria guaranteeing sign changes from a paper of Ram
  Murty, and he first thought of how to generalize the criteria to sequences of real and
  complex coefficients.
  I had thought it was possible to further generalize towards smoothed sums, but we failed
  in this regard.
  This morphed into our generalization, stating how to translate from analytic properties
  of Dirichlet series directly into sign-change results of the coefficients.
We each focused on the parts that interested us most: I was interested in $S_f(n)$ and
$S_f^\nu(n)$ (defined below), Hulse was interested in real and complex coefficients, and
Keong was interested in $\GL(3)$.
In the end, each aspect strengthened the overall paper and led to a nice unified
description of somewhat disjoint sign-change results.}

In this paper, we proved a veritable cornucopia of results concerning the sign changes of
coefficients and sums of coefficients of cusp forms on $\GL(2)$ and $\GL(3)$.
Here, I emphasize what I focused on, and what follows most naturally from the
considerations of Chapter~\ref{c:sums}.

Let $f$ be a weight $0$ Maass form or a holomorphic cusp form of full or half-integer
weight $k$ on a congruence subgroup $\Gamma \subseteq \SL(2, \mathbb{Z})$, possibly with
nontrivial nebentypus.\index{sign change results}
Write $f$ as either
\begin{equation}
  f(z) = \sum_{n \neq 0} A_f(n) \sqrt{y} K_{it_j} (2\pi \lvert y \rvert n) e^{2\pi i n x}
\end{equation}
if $f$ is a Maass form with eigenvalue $\lambda_j = \frac{1}{4} + t_j^2$, or
\begin{equation}
  f(z) = \sum_{n \geq 1} A_f(n)n^{\frac{k-1}{2}} e^{2 \pi i n z}
\end{equation}
if $f$ is a holomorphic cusp form.
We write the coefficients of $f$ as $a_f(n) = A_f(n)n^{\kappa(f)}$ with
\begin{equation}
  \kappa(f) = \begin{cases}
    \frac{k-1}{2} & \text{if } $f$ \; \text{is a holomorphic cusp form}, \\
    0 & \text{if } $f$ \; \text{is a Maass form},
  \end{cases}
\end{equation}
Then $n^{\kappa(f)}$ conjecturally normalizes the coefficients $A_f(n)$ correctly
depending on whether $f$ is a holomorphic cusp form or a Maass form.
It is expected that $A_f(n) \ll n^\epsilon$, but in general it is only known that
\begin{equation}
  A_f(n) \ll n^{\alpha(f) + \epsilon}
\end{equation}
where
\begin{equation}
  \alpha(f) = \begin{cases}
    0 & \text{if f is full-integral weight holomorphic}, \\
    \frac{3}{16} & \text{if f is half-integral weight holomorphic}, \\
    \frac{7}{64} & \text{if f is a Maass form}.
  \end{cases}
\end{equation}
Define the partial sums of normalized coefficients $S_f^\nu(n)$ as
\index{S@$S_f^\nu(n)$}
\begin{equation}
  S_f^\nu(n) = \sum_{m \leq n} \frac{a_f(m)}{m^\nu}
\end{equation}
Then by studying the Dirichlet series $D(s, S_f^\nu, \overline{S_g^\nu})$ and $D(s,
S_f^\nu)$ in~\cite{hkldwSigns}, we proved the following theorem.


\begin{theorem}
  Let $f$ be a weight $0$ Maass form or holomorphic cusp form as described above.
  Suppose that $0 \leq \nu < \kappa(f) + \frac{1}{6} - \frac{2 \alpha(f)}{3}$.
  If there is a coefficient $a_f(n)$ such that $\Re a_f(n) \neq 0$ (resp. $\Im a_f(n) \neq
  0$), then the sequence $\{\Re S_f^\nu(n)\}_{n \in \mathbb{N}}$ (resp. $\{\Im
S^\nu_f(n)\}_{n \in \mathbb{N}}$) has at least one sign change for some $n \in [X, X +
X^{r(\nu)}]$ for $X \gg 1$, where
  \begin{equation}
    r(\nu) = \begin{cases}
      \frac{2}{3} + \frac{2\alpha(f)}{3} + \epsilon & \text{if } \nu < \kappa(f) +
      \frac{\alpha(f)}{3} - \frac{1}{6}, \\
      \frac{2}{3} + \frac{2\alpha(f)}{3} + \Delta + \epsilon & \text{if } \nu = \kappa(f)
      + \frac{\alpha(f)}{3} - \frac{1}{6} + \Delta, 0 \leq \Delta < \frac{1}{3} - \frac{2
      \alpha(f)}{3}.
    \end{cases}
  \end{equation}
\end{theorem}


In other words, we showed high regularity of the sign changes of sums of normalized
coefficients, depending on the amount of normalization.
As should be expected, higher amounts of normalization lead to fewer guaranteed sign
changes.


However, we show that it is possible to take $\nu$ slightly larger than $\kappa(f)$, so
that the individual coefficients $a_f(n)/n^\nu$ are each decaying in size.
For example, for full-integer weight holomorphic cusp forms, we can take
\begin{equation}
  \nu = \frac{k-1}{2} + \frac{1}{6} - \epsilon
\end{equation}
and guarantee at least one sign change in $\{S_f^\nu(n)\}_{n \in \mathbb{N}}$ for some $n$
in $[X, 2X]$ for sufficiently large $X$.
Yet for this normalization, we have
\begin{equation}
  S_f^\nu(n) = \sum_{m \leq n} \frac{a_f(m)}{m^{\frac{k-1}{2} + \frac{1}{6} - \epsilon}},
\end{equation}
so that the coefficients are \emph{decaying} and look approximately like $n^{-1/6}$.
It is a remarkable fact that the coefficients are arranged in such a way that there are
still infinitely many sign regularly-spaced sign changes even though they are
\emph{over-normalized}.

This suggests a certain regularity of the sign changes of individual coefficients
$a_f(n)$, but it is challenging to describe the exact nature of this regularity.



\section{Directions for Further Investigation: Non-Cusp Forms}



In the investigations carried out thus far, we have taken $f$ to be a cusp form.
But one can attempt to perform the same argument on sums of coefficients of noncuspidal
automorphic forms.


One particular example would be to consider sums of the form
\begin{equation}
  S_{\theta^k}(n) = \sum_{m \leq n} r_k(m),
\end{equation}
where $r_k(m)$ is the number of ways of representing $m$ as a sum of $k$ squares.
This is equivalent to the Gauss $k$-dimensional sphere problem, which asks how many
integer lattice points are contained in a $k$-dimensional sphere of radius $\sqrt{n}$?


My collaborators and I have been focusing our attention on this problem, and we expect to
have a result by mid 2017.


It is natural to consider a general case of non-cuspidal automorphic forms.
The Gauss $k$-dimensional sphere problem suffers from the fact that the underlying
automorphic form is of half-integral weight for odd $k$.


\begin{remark}
  There are limitations to this technique.
  We can only consider forms $f$ for which we understand the shifted convolution sum
  coming from $f \times f$ sufficiently well.
  So we are not capable of understanding sums of coefficients of Maass forms at this time,
  since shifted convolution sums of Maass forms with Maass forms remain mysterious.
\end{remark}


% vim: tw=90 cc=90
